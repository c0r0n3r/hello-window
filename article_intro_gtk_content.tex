%FIXME: mi az, honnan jön, mire érhető el, ...

\section{A függvénykönyvtár moduljai}

\subsection{GTK+}

Elöljáróban fontos lehet tudni a nyelvet megvalósító implementáció szervezéséről, hogy példaértékűen választja szét a funkcionalitás egyes elemeit több különálló, jól elhatárolt részegységre, melyek ugyan támaszkodnak egymásra, de a megoldás nagyban elősegíti a rugalmasságot és a portabilitást.

\subsubsection{GLib}
\index{GLib@\textit{GLib}}

A GLib maga egy önálló függvénykönyvtár, mely \textit{GTK}-tól független, de javarészt az által is használt hasznos --a \textit{C} nyelvi elemként nem létező-- programozói segédeszközt tartalmaz. Ezek közül a leggyakrabban használtakat említjük meg. Használatuk részleteiről a későbbi részekben esik majd szó.

\paragraph{Alapvető eszközök}

Számos --a platformfüggetlen programozás szempontjából-- fontos eszközt bocsát a \textit{GLib} rendelkezésünkre, melyek jelentékeny részére amúgy is definiálnánk, így viszont készen apjuk őket.

\begin{description}
 \item[protábilis típusdefiníciók] \texttt{guint32}, \texttt{guintptr}, \dots
 \item[típusok határértékei] \texttt{G\_MININT32}, \texttt{G\_MAXDOUBLE}, \dots
 \item[általános célú makrók] \texttt{MIN}/\texttt{MAX}, \texttt{TRUE}/\texttt{FALSE}, \texttt{G\_CONST\_RETURN}, \dots
 \item[típuskonverziós makrók] \texttt{GPONITER\_TO\_INT}, \texttt{GSIZE\_TO\_POINTER}, \dots
 \item[byte-order konvenciós makrók] \texttt{g\_htonl}, \texttt{GSIZE\_FROM\_BE}, \texttt{GUINT32\_SWAP\_BE\_LE}, \dots
 \item[matematika konstansok] \texttt{G\_E}, \texttt{G\_LN2}, \texttt{G\_PI}, \dots
 \item[fordítási opció makrók] \texttt{G\_GNUC\_NULL\_TERMINATED}, \texttt{G\_GNUC\_MALLOC}, \dots
 \item[atomikus műveletek] \texttt{g\_atomic\_inc\_int}, \texttt{g\_atomic\_pointer\_or}, \dots
\end{description}

\paragraph{Alkalmazásfejlesztési eszközök}
%FIXME
\paragraph{Hasznos segéseszközök}
%FIXME
\paragraph{Adattípusok}
%FIXME

%\paragraph{Alkalmazásfejlesztési támogatás} Szálak, aszinkron kommunikáció a szálak között, dinamikus modulbetöltés, memória-, pipe-, socket-, fájlkezelés, többszintű logolás.
%\paragraph{Konverziós eszközök} Karakterlánc-, dátum-, idő-, karakterkonverziós eszközök, parancssori paraméterek, XML, .ini, bookmark fájlok feldolgozása, időzítők, reguláris kifejezések.
%\paragraph{Adattípusok} Láncolt listák, fák, asszociatív tömbök, szekvenciák, sorok, dinamikusan méretezhető tömbök.
%\paragraph{Objektum rendszer} A \textit{GObject}, mely a \textit{GLib} része minden widget ``ősosztálya'', ugyanakkor saját eszközök alapkövévé is thetjük, hisz egyebek mellett lehetővé tesz referencia számlálást, futásidejű típusellenőrzést, tulajdonságok hozzárendelését és azok értékének futásidőben történő változtatását.

\subsubsection{GDK}
\index{GDK@\textit{GDK}}

A \textit{GIMP Drawing Kit} (\textit{GDK}) az alacsony szintű rajzolási és ablakkezelési feladatok megvalósítására és egyszersmind elfedésére szolgáló függvénykönyvtár. Fontos szerepet tölt be a \textit{GTK} különböző platformok közötti hordozhatóságának megteremtésében., hiszen az általa nyújtotta viszonylag szűk körű funkcionalitást újraimplementálva a \textit{GTK} alkalmas lehet egy újabb grafikus környezetben való futásra, legalábbis amia rajzolási feladatokat illeti, az egyéb megoldandó problémák java részét a korábban mér említett \textit{GLib} oldja meg.

Fentieknek köszönhető, hogy az eredetileg csak az \textit{X Window System} --ami a \textit{Linux} alapú rendszerek a \textit{GDK} születésének idejében kizárólagos grafikus szervere-- feletti működni képes \textit{GDK}, mára a \textit{Windows}, \textit{Mac OS X} rendszereken túl akár egy megfelelő webböngészőben is képes futni\footnote{ezen szolgáltatáshoz eléréséhez \textit{websocket} támogatásra van szükség a böngészőben, illetve a \textit{broadway} elnevezésű \textit{backend} bekapcsolására a \textit{GTK+} fordításakor}.

\subsubsection{Cairo}
\index{cairo@\textit{cairo}}

A \textit{cairo} egy eszközfüggetlen kétdimenziós vektorgrafikai függvénykönyvtár, melyet kifejezetten a hardveres gyorsítókkal való együttműködésre terveztek, s mellyel a \textit{GDK} a rajzolási feladatait végzi. Érdemes megemlíteni, hogy a \textit{cairo} nem a \textit{GNOME}, hanem a \href{http://freedesktop.org}{\textit{freedesktop.org}} projekt része.

\subsubsection{Pango}
\index{Pango@\textit{Pango}}

A \textit{Pango} szövegek képi formában történő előállításáért (\textit{rendering}) és megjelenítésért (\textit{lay out}) felelős a \textit{GTK}-n belül, de természetesen a \textit{GTK}-tól függetlenül is használható, lévén a függvénykönyvtár az előbbiekhez hasonlóan számos platformot támogat.


\subsection{gtkmm}
\index{gtkmm@\textit{gtkmm}}

A \textit{gtkmm}, illetve annak függőségei adják a \textit{GTK} projekt \textit{C++} nyelvű változatát. Ezek a függvénykönyvtárak nem egyebek, mint wrapperek az eredeti \textit{C} változat fölött, az ebből fakadó előnyökkel és korlátokkal együtt. Ezen kódok jelentékeny része wrapper mivoltukból következően generált, ugyanakkor számos helyen --ahol ez funkcionalitáshoz a programozási nyelvhez leginkább illeszkedő megvalósításához szükséges-- eredeti kódot is tartalmaz.

A \textit{C}, illetve \textit{C++} nyelvű változatok a lehető legkisebb mértékben térnek el egymástól. Ez egyben azt is jelenti, hogy az egyes nyelvi változatok nem tartalmaznak a többihez képest többlet funkcionalitást. Nem lehet azonban eltekinteni az egyes programozási nyelvek adta lehetőségek előnyeitől, hátrányaitól, melyek könnyebbé vagy nehezebbé teszik a \textit{GTK} adott nyelven való használatát.

\subsubsection{Libsigc++}
\index{libsigc++@\textit{libsigc++}}

A \textit{gtkmm} implementációjánál használt függvénykönyvtár, ami lehetővé teszi a szignálkezelés típusbiztos megvalósítását, mely a \textit{C} változatnál --a nyelvi sajátosságokból fakadóan-- nem adott.

\subsection{PyGTK}
\index{PyGTK@\textit{PyGTK}}
%FIXME


\subsection{Dogtail}
\index{dogtail@\textit{dogtail}}
%FIXME

\section{A megvalósítás koncepciója}

Ebben a fejezetben arra próbálunk rávilágítani, hogy a \textit{GTK+} ugyan \textit{C} nyelven íródott, de számos -- az objektum-orientált -- nyelvek esetén megszokott terminológiát használ, sőt ezeket a nyelvi eszközök adta mértékben meg is valósítja. Az \textit{OOP} kifejezéseit ezért tudatosan használom az olyan esetekben is, ahol \textit{GTK+} nyelvű fejlesztésről esik szó.

\subsection{Öröklődés}
 \index{GObject@\texttt{GObject}}

Annak ellenére, hogy a mechanizmust nyelvi szinten a \textit{C} nem, csak a \textit{C++} támogatja lehetséges objektum-orientált megközelítéssel élni az előbbi esetben is. Erre kitűnő példa -- egyebek mellett -- a \textit{GTK+}. Megoldott a \textit{widget}ek egymásból történő származtatása, sőt felhasználói widgetek is definiálhatóak a már meglévőekre támaszkodva. Jól mutatja ez, hogy a \texttt{GObject} osztály minden \textit{widget}, illetve más a \textit{GTK}-ban használt nem vizuális elem őse. Meg kell jegyezni, hogy a \textit{gtkmm} esetén -- lévén ott a nyelv \textit{C++} -- természetesen a származtatás egy nagyságrenddel egyszerűbb, de a letölthető példákat felhasználva némi rutinnal a \textit{GTK+} esetén sem igényel különösebb erőfeszítést.

\subsection{Típusbiztosság}
 \index{GObject@\texttt{GObject}}

Hasonlóan az öröklődéshez -- pusztán nyelvi szinten -- ez az eszköz sem megvalósítható (\textit{C} esetén), ugyanakkor a \textit{GTK+} minden \textit{widget}típushoz -- mondhatni osztályhoz -- definiál egy-egy makrót, melyek segítségével, fordítási időben (compile time) ugyan nem, de futásidőben (run time) ellenőrizhető egy adott widget valódi típusa, hasonlóan ahhoz, mint amire a \textit{dynamic\_cast} használata jelent a \textit{C++}-ban. Azt a mechanizmust, melynek révén lehetővé válik a \textit{GTK+}-ban a futás idejű típusellenőrzés, a már említett \texttt{GOject} osztály implementálja.

\subsection{Signal}

Ezen osztályon keresztül valósul meg a szignálkezelés, mely lehetővé teszi adott eseményekhez kezelőfüggvények (\textit{callback}) kapcsolását. Itt kell megjegyezni, hogy mivel az object nem csupán a \textit{widget}eknek őse, így olyan \textit{GTK}-s, vagy akár saját, elemeknek is lehetnek szignáljai, melyek közvetlenül nem láthatóak, mint az általunk létrehozott \textit{GUI} részei.

\subsection{Referencia-számlálás}

Minden \textit{object}ből származó osztály, így a widgetek is, rendelkeznek referencia-számmal, mely tulajdonképpen azt fejezi ki, hogy hányan hivatkoznak az adott elemre. A \textit{GTK}, pontosabban ez esetben a \textit{GLib} lebegő referenciát (\textit{floating reference}) alkalmaz, mely azt jelenti, hogy az objektum létrejöttekor annak referenciája 1 lesz, de ezt a referenciát úgymond nem birtokolja senki, azaz ha a \textit{widget}et egy konténer osztályba (melyekről részletesebben a következő rész szól majd) kívánjuk tenni, akkor -- az első ilyen alkalommal -- a refernciaszám nem nő, annak ellenére sem, hogy ez valójában hivatkozást jelent az adott elemre. A változatlanul hagyott referencia-érték jelenti a lebegő referencia elsüllyesztését (\textit{sink}). Minden azt követő esetben a konténerből történő eltávolítás csökkenti, ahoz való hozzáadás pedig növeli a referencia értékét. Érdemes felhívni a figyelmet arra, hogy az elmondottak alapján, ha hozzáadtuk \textit{widget}ünket egy \textit{container}hez, majd pedig eltávolítjuk belőle azt, akkor annak referenciája 0-ra csökken, ami maga után vonja a destruktor lefutását. Ezt elkerülhetjük, ha az eltávolítás előtt explicit módon növeljük a referenciát, amit aztán csökkentenünk kell, ha egy másik osztály ``birtokába'' adjuk a \textit{widget}et.

\section{Programozási nyelv}

%FIXME: python, perl, ... néhány mondat
Megfeledkezni azonban nem lehet a tényről, hogy a \textit{C++} fordító közel sem olyan gyakran áll rendelkezésre, mint azt gondolnánk. Számos olyan terület létezik ugyanis, melyek esetén alapvető megkötés a \textit{C} nyelv. Megemlítendő továbbá, hogy a \textit{gtkmm} nem az egyetlen port, hiszen létezik többek között \textit{python} (\textit{pyGTK}), \textit{perl} (gtk2-perl), \textit{ruby}, \textit{java} nyelvű változat is.

Mielőtt a kódolás technikai részleteire rátérnénk, tegyünk egy rövid kitérőt a tisztánlátás érdekében és hasonlítsuk össze a két felületprogramozási nyelv alapvető tulajdonságait:\vspace{16pt}

\begin{table}[H]
\begin{center}
\begin{tabular}[t]{l c c}
                                       & \textbf{\textit{GTK+}}            & \textbf{\textit{gtkmm}}  \\\\
\textbf{Felhasználási terület:}        & GUI fejlesztés                    & GUI fejlesztés           \\\\
\textbf{Implementáció nyelve:}         & C                                 & C++                      \\\\
\textbf{Implementáció módja:}          & natív                             & wrapper                  \\\\
\textbf{Objektumorinentált}            &                                   &                          \\
\textbf{technikák használata:}         & közvetett                         & natív                    \\\\
\textbf{Típusellenőrzés:}              & futási időben                     & fordításai dőben         \\\\
\textbf{Licenc:}                       & LGPL                              & LGPL                     \\\\
\textbf{Ismertebb projektek:}          & GNOME,                            & GNOME,                   \\
                                       & Evolution, Firefox, Gimp, \dots   & GParted, Inkscape, \dots \\
\end{tabular}
\caption{A \textit{GTK+} és \textit{gtkmm} tulajdonságainak összehasonlítása}
\end{center}
\end{table}
%FIXME

Ahogy az a fentiekből is látszik a a \textit{C++} nyelvű változatnak megvannak a maga komoly előnyei. Ezek közül talán a legfontosabb, hogy a nyelv nyújtotta módszereket, mint például az örököltetés, itt közvetlenül használhatjuk ki. Ez természetesen nem jelenti azt, hogy a \textit{GTK+} esetén erre ne lenne lehetőségünk, ugyanakkor meg kell jegyezni, hogy míg a \textit{gtkmm} esetén ez játszi könnyedséggel megtehető, addig az eredeti változat alkalmazásával ez kissé körülményes. Egy másik megfontolandó érv a típusbiztosság, melynek előnyeit nem lehet eléggé hangsúlyozni, hiszen hosszú órákon át tartó hibakereséstől óvhat meg minket. A \textit{GTK+} a \textit{GObject} révén --melyre egy későbbi részben részletesebben is foglalkozunk majd-- rendelkezik egy frappáns mechanizmussal, mely lehetővé teszi a futásidejű típusellenőrzést a \textit{widget}ek esetén, de nem vetekedhet a \textit{C++} nyújtotta fordítási idejű hibaüzenetekkel.

\section{Fejlesztési sajátosságok}

\subsection{Projektek különbözőségéből adódó eltérések}

A legtöbb nyílt forrású projekt esetén igaz, hogy a kódolás és kódszervezés során egy meghatározott konvenciót követnek, ám amint ez a korábbi példából is látszik, egy olyan méretű projekt esetén, mint a \textit{GNOME} az egyes részterületeken lehetnek eltérések.

\paragraph{Forráskód formázása}

A \textit{GTK+} fejlesztői a \textit{GNU coding standard} irányelveit alkalmazzák, míg a \textit{gktmm} ettől minimális mértékben eltérő változat az alkalmaz.

\paragraph{Fejlécfájlok helye}

Mint az későbbiekben megfigyelhető lesz a \textit{GTK+} \textit{header}ek esetén a fájloknak van egy \textit{gtk} prefixe, ugyanakkor a \textit{gtkmm}-nél ez nem használatos. Ez egyébiránt illeszkedik a nevezéktannál tapasztaltakkal.

\subsection{Programozási nyelv okozta sajátosságok}

\paragraph{Nevezéktan}

A \textit{GTK+} minden saját makrót/függvényt \texttt{GTK}/\texttt{gtk} prefixszel lát el (\textit{GLib} esetén ez pusztán csak egy kis/nagy \texttt{g} betű), sőt egy adott részterület --például egy \textit{widget}-- saját ``névterülettel'' is rendelkezhet, azaz újabb prefixet vezethet be. Ezeket egymástól, illetve a ``valódi'' funkciót jelölő nevektől \texttt{\_} (aláhúzás) jellel választják el. A C++ wrapper kódát olvasva láthatjuk, hogy kihasználva a kézenfekvő nyelvi lehetőséget, a prefixek szerepét a névterek veszik át, annyi különbséggel, hogy ezek neveiben csak az első betű nagy.

\paragraph{Nyelvhez kötődő módszerek}

A mintapéldában megfigyelhető (az azonos funkciójú sorok párba állítása révén), hogy a \textit{C++} kód némileg egyszerűbb, hiszen az ablakunk létrehozásakor nem kell a window típusára (\textit{toplevel}) vonatkozó paramétert megadnunk, azonos helyen lehet a változó deklarációja és első felhasználása. Külön is említésre méltó, bár a \textit{C++} programozók számára nem meglepő, hogy a main függvényből való kilépéskor a \textit{gtkmm} változat windowja felszabadul, míg a \textit{GTK+} verzió esetén ez csak azért történik meg, mert magából a programból is kiléptünk egyúttal.


