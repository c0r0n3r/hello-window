\section{Bevezetés}

Jelen rész a \textit{GTK+} . Ehhez egy új példprogramot veszünk górcső alá, mely forráskódját tekintve némiképp ugyan bonyolultabb az előző részben tárgyalttól, működésére nézve azonban nem sokban különbözik tőle.

\subsection{Fogalmak}

A korábban már megismert alapfogalmak újra előkerülnek majd, most azonban általános ismertetésük helyett a jelen témánkhoz kapcsolódó specifikumaikat vázoljuk fel.

\subsubsection{Main Loop}

Ahogy arról az előző részekben szó esett -- minden más felületprogramozási nyelvekhez hasonlóan -- a \textit{GTK} is eseményvezérelt (\textit{event-driven}). De mit is jelent ez?  Azt, hogy felhasználói interakciók híján -- figyelmen kívül hagyva az ütemezett eseményeket és néhány későbbi részekben részletezendő funkciót -- a \textit{GTK} a \textit{main loop}ban áll és várakozik valamiféle eseményre (\textit{event}), mint az egér megmozdítása, vagy egy kattintás, esetleg egy billentyűleütés. A \textit{main loop}ba történő belépésre szolgál \textit{GTK+} esetén a \texttt{gtk\_main()}, míg \textit{gtkmm} esetén a \texttt{Gtk::Main::run()} függvény. Ha az említett akciók közül valamelyik bekövetkezik, az addig várakozó ciklus úgymond ``felébred'', majd a bekövetkezett eseményt propagálja -- kvázi üzenetet küld -- a megelelő \textit{widget}(ek) felé.

\subsubsection{Signal}

Ez a közvetítő üzenet (jel, jelzés) a szignál. Ilyen üzenetből számos létezik, hisz az egyes \textit{widget}típusokon különböző események lehetnek értelmezettek, gomb esetén a rá történő kattintás, egy legördülő menünél az egér menüelem fölé történő mozgatása, míg egy \textit{widget}nél annak átméretezése. Minden ilyen eseménynek megvan a saját neve, mellyel hivatkozni lehet rá (pl.: ``button-pressed'', ``enter-notify-event``, ''size-request``). Itt érdemes megjegyezni, hogy a \textit{szignál}ok örklődnek, azaz egy specifikus \textit{widget}, mint amilyen mondjuk egy \textit{RadioButton}, vagy egy \textit{CheckButton}, minden olyan szignállal rendelkezik, amivel őse a \textit{Button}, vagy akár annak az őse az a \textit{Widget} típus. A \textit{szignál}ok egyrészről arra szolgálnak, hogy a \textit{GTK} rendszerén belül az egyes \textit{widget}ek egymással kommunikálhassanak. Ha például egy gombot lenyomunk, akkor azt (illetve annak részeit) újra kell rajzolni, ha egy menüelemet kiválasztunk, azt át kell színezni, illetve az esetleges almenüpontokat ki kell rajzolni, míg átméretezésnél az egyes \textit{widget}ek helyigényét újra kell számolni. Másfelől ha a program írói valamely esemény bekövetkezéséről értesülni szeretnének, megadhatnánk eseménykezelő függvényeket, melyek ezen esetekben meghívódnak.

\subsubsection{Callback}

Ezen függvények elnevezése a \textit{GTK+} terminológiában \textit{callback}. Az egyes konkrét prototípusok a \textit{szignál} fajtájától függenek. A \textit{C} nyelvű változat esetén első paraméterük jellemzően az a \textit{Widget} -- pontosabban szólva \textit{Object}, hiszen a \textit{szignál}kezelés ezen a szinten került implementásra a \textit{Glib}-ben -- melyen az esemény kiváltódott. Ezt a paramétert követik a \textit{szignál}hoz kapcsolódó egyéb jellemzők, az utolsó pedig a szignál bekötésekor megadott, úgynevezett \textit{user data}, amiről a példaprogram kapcsán részletesebben szólunk. Elöljáróban csak annyit, hogy ez egy meglehetősen kényelmetlen és gyakorta nehézkesen használható megoldás, melyre a \textit{C++} nyelvű változat remek alternatívát kínál.

Ennyi bevezető után lássuk az ''ehavi`` példaprogramunkat, majd annak értelmezését.

\section{Szignálkezelés dióhéjban}

Az előző szám módszertanától eltérve az alábbiak szerint elemezzük a kódokat:

\begin{itemize}
 \item külön-külön vesszük számba ez egyes nyelvi változatok sajátosságait, mivel ezen példaprogramok kódjának hasonlósági foka szerény
 \item először a \textit{C} nyelvű verziónak fogunk neki, hogy túl lehessünk a nehezén, azután meglátjuk mennyiben más a helyzet, ha \textit{gtkmm} használatára van módunk \textit{GTK+} helyett
 \item ezt követően egy külön rész foglalkozik a két verzió összehasonlításával
 \item a kódot nem sorfolytonosan, hanem a futás logikája szerint követjük, lévén egy kicsit is bonyolultabb esetben -- mint amilyennek ez is mondható -- már ez a logikusabb
\end{itemize}

Az alábbi példaprogramok \textit{C/C++} nyelvű forrásfájljai, illetve azok eredetijei, a \textit{FLOSSzine}, valamint a \textit{GTK+}/\textit{gtkmm} oldalain az alábbi linkeken érhetőek el:
\ \\\\
\url{http://www.flosszine.org/sources/gtk_signal.c}\\
\url{http://www.flosszine.org/sources/gtkmm_signal.cc}\\
\url{http://library.gnome.org/devel/gtk-tutorial/stable/c39.html#SEC-HELLOWORLD}\\

\subsection{A kód}

\lstinputsources
{sources/gtk_signal.c}
{sources/gtkmm_signal.cc}
{Minimál példa \texttt{GtkWindow}hoz}
{lst:gtksignal}

\subsubsection{\textit{GTK+} részletek sorról sorra}

\begin{enumerate}
 \item[27-28] Annak szükségességéről, hogy a változók egy függvény elején legyenek deklarálva az előző részben esett szó, így erre itt nem térnünk ki. Ellenben fontos, hogy a változók típusa \texttt{GtkWidget}, a specifikusabb \texttt{GtkWindow}, illetve \texttt{GtkButton} helyett. Ennek megyarázata, hogy minden olyan függvény a \textit{GTK+}-ban mely egy új \textit{widget}et hozhatunk létre -- azaz a \texttt{\_new} végű metódusok -- egy \texttt{GtkWidget} típusú objektumra mutató pointerrel tér vissza. Ennek kényelmi okai vannak, jelesül, hogy elkerülhessük a folytonos típuskényszerítéseket, hisz legtöbbször olyan függvyéneket használunk, melyek \texttt{GtkWidget}eket kezelnek, tehát ilyen típusra mutatót vesznek át első paraméterként. Ha egy specifikus -- mondjuk \texttt{GtkButton}-t kezelő -- függvényt akarunk hívni, akkor vagy fordítási, vagy ami javasol, futás idejű típuskényszerítés alkalmazandó.

 \item[32] Főablakunk létrehozása, ahogy eddig is.

 \item[34] Az első szignálbekötés. Viszonylagos egyszerűsége ellenére számos apróságra érdemes figyelmet fordítani. Az első maga a \texttt{g\_signal\_connect}, ami függvénynek tűnhet, pedig ugyanúgy, mint a \texttt{g\_signal\_connect\_swapped}, makró, melyek a \texttt{g\_sinal\_connect\_data} függvényt burkolják. A soron következő érdekesség a \texttt{G\_OBJECT} makró, mely futás idejú típusellenőrzést hajt végre a neki megadott paraméteren, majd egy \texttt{GObject} típusra mutatóval tér vissza. A megrögzött \textit{C++} programozók joggal kérdezhetik, mi szükség erre, hisz egyfelől majd elvégzi a típusellenőrzést a fordító, meg hát a \texttt{GtkWindow} típus úgy is leszármazottja a \textit{GObject} ''osztálynak``. Ez így is lenne, na de ez itt \textit{C}, tehát ős-, illetve származtatott osztályokról csak logikailag lehet szó, a típusellenőrzés tehát nem végezhető, sőt minden esetben a hívott függvénynek megfelelő típuskényszerítő makrót kell alkalmazni. A második paraméter a a szignál neve, jelen esetben, mellyel azt adjuk meg, hogy az előző paraméterként megadott \textit{object} melyik szignáljára is szeretnénk \textit{callback}et kötni. Harmadik paraméter azon függvény címe, amelyik meghívódását ki szeretnénk váltani az esemény bekövetkezésekor. A függvénynevet itt is egy makró segítségével adjuk át, ami az előzőekhez hasonlóan \textit{C} nyelvi hiányosságokra vezethető vissza. Mivel a meghívandó \textit{callback}ek prototípusai igen sokfélék lehetnek (ami magából a példából is látszik valamelyest és ezek mind külön típusnak minősülnek a \textit{C} nyelvben ezért ahány féle \textit{callback} variáció létezik, annyiféle \textit{g\_signal\_connect} függvényre lenne szükség. Könnyen belátható, hogy a jogos lustaság más irányba vitte a \textit{GTK} fejlesztőit. A \texttt{G\_CALLBACK} tulajdonképpen egy fordítási idejű típuskényszerítés egy általános függvénytípusra, amivel ugyan egyre megoldottuk, hogy csak egyetlen \texttt{g\_signal\_connect\_data} függvényre legyen szükség, de elvesztettünk minden nemű típusbiztosságot. Ha például egy az adott szignálnak nem megfelelő típusú függvényt adunk meg paraméterként, amit a példabeli függvénynevek felcserélésével könnyen megtehetünk, csúnya meglepetésekben lesz részünk, de csak futásidőben. Az utolsó paraméter az úgynevezett \textit{user data}, ami arra szolgál, hogy az eseménykezelő függvényünknek, olyan adatokat adjunk át, melyek az esemény megtörténtéből nem következnek. Ilyenek lehetnek például más \textit{widget}ek címe, ahogy azt látni is fogjuk. Ez esetben az átadott paraméter \texttt{NULL}, ami szintén egy makró mely egy jól nevelt \texttt{((void*) 0)} kifejezésre fejtődik ki \textit{C} kód esetén. Zárszóként ehhez a sorhoz csak annyit, hogy a \texttt{delete\_event} eseményt az ablakkezelő váltja ki, akkor, amikor az ablakot valamilyen módon (billentyűzet, menü, egér) bezárjuk.

 \item[9] Ez a \texttt{delete\_event} szignálkezelő függvénye, aminek egy \textit{gboolean} érékkel kell visszatérnie. Ha ez az érték 0, akkor a \textit{GTK} folytatja a szignál kezelését, azaz meghívja a \texttt{gtk\_widget\_destroy} függvényt az ablakunkra. Ha azonban a visszaetérési érék nem 0, ahogy itt sem az, akkor ezzel arra utasítjuk a \textit{GTK}-t, hogy a szignál további feldolgozásától tekintsen el, aminek révén elérhetjük, hogy hiány nyomogatjuk a jobb felső sarokban a \textit{x}-et ennek hatására bizony szinte semmi nem történik. Itt érdemes felhívni a figyelmet arra, hogy mivel a \textit{C} nyelvben, a \textit{C++}-al ellentétben, nincs bool típus annak analógiájára definiálták a \texttt{gboolean} típust (ami tulajdonképpen egy \texttt{int}) és annak két értékét makróként (\texttt{TRUE}, \texttt{FALSE}).

 \item[37] Az előzőhöz teljesen hasonló, annyi eltéréssel, hogy itt a \texttt{destroy} szignálra kötünk be eseménykezelőt, ami akkor hívodik meg, ha a \texttt{window} változóval meghívódik a \texttt{gtk\_widget\_destroy} függvényt. Ez több módon is lehetséges. Egyfelől direkt módon egyszerű függvényhívással, amire ebben a kódban nincs példa, másrészről indirekt módon, amire viszont van, erre lesz jó a \textbf{47} sor. Illetve lenne még egy út, amit ebben a példakódban szintén elkerülünk.

 \item[18] A \textit{destroy} szignál kezelése általánosságban nézve ritka és itt is csak az a szerepe, hogy a program valamilyen módon ki tudjon lépni. Ha ez a függvény meghívódik, akkor az ablakunk épp megszűnő félben van. Ha ez az eset áll is fenn a programunk futása annak ellenére sem érne véget, hogy az ablakunk bezáródik, hiszen a \textit{main loop}ból nem lépnénk ki. Ezen helyzet elkerülésére ebben az eseménykezelő függvényben meghívjuk a \textit{main loop}ból való kilépésre szolgáló \textit{gtk\_main\_quit}-ot.

 \item[42] Egy nyomógom létrehozása.

 \item[44] Eseménykezelő függvény bekötése a gomb \texttt{clicked} szignáljára, ami a gomb lenyomásakor hívódik meg.

 \item[47] A \textbf{44}. sortól csak a meghívandó eseménykezelő függvényben, illetve az annak átadandó paraméterekben sorrendjében tér el. Ahogy az a függvény nevéből (\texttt{g\_signal\_connect\_swapped}) következik, arról van szó, hogy a gomb lenyomásakor meghívandó \textit{callback} -- jelen esetben a \texttt{gtk\_widget\_destroy} -- paramétereiben a \textit{user\_data}, illetve az az object, amin az esemény kivátótik felcserélésre kerül. Kicsit konkrétabban fogalmazva a \textit{user\_data} lesz a \textit{callback} első paramétere és a gomb a második. Mivel itt a \textit{callback} a \texttt{gtk\_widget\_destroy} függvény, ami paraméterként a destruálandó \textit{widget}et várja, a \textit{user\_data} pedig az ablakunk, nem nehéz kitalálni, hogy a gombra való kattintás eredményeként az ablak meg fog szűnni, de csak azután, hogy a \textit{Helló Világ} üzenet megjelent a konzolban.

 \item[3] A fenti állítás csak azért igaz, mert a \textit{hello} függvény, mint eseménykezelő előbb kerül felkötésre, mint a \texttt{gtk\_widget\_destroy}, valamint azért, mert az eseménykezelők alapvetően a felkötés sorrendjében kerülnek meghívásra. Fordított esetben előbb hívódna meg a \textit{destroy} az ablakra, ami --sok egyéb mellett -- leköti az eseménykezelő függvényeket.

 \item[51] A nyomógomb hozzáadása az ablakhoz.

 \item[53,54] A létrehozott \textit{widget}ek megjelenítése.

 \item[56] Belépés az eseményvezérelt szakaszba.

\end{enumerate}

Fentiek ismeretében nagy biztonsággal jósolhatjuk meg példaprogramunk működését. Az elindított alkalmazás egy ablakot jelenít meg, melyben egy \textit{Helló Világ} feliratú gomb lesz. Az ablak bezárásával hiába próbálkozunk egér, vagy billentyűzet segítségével, ezen kísérletek eredmény csupán egy-egy ''\textit{delete event occurred}`` sor a konzolban. Ha azonban le találnák nyomni gombunkat az ablak hirtelen eltűnik a konzolban egy ''\textit{Helló Világ}`` felirat jelenik meg és a program kilép. Lássuk, hogy érhetünk ehhez teljesen hasonló funkcionalitást \textit{C++}-ban.

\subsubsection{\textit{Gtkmm} részletek sorról sorra}

\begin{enumerate}
 \item[43-47] Az korábbi számokban megjelent példaprogramokhoz képest egyetlen eltérést vehet észre a figyelmes olvasó. Mégpedig azt, hogy \texttt{Gtk::Window} helyett \texttt{MyWindow} típust használunk főablakunk létrehozásához. Mivel azonban a \texttt{MyWindow} publikusan származik a \texttt{Gtk::Window} típusból ez a \textit{gtkmm } számára nem jelent különbséget. A \textit{C} változathoz képest a számaztatás itt nem csupán ''logikai'', vagyis minden a \textit{C++}-an megszokott előny könnyedén realizálható. Erre példa, hogy a származtatás miatt nincs szükség semmilyen típuskényszerítésre mikor a \textit{Gtk::Main::run} hívjuk, ami pedig egy \textit{Gtk::Window} referenciát vesz át paraméterként.

 \item[28-37] Saját osztályunk konstruktorában megtehetjük mindazokat a lépéseket, melyeket a \textit{C} nyelvű változat esetén a \texttt{main} függvényben voltunk kénytelenek implementálni, úgy is mint a szignálok felkötése, a gomb hozzáadása az ablakhoz, a \textit{widget}ek megjelenítése. Az egységbezárás ezen előnyén túl a származtatásból fakadó örömöket is élvezhetjük, ugyanakkor persze az ebből fakadó kötelességeknek is eleget kell tenni. Ez esetben ez a konstruktor meghívását jelenti, ami rejtett módon megy végbe. Az ősosztály konstruktorának explicit hívása hiányában a \texttt{Gtk::Window} azon konstruktora fut le, mely paraméterek nélkül is hívható. Másrészről viszont az adattagként tárolt \textit{button}t is inicializálnunk kell. Itt is lehetne közvetve implicit módon hívni a paraméter nélküli konstruktort, azonban kézenfekvőbb azt a változatot használni, amivel egyszerre a gomb feliratát (\textit{label}) is megadhatjuk, így egy hívás a későbbiekben megspórolható. Külön szót érdemelnek a szignálok bekötései. Különösebb programozó géniusz nem kell, hogy felfedezzük a szignálok eléréséhez egy \textit{signal\_}\textit{szignálnév} nevű tagfüggvény meghívását használhatjuk fel. Az ilyen formájú függvények egy, a \texttt{Glib::SignalProxyBase}-ből származó, osztályt adnak vissza, melyek \texttt{connect} nevű metódusai valósítják meg azt, amit a \textit{GTK+} esetén a \texttt{g\_signal\_connect} makró tett meg. Előnye ennek a módszernek, hogy típusbiztos, azaz a \texttt{connect} paraméterként csak olyan függvényt (\textit{slot}) fogad el, aminek a típusa megfelel az adott szignál kezeléséhez. További előny, hogy a \textit{slot}okhoz nem csupán egy \textit{user data} csatolható, hanem tetszés szerinti, s ezek típusa is ellenőrzésre kerül fordításkor. Amennyiben azonban sikerül csupán egy apróságot is félreírnunk a szignál bekötésénél, vagy a \textit{slot} típusának megadásánál, a \textit{C++}-is template-ekkel történt megvalósításnak hála akár több oldalas, nehezen kibogarászható hibaüzenetekkel találhatjuk magunkat szemben.

 \item[31-32] Lássuk akkor miként is érhető el ugyanaz \textit{gtkmm}-ben, mint ami korábban \textit{GTK+}-ban. Első pillantásra is szembeszökő, hogy mindkét sorban találunk olyan hívást, melyek nem a \texttt{Gtk} névtérben definiáltak. Ennek oka, hogy a \textit{gtkmm}-ben a szignálkezelést \textit{libsigc++} nevű függvénykönyvtárral valósították meg, bár az ilyen hívások jellemzően csak a szignálok felkötésekor kerülnek elő. A két tárgyalt sor közötti eltérés könnyen indokolható, ha figyelembe vesszük a megadott eseménykezelő függvények típusát.

 \item[31] Ha a megadni kívánt függvény statikus -- függetlenül attól, hogy osztályhoz tartozik-e vagy azon kívül definiált, esetleg tisztán \textit{C} nyelvű kódból származik -- \textit{slot} létrehozásához a \textit{sigc::ptr\_fun} alkalmazandó. Ebben a konkrét esetben a \textit{slot} létrehozásán túl, paramétereket is hozzákapcsolunk a \textit{clicked} esemény bekövetkeztekor meghívandó függvényhez. Ennek eszköze a \textit{sigc::bind}, melynek első paramétere egy \textit{slot}, a továbbiak pedig a csatolandó paraméterek. Itt csupán egy ilyen van, a gomb lenyomására kiírandó üzenet szövege. Ez persze kissé kényszeredett, hiszen a paraméter értéke soha nem változik, így ennek igazi hasznát ebből a példából nehéz belátni.

 \item[4-7] Statikus függvényük a lehető legegyszerűbb csupán azt szemlélteti miként is kell az átadott paraméterekez használni, ezesetben annak értékét a console-ra kiíratni, működését és funkcióját tekintve a \textit{C}-s változat azonos nevű függvényével analóg.

 \item[32] Ha a megadni kívánt eseménykezelő egy osztály tagfüggvénye, akkor a \textit{sigc::mem\_fun} használható arra, hogy \textit{slot}ot hosszunk létre belőle, azzal a különbséggel, hogy az osztály példányának címe lesz az első paraméter és csak ezt követi a függvény címe. Fontos, hogy a függvényt teljes névtérlistával együtt kell megadni. Természetesen a korábban megismert \texttt{sigc::bind} it is alkalmazható.

 \item[22-25] Ez a függvény sem több csupán, mint egy egyszerű példa, ugyanazt a célt szolgálja, mint a \textit{GTK+}-os verzió azonos nevű függvénye.

 \item[15-20] Ezen függvény megértésének kulcsa a \texttt{virtual} kulcsszóban rejlik. Minden szignálhoz tartozik ugyanis egy -- az adott \textit{widget} által implementált -- kezelő függvény, mely alkalmasint felülbírálható (ovverride). Ha ezt megtesszük, azzal a szignál kezelésének teljes folyamatát mi irányítjuk, ami mellett komoly érvek szólhatnak, de nem árt körültekintőnek lenni. Ebben az esetben a cél pont annak demonstráljuk, szabad kezet ad a \textit{gtkmm} abban, hogy egy származtatott \textit{widget} miként kívánja kezelni az ősosztály eseményeit. A visszatérési érték szerepe ugyan az, mint az előző példában.
\end{enumerate}

A szignálkezelésről összegzésképpen annyit, hogy alapvetően két lehetőség kínálkozik arra,. hogy az egyes \textit{widget}ek eseményei kezeljük:

\begin{itemize}
 \item \textit{Callback}eket kapcsolni azon \textit{widget}ek azon eseményihez, melyek számunkra érdekesek és ezekben megtenni a megfelelő lépéseket

 \item Felülbírálni a \textit{widget} saját eseménykezelőjét az öröklődés mechanizmusai útján. Erre mindkét változat esetén van lehetőség, ám a \textit{GTK+} megoldása kissé körülményes és nehezebben megérthető, így annak ismertetése valmely későbbi részre marad. A \textit{C++} nyelvi eszközeit kihasználva a \textit{gtkmm} viszont ezt oly könnyedén oldja meg, hogy kár lett volna kihagyni a bemutatást annak ellenére is, hogy a módszerre ritkán van szükség, hiszen többnyire arról van szó, hogy a különböző \textit{widget}példányok azonos szignáljainak kiváltódásakor más-más irányba szeretnénk terelni a program futását. A felülbírálás révén viszont arra nyílik lehetőség, hogy a szignál kezelésének módját változtassuk meg. Ha nem kívánunk egyebet tenni, mint ami amúgy is történne, hívjuk meg a felülbírált függvény szülőosztálybeli változatát. Ha azonban  ez előtt, vagy után még valami mást is tenni szeretnénk, megtehetjük, hogy csak a függvény közepéről hívjuk a szülő metódusát, vagy akár el is hagyhatjuk az ha tudjuk mit és főként hogyan szeretnénk kezelni.
\end{itemize}

\subsection{Egy "nyomkodható" ablak}

\subsubsection{Fordítás és linkelés}

A korábbiakhoz hasonlóan az alábbi parancssorok segítségével fordíthatóak e\-lem\-zett programjaink:

\fontsize{8pt}{8pt}
\ \\
\texttt{gcc gtk\_signal.c -o gtk\_signal \`{}pkg-config {-}-cflags {-}-libs gtk+-2.0}
\ \\
\texttt{g++ gtkmm\_signal.cc -o gtkmm\_signal \`{}pkg-config gtkmm-2.4 {-}-cflags {-}-libs\`{}}
\ \\

\subsubsection{Futtatás}

Próbálkozzunk ezúttal is a \texttt{./gtk\_window}, illetve a \texttt{./gtkmm\_window} paranccsokkal abban a könyvtárban, ahol a fordítást elkövettük.

\subsubsection{Eredmény}

Bármily hihetetlen ezúttal sem történik semmi egyéb, mint az előző alkalommal. Remélhetőleg azonban a különbség mégis érzékelhető annyiban, hogy legutóbb a meglepetéssel teli borzongást ablakunk váratlan felbukkanása, míg most a bennünk szikraként felvillanó megértés okozza.
