\documentclass[a4paper, titlepage]{report}

\usepackage[a4paper]{geometry}
\usepackage{makeidx}
\geometry{verbose, tmargin=2.0cm, bmargin=2.0cm, lmargin=2.0cm, rmargin=2.0cm}

\input{common_preamble}

%hyper references pdf attributes
\usepackage{hyperref}
\hypersetup{
pdftitle={Helló Window!},
pdfauthor={Pfeiffer Szilárd},
pdfsubject={GTK alapú felhasználó felületfejlesztés és tesztelés C, C++ és Python nyelven}
pdfkeywords={pfeiffer szilárd, gtk, gtk+, gtkmm, gnome, hig, dogtail, gobject introspection, python, linux},
bookmarks={true},
unicode={true},
colorlinks={false},
hyperfigures={true},
a4paper={true}
}

\author{
\href{http://pfeifferszilard.hu}{Pfeiffer Szilárd}
\thanks{A könyv létrejöttét a FSF.hu Alapítvány a Szabad Szoftver Pályázat\cite{fsftender2011} keretében támogatta}
}
\title{
Helló Window!\\\medskip
\large{\textit{GTK} alapú felhasználó felületfejlesztés és tesztelés C, C++ és Python nyelven}
}

\makeindex

\begin{document}

\maketitle

\pagenumbering{roman}
\tableofcontents
\newpage
\pagenumbering{arabic}

\part{Bevezetés}

\chapter{\textit{Gimp Tool Kit}}
\section{Általánosságok}

\subsection{Története}
\index{GIMP@\textit{GIMP}}
\index{Motif@\textit{Motif}}
\index{GUI eszközkészlet@\textit{GUI} eszközkészlet}

Mint megannyi szoftver a nyílt forrás több, mint negyedszázados történetében a \textit{GTK+} is felhasználói elégedetlenség eredményeként született. Peter Mattis és Spencer Kimball, a Kaliforniai Egyetem (Berkeley) hallgatói -- az azóta is a nyílt forrású szoftverek egyik zászlóshajójának számító képszerkesztő program, a \textit{GIMP} fejlesztése közben -- szembesültek az akkori időkben amúgy is csak kis számban rendelkezésre álló grafikus felhasználói felület készítő eszközök hiányosságival. Ezek leküzdésére döntöttek úgy -- egyébiránt a \textit{Motif} használata helyett --, hogy belekezdenek egy saját függvénykönyvtár fejlesztésébe.

Az azóta nagykorúvá lett \textit{GTK+} több ízben is komoly átalakuláson ment keresztül. Ezek közül talán a legmeghatározóbb, hogy az eredetileg \textit{GUI} eszközkészletként indult projekt jócskán túllépett eredeti keretein. Ennek lehetőségét a moduláris felépítés teremtette meg, mely később még több ízben hasznára vált a projektnek. Azon részek, melyek nem közvetlenül függenek össze a grafikus felhasználói felületek fejlesztésével, vagy másutt is hasznosak lehetnek, külön modulokban kaptak helyet, egy flexibilis eszközkészletet hozva így létre, melyből mindenki pontosan annyit és csak annyit használ fel, amennyire feltétlenül szüksége van.

A legutóbbi főverzió -- vagyis a \textit{GTK+ 3} --, melyről a továbbiak szó lesz majd, nem csupán folytatja a hagyományokat, de igyekszik mindinkább kiszélesíteni a modularitás adta előnyok alkalmazási területeit.

\subsection{Elérhetősége}
\index{licenc!LGPL@\textit{LGPL}}
\index{Git@\textit{Git}}

A \textit{GTK+} többféle formában, többféle operációs rendszerre és grafikus szerverre is elérhető. A forma alatt ez esetben az értendő, hogy a függvénykönyvtár nem csupán forráskódként, hanem bináris  változatban is letölthető. Ez nyilván nem különösebb meglepetés, már csak azért sem, mert nyílt forrású szoftverről van szó. Itt érdemes megjegyezni, hogy a licenc a \textit{GTK+} -- annak függőségei, illetve számos kapcsolódó függvénykönyvtár esetén -- a \href{http://www.gnu.org/licenses/lgpl-2.1.html}{GNU Lesser General Public License}, rövidítve az \textit{LGPL}, ami összefoglalva annyit tesz, hogy nyílt, illetve zárt forrású szoftverek egyaránt fejleszthetőek a \textit{GUI} eszközkészlet használatával, azzal a kitétellel, hogy a függvénykönyvtár általunk használt -- esetleges módosításokat is tartalmazó -- változatának forrását ügyfeleinknek kérésre át kell adnunk. A szokásjog, illetve a saját érdekünk azt diktálja ugyanakkor, hogy a javítások mielőbb bekerüljenek a fejlesztők által karbantartott változatba. Visszatérve az elérhetőséghez a forráskód mind \href{http://git.gnome.org/browse/gtk+/}{verziókezelő rendszer}en (\textit{Git}) keresztül, mind \href{ftp://ftp.gtk.org/pub/gtk/}{archív állományok} formájában letölthető.

\index{Linux@\textit{Linux}}
\index{Windows@\textit{Windows}}
\index{Mac OS X@\textit{Mac OS X}}
\index{operációs rendszer!Linux@\textit{Linux}}
\index{operációs rendszer!Mac OS X@\textit{Mac OS X}}
\index{operációs rendszer!Windows@\textit{Windows}}
A bináris változatok tekintetében elmondható, hogy mindaddig, amíg \textit{GNU}/\textit{Linux} alapú rendszereken dolgozunk, különösebb nehézségekbe nem fogunk ütközni, hiszen nagy valószínűséggel az általunk használt disztribúció mind a fejlesztéshez (\textit{devel}), mind a hibajavításhoz (\textit{debug}) szükséges csomagokat tartalmazza. Ugyanakkor a \textit{GTK+} egy multiplatform eszköz, vagyis joggal várható el, hogy több operációs rendszeren is működjenek az általunk megírt kódok. A már említett \textit{GNU}/\textit{Linux} disztribúciókon túl -- melyet a továbbiakban elsődlegesen, de közel sem kizárólagosan használunk majd -- a \textit{GTK+} mind a Microsoft \textit{Windows}, mind pedig az Apple \textit{Mac OS X} rendszerein elérhető.

A használható grafikus szerverekről elöljáróban annyit érdemes megemlíteni, hogy a \textit{GTK+} portabilitásának két alappillére közül az első az a megoldás, mely a felhasználó felület építőköveinek -- a \textit{GTK+} által használt terminológiával élve \textit{widget} -- implementációját elválasztja az azok megjelenítésére szolgáló rajzoló primitívek megvalósításától. Ennek révén biztosítható, hogy a \textit{GTK+} forráskód túlnyomó részének változatlansága mellett -- csupán egy újabb, úgynevezett \textit{backend} hozzáadásával -- alkalmassá tehető egy újabb grafikus szerver, vagy alrendszer (pl: X11, frame buffer, HTML5, \dots) alá.

\section{Részegységek}
\index{GTK@\textit{GTK}}

A \textit{GTK}\footnote{a \textit{GTK} rövidítés alatt a továbbiakban az általánosságban vett grafikus felhasználói felület fejlesztői eszközt, míg \textit{GTK+} alatt ennek \textit{C} nyelvű változatát értjük} szervezéséről fontos elöljáróban megemlíteni, hogy példaértékűen választja szét a funkcionalitás egyes elemeit, jól elhatárolt implementációs részegységre, melyek fejlesztése egymástól függetlenül, a \textit{GNOME} projekttel együttműködésben folyik.

Ez a megoldás -- számos előnye mellett -- jár természetesen néhány nehézséggel is. A modulok csak publikus interfészeiken keresztül tudnak egymással kommunikálni. Ez egyrészről függetlenséget jelent a modulok belügynek tekinthető implementációs részleteinek terén, viszont komoly kötöttség a publikus felület oldalán, lévén annak változatlanságától nem csak a külső projektek, de az egyes \textit{GTK} modulok is függenek.

A hosszan fenntartandó állandóság hatásait jól példázza, hogy a \textit{GTK+} csak hat év után indított új főverziót (\texttt{3.x}), mely felszámolta a korábbi változattal való -- helyenként csaknem teljesen felesleges -- kompatibilitást. Az azóta eltelt időben\footnote{a \texttt{3.0.0} verzió megjelenése időpontja 2011. február} viszont vajmi kevés projekt döntött úgy, hogy átáll az új verzióra, még azzal együtt sem, hogy az egyszerűbb alkalmazások tekintetében ez különösebben komoly erőforrást nem igényel. Célszerűen tehát ezek az inkompatibilis váltások nem lehetnek túl gyakoriak.

A következőkben sorra vesszük, mik azok a részegységek, melyek együtt a \textit{GTK} \textit{C} nyelvű változatát alkotják, s melyek természetesen a további nyelvi változatok alapjául szolgálnak.

\subsection{GTK}
\index{GTK@\textit{GTK}}
\index{GUI}

A \textit{GTK+} a grafikus felhasználó felületek (\textit{GUI}) fejlesztéséhez szükséges felületi elemek (beviteli mezők, rádiógombok, listák, dialógus ablakok, \dots), azaz \textit{widget}ek tárháza, vagyis a \textit{GTK} keretrendszer lelke. A függvénykönyvtár a korábban leírtaknak megfelelően csak a közvetlenül szükséges implementációt, vagyis a \textit{widget}ek kirajzolásához, interakcióinak, adattárolásának megvalósításához szükséges kódok összességét jelenti. Ettől persze némiképp többet, de erről később (\ref{sec:gail}) esik szó.

\subsection{GDK}
\index{GTK@\textit{GTK}!GDK@\textit{GDK}}
\label{sec:gdk}

A \textit{GIMP Drawing Kit} (\textit{GDK}) -- a \textit{GTK+} részeként terjesztett -- alacsony szintű rajzolási és ablakkezelési feladatok megvalósítására, egyszersmind a magasabb szintű rutinok elöl történő elfedésére szolgáló függvénykönyvtár. A \textit{GDK }fontos szerepet tölt be a \textit{GTK} különböző platformok közötti hordozhatóságának megteremtésében, lévén az általa nyújtott -- viszonylag szűk körű -- funkcionalitást újraimplementálva a \textit{GTK+} alkalmas lehet egy újabb grafikus környezetben való futásra. Ezen funkciók alatt a már említett rajzolási primitíveken túl többek között rasztergrafikai feladatok, kurzor megjelenítése, illetve alacsony szintű ablakesemények implementálása értendő.

\index{Linux@\textit{Linux}}
\index{Mac OS X@\textit{Mac OS X}}
\index{Windows@\textit{Windows}}
\index{GDK backend@\textit{GDK} backend!Broadway@\textit{Broadway}}
\index{GDK backend@\textit{GDK} backend!X11@\textit{X11}}
\index{GDK backend@\textit{GDK} backend!Wayland@\textit{Wayland}}
\index{grafikus szerver!X11@\textit{X11}}
\index{grafikus szerver!Wayland@\textit{Wayland}}
\index{operációs rendszer!Linux@\textit{Linux}}
\index{operációs rendszer!Mac OS X@\textit{Mac OS X}}
\index{operációs rendszer!Windows@\textit{Windows}}
A fentieknek köszönhető, hogy az eredetileg csak az \textit{X Window System}\footnote{a \textit{Linux} alapú rendszerek, a \textit{GDK} születésének idejében, gyakorlatilag kizárólagos grafikus szervere} (\textit{X11}) felett működni képes \textit{GDK}, mára nemcsak egyéb \textit{Linux} alapú szerveren (\textit{Wayland}), de más operációs rendszereken (\textit{Windows}, \textit{Mac OS X}), sőt akár webböngészőben is működni képes\footnote{ezen szolgáltatáshoz eléréséhez \textit{websocket} támogatásra van szükség a böngészőben, illetve a \textit{Broadway} elnevezésű \textit{backend} bekapcsolására a \textit{GTK+} fordításakor}. A grafikus alrendszer portolása mellett felmerülő problémák jelentékeny részét a \textit{GLib} függvénykönyvtár oldja meg.

\subsection{GLib}
\index{GTK@\textit{GTK}!GLib@\textit{GLib}}

A \textit{GLib} a \textit{GNOME} projekt egyik fundamentális eleme, mely történetileg ugyan kötődik a \textit{GTK+}-hoz, mára azonban teljesen önállóvá vált. Eredendően a \textit{GTK+} által használt, de attól függetlenül is létjogosultsággal bíró, platformfüggetlen kódok kiemelése egy külön függvénykönyvtárba, melyet számos -- a \textit{GNOME}, vagy a \textit{GTK} projektekhez akár nem is kapcsolódó -- szoftver\footnote{példának okáért a \textit{Compiz}, a \textit{Midnight Commander}, vagy a \textit{syslog-ng}} alkalmaz. A meglehetősen szerteágazó funkcionalitáscsomag melyet a \textit{Glib} megtestesít, röviden a következőben foglalható össze; \textit{C} programozási nyelven írt standard függvénykönyvtárak nem, vagy csak részben elérhető, esetleg nehézkesen használható eszközök összessége.

A teljesség igénye nélkül megemlítendőek a \textit{GLib} adattárolói (láncolt listák, fák, dinamikus tömbök, hash táblák, \dots), hordozható adattípusai (\texttt{guint32}, \texttt{gboolean}, \dots), memória allokációs függvényei (\texttt{g\_new}, \texttt{g\_allocator\_new}, \texttt{g\_slice\_alloc}, \dots), gyakran használt formátumok (dátum, idő, URI, fájl- és könyvtárnév, \dots) kezelésére szolgáló eszközei, konverziós algoritmusai (base64, endianness, karakterkódolás, \dots) széles körben alkalmazott módszerek implementációi (XML, Glob, Regex, \dots). Mindezeken túl a \textit{GLib} tartalmaz néhány külön is említésre méltó alrendszert.

\subsubsection{GObject}
\index{GTK@\textit{GTK}!GLib@\textit{GLib}!GObject@\textit{GObject}}

A \textit{GLib Object System} egy \textit{C} nyelven írt objektumorientált keretrendszer, mely megvalósít számos olyan funkciót (származtatás, virtuális függvények, típus, objektumok memória menedzsmentje, \dots) melyek például a \textit{C++}, vagy a \textit{Java} esetén nyelvi szinten adottak. A \texttt{GObject}, mint osztály szolgál például alapjául a \textit{GTK+} által implementált minden egyes \textit{widget}nek. A kép ezzel azonban még közel sem teljes.

A \textit{GObject} implementál számos alapvető fontosságú egyszerű (\texttt{gdouble}, \texttt{gint}, \dots) és összetett típust, illetve támogatja ezek felhasználását saját típusok létrehozásakor, illetve a \textit{GObject}-ből származó osztályokban adattagként való elhelyezését. Emellett megvalósít egy -- az objektumok állapotváltozásainak követésére szolgáló -- kommunikációs alrendszert (\textit{signal}).  Fentiek létrehozásakor kifejezett cél volt a rugalmas bővíthetőség és a könnyű adaptálhatóság más nyelvekre. Ez utóbbi a később (\ref{sec:otherlanguages}) említésre kerülő nyelvi változatok egyszerű megvalósíthatóságának előfeltétele.

\subsubsection{GModule}
\index{GTK@\textit{GTK}!GLib@\textit{GLib}!GModule@\textit{GModule}}
\index{dinamikus modulbetöltés}

A \textit{GModule} egy dinamikus modulok betöltésére szolgáló függvénykönyvtár, mely rendelkezésre áll mindazon rendszereken, ahol a \textit{GLib} is, elfedve ez egyes operációs rendszer különbözőségeit ezen a területen.

\subsubsection{GThread}
\index{GTK@\textit{GTK}!GLib@\textit{GLib}!GThread@\textit{GThread}}
\index{szálkezelés}

A \textit{GThread} célja erősen hasonlatos az imént említett \textit{GModule}-éhoz, vagyis biztosítani egy, a platformok sajátosságaitól független megoldást ezúttal nem a dinamikus modulbetöltés, hanem a szálkezelés tekintetében.

\subsubsection{GIO}
\index{GTK@\textit{GTK}!GLib@\textit{GLib}!GIO@\textit{GIO}}
\index{GObject@\textit{GObject}}
\index{POSIX}

Az eddigieket folytatva a \textit{GIO} is egy, a multiplatformos programozás során gyakran felmerülő probléma -- jelesül a fájlok, fájlrendszerek, meghajtók kezelése -- megoldására született. A megfelelő POSIX hívásokkal eddig is megvalósítható volt egy kvázi platformfüggetlen fájlkezelés, így ennek önmagában nem lenne számottevő haszna. A \textit{GIO} ugyanakkor több, mint egy egyszerű POSIX hívásokat burkoló függvényhalmaz. A \textit{GObject}-re támaszkodva egy magasabb szintű, dokumentumközpontú interfészt valósít meg.

\subsection{Cairo}
\index{Cairo@\textit{Cairo}}
\index{GTK@\textit{GTK}!GDK@\textit{GDK}}

A \textit{Cairo} egy eszközfüggetlen\footnote{értsd hardvereszközöktől független} kétdimenziós vektorgrafikai függvénykönyvtár, melyet kifejezetten a hardveres gyorsítókkal való együttműködésre terveztek, s mellyel a \textit{GDK} a kétdimenziós rajzolási feladatait végzi. Érdemes megemlíteni, hogy a \textit{Cairo} nem a \textit{GNOME}, hanem a \href{http://freedesktop.org}{\textit{freedesktop.org}} projekt része.

\subsection{Pango}
\index{Pango@\textit{Pango}}

A \textit{Pango} szövegek képi formában történő előállításáért (\textit{rendering}) és megjelenítésért (\textit{lay out}) felelős a \textit{GTK}-n belül, de természetesen a \textit{GTK}-tól függetlenül is használható, lévén a függvénykönyvtár az előbbiekhez hasonlóan számos platformot támogat.

\section{Nyelvi változatok}
\label{sec:otherlanguages}

\subsection{GTK minus minus}
\index{gtkmm@\textit{gtkmm}}
\index{wrapper}
\index{C++}

A \textit{gtkmm}, illetve annak függőségei adják a \textit{GTK} projekt \textit{C++} nyelvű változatát. Ezek a függvénykönyvtárak wrapperek az eredeti \textit{C} változat fölött, az ebből fakadó előnyökkel és korlátokkal együtt. Ezen kódok jelentékeny része wrapper mivoltukból következően generált, ugyanakkor számos helyen -- ahol ez funkcionalitáshoz a programozási nyelvhez leginkább illeszkedő megvalósításához szükséges -- eredeti kódot is tartalmaz.

A \textit{C}, illetve \textit{C++} nyelvű változatok a lehető legkisebb mértékben térnek el egymástól. Ez egyben azt is jelenti, hogy az egyes nyelvi változatok nem tartalmaznak a többihez képest többlet funkcionalitást. Nem lehet azonban eltekinteni az egyes programozási nyelvek adta lehetőségek előnyeitől, hátrányaitól, melyek könnyebbé vagy nehezebbé teszik a \textit{GTK} adott nyelven való használatát.

\subsubsection{Libsigc++}
\index{libsigc++@\textit{libsigc++}}

A \textit{gtkmm} implementációjánál használt függvénykönyvtár, ami lehetővé teszi a szignálkezelés típusbiztos megvalósítását, mely a \textit{C} változatnál -- a nyelvi sajátosságok okán -- nem adott.

\subsection{PyGobject}
\index{PyGObject@\textit{PyGObject}}
\index{Python}
\index{C}

A \textit{Python} változat -- ahogy számos más egyéb nyelvi variáció is -- alapjai gyökeresen megváltoztak a \textit{GTK+} új főverziójának megjelenésével. A korábbi -- a \textit{gtkmm} által is alkalmazott -- módszer, az eredeti változatot rejti el, burkolja be (\textit{wrap}), vagy egy köztes réteget képez a \textit{C}, illetve a cél nyelv -- esetünkben a \textit{Python} -- között. Ezen réteg többé-kevésbé természetesen automaták (pl: generátor szkriptek) révén jön létre, ugyanakkor igaz az, hogy nem közvetlenül az eredeti kódbázist használja a burkoló réteg létrehozására. Következésképpen a \textit{GTK+} publikus felületében bekövetkező változásokat az egyes \textit{wrapper}eknek rendre követniük kell, holott a \textit{GTK+} kódjának írásakor is adottak azok a metaadatok, melyek mondjuk egy \textit{Perl}, vagy \textit{Python} elkészítéséhez szükségesek.

\subsubsection{GObject Introspection}
\index{GObject Introspection@\textit{GObject Introspection}}

Az előbbi gondolatot tovább fűzve juthatunk el ahhoz a kézenfekvő kérdéshez, hogy miért nincsenek az említett metaadatok rögtön a \textit{GTK+} -- illetve a \textit{GLib}, \textit{Pango} és a többi függőség -- kódja mellett, ahonnan kinyerve azokat az egyes nyelvi változatokat egyszerűen generálni lehetne. A \textit{GObject Introspection} pontosan ezt célozza. Egy \textit{binding} elkészítéséhez szükséges adatok a \textit{C} nyelvű változatok -- kódjában egy erre a célra meghatározott formátumban -- megjegyzésként szerepelnek, a különböző nyelvű változatok pedig ezt felhasználva jönnek létre.

\subsection{Összehasonlítás}

Az egyes változatoknak megvannak a maguk -- jellemzően a programozási nyelv sajátosságaiból következő -- előnyei. Ilyenek lehetnek például a \textit{C} nyelv, illetve a fordítók széles körű elterjedtsége, a \textit{C++} azon sajátossága, hogy a nyelv nyújtotta módszereket, mint például az örököltetés, itt közvetlenül használhatjuk ki, vagy a \textit{Python} nyelvű fejlesztés sebessége. Míg mondjuk a \textit{C} nyelv esetében bizonyos funkciók kissé nehézkesen használhatóak, addig a \textit{C++} objektumorientált megközelítése mellett ugyanez a funkció játszi könnyedséggel elérhető, vagy éppen a \textit{Python} nyújtotta szkript környezet ad könnyebb, rugalmasabb kezelhetőséget. Az említett három változat tekintetében a főbb ismérveket a \ref{tab:comparselanguages} táblázat tartalmazza.

\begin{table}
\begin{center}
\begin{tabular}[t]{l c c c}
                                & \textbf{\textit{GTK+}}          & \textbf{\textit{gtkmm}}  & \textbf{\textit{PyGObject}} \\\\
\textbf{Nyelv:}                 & C                               & C++                      & Python                      \\\\
\textbf{Implementáció módja:}   & natív                           & wrapper                  & binding                     \\\\
\textbf{Objektumorinentált}     &                                 &                          &                             \\
\textbf{technikák használata:}  & közvetett                       & natív                    & natív                       \\\\
\textbf{Licenc:}                & LGPL                            & LGPL                     & LGPL                        \\\\
\textbf{Ismertebb projektek:}   & Evolution, Firefox, Gimp, \dots & GParted, Inkscape, \dots & gedit                       \\

\end{tabular}
\caption{A \textit{GTK+}, \textit{gtkmm}, \textit{PyGObject} összehasonlítása}
\label{tab:comparselanguages}
\end{center}
\end{table}

\section{Kapcsolódó projektek}

\subsection{Automata tesztelés}
\index{automata tesztelés}

A grafikus felületek kapcsán sajnálatosan elhanyagolt terület az automata tesztelés, azon nyilvánvaló tény ellenére is, hogy a felhasználó épp ezeken a felületeken keresztül éri el az érdemi funkcionalitást és nyeri első benyomásait a szoftverrel kapcsolatban, így ennek megjelenése, valamint helyes működése döntő az alkalmazás későbbi sikerességének tekintetében. Ezzel együtt igaz továbbá, hogy teljes körű (\textit{end-to-end}) megvalósított tesztelés mindenképpen a felhasználó felületről indított akcióval kell induljon és az ugyanott tapasztalt reakció ellenőrzésével kell végződjön.

A \textit{GTK} felhasználásával fejlesztett felületek tesztelésénél rendelkezésünkre áll a megfelelő keretrendszer, mely lehetőséget teremt, hogy az elkészült felületi elemek működését automaták segítségével teszteljük. A későbbiek során bemutatásra kerülő mintapéldák esetén mindenütt kitérünk majd az azok kapcsán felmerülő tesztelési feladatok megoldásának mikéntjére. Most azonban lássuk nagy vonalakban hogyan is működik ez a tesztelési keretrendszer.

\subsubsection{Accessibility Tool Kit}
\index{ATK@\textit{ATK}}
\label{sec:atk}

Elsőre talán egymástól távoli területnek tűnik a szoftverek akadálymentesítése (\textit{accessibility}), valamint a felhasználó felületek automata tesztelése, egy valami mégis összeköti őket. Ez pedig az a követelmény, aminek a szoftver mindkét cél elérése érdekében eleget kell tegyen, ami nem más, mint az alkalmazás vezérelhetősége bizonyos felhasználói interakciók kizárása mellett is. Az automata tesztelés esetén ez gyakorlatilag az összes eszköz (billentyűzet, egér, \dots) kizárását jelenti, hiszen a felhasználót ez esetben teljes egészében a tesztelést végző szoftver helyettesíti.

A szoftverek akadálymentesítésének biztosítása egy speciális megközelítést igényel, mely a fogyatékossággal élő emberek szoftverekkel végzett munkájának megkönnyítését helyezi előtérbe. Ehhez az \textit{ATK} csupán annyit tesz, hogy definiál egy interfészt, melyen keresztül az adott szoftvert el lehet érni. Indulva onnan, hogy egy adott alkalmazást ki lehet választani az összes aktuálisan futó alkalmazás közül, folytatva azzal, hogy le lehet kérdezni az általa megnyitott ablakokat, az abban lévő felületi elemeket (\textit{widget}), egészen odáig, hogy az általuk tárolt értékeket (egy beviteli mező szövege, folyamatindikátor értéke, \dots), illetve állapotokat (rádiógomb kiválasztott állapota, beviteli mező szerkeszthetősége, \dots) írni olvasni lehet, rajtuk akciókat (gomb lenyomása, menüelem kiválasztása) végezhetünk.

\subsubsection{Gail}
\index{ATK@\textit{ATK}}
\index{Gail@\textit{Gail}}
\label{sec:gail}

Lévén az \textit{ATK} lényegében csak egy interfész definíció, ahhoz minden esetben\footnote{már amennyiben az adott grafikus felületfejlesztői rendszer -- mint amilyen a \textit{GTK+}, vagy mondjuk a \textit{Qt} -- biztosítani kívánja az \textit{ATK}-n keresztüli elérést} tartozik egy implementáció, mely az adott felületfejlesztői rendszer működését megfelelteti az \textit{ATK} által definiáltaknak. A \textit{GTK}  esetén ez az implementáció a \textit{Gail}.

\subsubsection{Dogtail}
\index{Dogtail@\textit{Dogtail}}

A \textit{Dogtail} egy \textit{Python} nyelven írt és \textit{Python} nyelven használható tesztautomatizációs eszköz, illetve keretrendszer. Segítségével létrehozhatók felhasználói felületek -- a már említett \textit{ATK} interfészen keresztül -- tesztelő szkriptek, többféle formában és módon is. Ami a módot illeti, lehetőségünk van egyrészről effektíve forráskód -- azaz egy \textit{Python} szkript -- formájában létrehozni a tesztjeinket, vagy úgymond ''felvételt'' készíteni magáról a tesztelésről, majd az így rögzített eseményeket mint tesztet visszajátszani. Ha az előbbi módszernél maradunk -- amiről a további részekben is szó esik --, akkor is két lehetőség adódik, hiszen a \textit{Dogtail} rendelkezik egy procedurális, illetve egy objektumorientált megközelítésű API-val, melyek tetszés szerint használhatóak a tesztek elkészítésekor.

\subsubsection{Accerciser}
\index{Accerciser@\textit{Accerciser}}

Mind az automata tesztelő szkriptek megírásakor, mind egy konkrét alkalmazás felületének feltérképezésére, mind pedig az \textit{ATK} interfésszel történő ismerkedésre alkalmas eszköz az \textit{Accerciser}, mely az akadálymentesített szoftverek feltérképezésére szolgáló eszköz és mint ilyen pontosan azon adatok megjelenítésére és módosítására, valamint azon akciók végrehajtására alkalmas, amire a \textit{Dogtail} szkriptek révén képesek vagyunk.


\chapter{Alapvető ismeretek}
\section{A fejlesztés fogalmai}

\subsection{A \textit{GTK+} objektum-orientáltsága}

Ebben a fejezetben arra próbálunk mélyebben is rávilágítani, hogy a \textit{GTK+} ugyan \textit{C} nyelven íródott, de számos az objektum-orientált nyelvek esetén megszokott terminológiát használ, sőt ezeket a nyelvi eszközök adta mértékben meg is valósítja. Ezért az objektum-orientált fejlesztés fogalmait, kifejezéseit joggal használjuk még akkor is, ha \textit{GTK+} nyelvű fejlesztésről esik szó.

Azzal együtt, hogy az objektum-orientált mechanizmusokat nyelvi szinten a \textit{C} nem, csak a \textit{C++}, illetve a \textit{Python} támogatja lehetséges ezekkel élni ezen változat esetén is. Lássuk mik lennének ezek és hogyan válik lehetővé alkalmazásuk a \textit{GTK+} esetén.

\subsubsection{Egységbezárás}
\index{egységbezárás}
\index{GObject@\texttt{GObject}}

Az objektum-orientált alapelvek közül \textit{C} nyelven nem könnyen biztosíthatóról beszélünk, ugyanakkor a \textit{GTK+} megteszi, amit ebben a tekintetben megtehető. Az adatstruktúrákat -- jelen esetben \textit{widget}eket -- és az azokon műveleteket végző függvényeket a lehetőségekhez mérten egységkén t kezeli, valamint elrejti őket a külvilág elől.

A \textit{GTK+} minden saját makrót/függvényt \texttt{GTK}/\texttt{gtk} prefixszel lát el (a \textit{GLib} esetén ez pusztán csak egy kis, illetve nagy \texttt{g} betű). Egy adott részterület -- például egy \textit{widget} -- saját ``névtérrel'' is rendelkezhet, azaz újabb prefixet vezethet be\footnote{a \textit{GtkWindow} típushoz tartozó függvények prefixe \texttt{gtk} helyett \texttt{gtk\_window}}. Ezeket egymástól, illetve a ``valódi'' funkciót jelölő nevektől \texttt{\_} (aláhúzás) jellel választják el. A \textit{C++} wrapper esetén -- kihasználva a kézenfekvő nyelvi lehetőséget -- a prefixek szerepét természetesen a névterek, illetve az osztályok veszik át\footnote{a \textit{Window} típus a \textit{gtkmm} esetén \texttt{Gtk} névtéren belül szereplő \texttt{Window} nevű osztály}.

\index{átlátszatlan mutató}
Ezen prefixelt makrók/függvények első paramétere minden esetben a prefix által meghatározott típusú objektum, elősegítve ezzel is ezen függvények, illetve az általuk kezelt objektumok egy egységként való kezelését. A privát adatok külvilág elöl való elrejtésére a \textit{C} nyelven erre a célra széles körben alkalmazott átlátszatlan mutatókat\footnote{a módszer egyebek mellett \textit{pimpl} (\textit{pointer to implementation idiom}) néven is ismert} (\textit{opaque pointer}) használja a \textit{GTK+}, ami lehetővé teszi az objektum által használt adatszerkezetek, implementációs módszerek elfedését a publikus interfész használó kódok elöl.

\subsubsection{Öröklés}
\index{öröklés}
\index{GObject@\texttt{GObject}}
\index{gtkmm@\textit{gtkmm}}
\index{PyGObject@\textit{PyGObject}}

Megoldott a \textit{widget}ek egymásból történő származtatása, sőt felhasználói widgetek is definiálhatóak a már meglévőekre támaszkodva. Az kód újrahasznosítását jól mutatja az a tény, hogy a \texttt{GObject} osztály -- mely önmagában is számos hasznos funkcióval rendelkezik -- minden \textit{widget}, illetve számos  más a \textit{GTK}-ban használt nem vizuális elemnek őse. Meg kell jegyezni, hogy a \textit{gtkmm}, illetve a \textit{PyGObject} esetén -- lévén ezen esetekben az objektum-orientáltságot támogatja mag a nyelv -- természetesen a származtatás egy nagyságrenddel egyszerűbb, de mintapéldákat felhasználva némi rutinnal a \textit{GTK+} esetén sem igényel különösebb erőfeszítést.

A \textit{widget}ek öröklési fájáról már itt érdemes megjegyezni, nem csupán két szintű, vagyis azzal együtt, hogy a \textit{GObject} típus a fa csúcspontja, de nem az egyetlen csomópontja. A hasonló funkcionalitású, következésképp rendszerint hasonló megjelenésű \textit{widget}ek (\ref{fig:widgetinheritance}) értelemszerűen egymásból származnak. Ez még a leghétköznapibb esetekben is igaz. A számok kezelésére is alkalmas \textit{widget} (\textit{spin button}) őse az egyszerű, egy sornyi szöveg befogadására alkalmas beviteli mezőt megvalósító \textit{widget} (\textit{entry}).

\includetwingraphics
{Egysoros beviteli mező}
{entry}
{entry.png}
{Számbeviteli mező}
{spinbutton}
{spinbutton.png}
{Öröklés hasonló funkciójú \textit{widget}ek között}
{widgetinheritance}

A mechanizmus további előnye az interfészek\footnote{a kifejezés alatt a \textit{Java} nyelv interfész, illetve a \textit{C++} absztrakt osztálya értendő} kialakításának lehetősége. A \textit{GTK+} a 3-as főverziót megelőzően törekedett arra, hogy a különböző \textit{widget}ek azonos funkciót megvalósító részeit (kattintható, szerkeszthető, görgethető elemek) egységes programozási felületen keresztül érhetjük el.

\subsubsection{Polimorfizmus}
\index{GObject@\texttt{GObject}}
\index{fordítási idejű típusellenőrzés}

Hasonlóan az öröklődéshez -- pusztán nyelvi szinten -- itt sem érhető el teljek körű megoldás \textit{C} nyelv esetén. Ugyanakkor a fő momentum, vagyis a származási hierarchia egyes osztályainak specifikus viselkedése egy adott funkciót megvalósító metódusok tekintetében elérhető. A \textit{widget}eket leíró struktúrákban ugyanis a származtatott osztályokból felülírhatóak az egyes funkciókat implementálható függvények mutatói\footnote{ami hasonló eredményre vezet, mint a \textit{C++} \texttt{virtual} kulcsszavának használata esetén}. Az így létrejövő többalakúság bár közel sem tökéletes, ám számos gyakorlati problémát megold.

Ezen felül a \textit{GTK+} minden \textit{widget}típushoz -- mondhatni osztályhoz -- definiál egy-egy makrót, melyek segítségével futásidőben ellenőrizhető egy adott \textit{widget} tényleges típusa, hasonlóan ahhoz, amire a \textit{dynamic\_cast} használata ad lehetőséget a \textit{C++}-ban. Azt a mechanizmust, melynek révén lehetővé válik a \textit{GTK+}-ban a futás idejű típusellenőrzés, a már említett \texttt{GObject} osztály implementálja, az ebből származó saját osztályainknak nem csupán lehetséges, de szükséges is a használata.

\subsection{A \textit{GTK} alapfogalmai}

\subsubsection{Widget}
\index{widget@\textit{widget}}

A fogalmat egyrészt, mint gyűjtőfogalmat használjuk a grafikus felhasználói felületek programozása során a felhasználó felületek egyes grafikai elemeinek megnevezésére\footnote{ebben az értelemben a ''window gadget'' kifejezés rövidítése}, mint amilyen például egy rádiódomb, egy szövegbeviteli mező, vagy akár egy kép.  Másrészről a \texttt{GtkWidget} minden -- ez előbbi értelemben vett \textit{widget} -- ősosztálynak a neve is -- még ha az öröklődés a \textit{C} nyelvben nem is létezik -- melyből minden egyes elem származik.

A \texttt{GtkWidget} osztály, mint a \textit{widget} fogalom objektum-orientált leképezése a felhasználó felület egyes elemeinek tulajdonságait, illetve az azokhoz kapcsolódó műveleteket zárja egységbe, biztosítva egyúttal az általa implementált funkciók újrahasznosíthatóságát éppúgy, mint a felülbírálhatóságukat. Kérdés persze, hogy ezen általánosság mögött pontosan mi is rejtőzik.

\subsubsection{Tulajdonságok}
\index{widget@\textit{widget}!tulajdonságok}

A mindennapi felhasználás -- ez esetben ugye a mindennapi szoftverfejlesztés -- során talán a leggyakrabban felmerülő kérdés -- nyilván csak azt követően, hogy megismerkedtünk milyen tulajdonságokkal bírnak a különböző \textit{widget}típusok --, hogy mik ezen tulajdonságok aktuális értéki konkrét objektumaink esetén. Ezen tulajdonságok (\textit{property}), pontosabban az általunk beállított értékeik határozzák meg \textit{widget}eink megjelenését, a felhasználói interakciókkal, illetve más \textit{widget}ekkel összefüggő viselkedését.

Ezek a tulajdonságok lehetnek egészen kézenfekvőek, mint amilyen például egy beviteli mezőben szereplő szöveg, egy folyamatindikátor értéke, egy rádiógomb be-, kikapcsolt állapota. Lehetnek teljesen általánosak, mint amilyen \textit{widget}ek neve, láthatósága, méretei, a tartalmazó konténerben elfoglal helyzetük, igazításuk. Tükrözhetnek valamilyen állapotát, mint hogy a \textit{widget} fókuszban van-e, fogad-e a felhasználói interakciókat és még sok minden más.

\subsubsection{Szignál}
\index{widget@\textit{widget}!szignálok}

Lévén egy alapvetően eseményvezérelt eszközről beszélünk a fent említett tulajdonságok értékeinél már csak a \textit{widget} által definiált események (\textit{event}) bekövetkezéséről, vagy éppen elmaradásáról való értesülés lehet fontosabb. Különösen azon esetekben mikor az esemény számunkra valamilyen szempontból jelentőséggel bír, következésképp arra reflektálni szeretnénk. Ez persze csak esetenként van így, hiszen mondjuk adatok bevitelére szolgáló ablakban egy rádiógomb állapotának változása nem különösebben érdekes, csupán csak annak aktuális értéke, mikor az ablakon található nyugtázó gombot (pl: \textit{Ok}, \textit{Alkalmaz}, \dots) lenyomtuk és az abban lévő adatoknak megfelelően szeretnénk eljárni. Ellenben ez utóbbi eseményről -- mármint hogy a gomb lenyomásra került -- minden körülmények között értesülni szeretnénk.

A \textit{widget}ek eseményeinek bekövetkeztéről a \texttt{GObject} osztály által implementált értesítési rendszer, azaz a szignálok (\textit{signal}) révén áll módunkban tudomást szerezni. Ezen mechanizmus keresztül valósítható meg az eseményekhez kezelőfüggvények (\textit{callback}) kapcsolását, ahol a felhasználó akcióra a megfelelő reakciót válthatjuk ki, az imént említett nyugtázó gomb lenyomásának hatására példának okáért bezárhatjuk az ablakot, vagy épp hibaablakot dobhatunk fel, ha a bevitt adatok a validáció során nem bizonyultak helyesnek. Itt érdemes megjegyezni, hogy a \texttt{GObject} nem csupán a \texttt{GtkWidget} osztály őse, hanem más \textit{GTK}-s elemeknek is, sőt akár saját osztályaink is származhatnak közvetlenül belőle. Azaz az olyan elemeknek is lehetnek szignáljai, melyek nem jelennek meg közvetlenül a felületen, amire egy adattároló (pl: \texttt{GtkTreeModel}, \texttt{TextBuffer}) objektum lehet jó példa, mely szignálok révén adnak jelzést arról, hogy a tárolt elemekben változás állt be.

A \textit{GTK} -- egyébiránt minden más felületprogramozási nyelv -- egyik kulcsszavának jelentése (jel, jelzés) tehát jól tükrözi funkcióját. Minden \textit{widget}hez tartoz(hat)nak különböző események -- mint amilyen egy gomb esetén annak lenyomása (vagy éppen felengedése), egy beviteli mezőnél az abba történő írás -- melyekről a \textit{widget}ek -- jellemzően \textit{GDK} révén, egy alacsony szintű\footnote{billentyű lenyomása, felengedése, egérkattintás, \dots} esemény formájában -- tudomást szereznek, majd elvégzik a megfelelő műveleteket -- gomb újrarajzolás, beviteli mezőbe karakter írása a karakter megjelenítése -- majd értesítést küldenek a program többi része felé, immár egy magasabb szinten\footnote{beviteli mező értékének változása, kattintás, \dots} értesítést. Ezt az értesítést, avagy jelzést nevezzük \textit{signal}nek.

A szignálokhoz alapvetően három tevékenységhez kapcsolódik, melyek közül kettő leginkább csak a saját \textit{widget}ek fejlesztése kerül elő, míg a harmadik gyakorlatilag még a legegyszerűbb esetekben is nélkülözhetetlen. Ez utóbbi a függvények kapcsolása (\textit{connect}) az eseményekhez, ezt azonban meg kell előzze a a másik két említett művelet. Időrendi sorrendben ez a szignálok regisztrációja \textit{register}) -- ahol meg kell adunk az eseményünk nevét, illetve paramétereit --, illetve a szignálok küldése, kibocsátása (\textit{emit}), ahol a regisztráció során megadott nevet, valamint paramétereket szükséges megadni. Nem meglepő módon az eseménykezelők kapcsolásakor szintén ezt a nevet használjuk fel, illetve meghívásukkor ezek a paramétereket kapjuk meg.

\subsubsection{Callback}
\index{callback}

Amennyiben egy adott \textit{widget}hez kapcsolódó valamilyen eseményről (\textit{event}) tudomást kívánunk szerezni a program futása során, ezt úgy tehetjük meg, hogy a \textit{widget} megfelelő típusú jelzéséhez (\textit{signal}) eseménykezelő függvényt (\textit{callback}) kapcsolunk. Itt minden olyan esemény elvégezhető, mely nem a \textit{widget}hez, hanem annak programunkban betöltött szerepéhez kötődik. A korábbi példánál maradva ha egy adatok bevitelére szolgáló ablak \textit{Ok} gombjának lenyomásánál szükséges lehet a felhasználó által megadott adatok szintaktikai, illetve szemantikai ellenőrzése, az ellenőrzött adatok mentésére, majd az ablak bezárására, probléma esetén hibaablak feldobására, valamint a beviteli folyamat újrakezdésére.  Egyébiránt az egyes tulajdonságok (\textit{property}) megváltozása szintén események minősül, vagyis ezekhez is módunk van függvényeket csatolnunk.

\subsection{A \textit{GTK+} működési sajátosságai}

\subsubsection{Main Loop}
\index{operációs rendszer!Mac OS X@\textit{Mac OS X}}
\index{operációs rendszer!Windows@\textit{Windows}}
\index{main loop}
\label{sec:mainloop}

Az események kezelése kapcsán megválaszolandó az a triviálisan adódó kérdés, hogy az egyes \textit{widget}ek hogyan szereznek tudomást a rajtuk -- a felhasználók, vagy éppen az automata tesztelő eszközök által -- végrehajtott akciókról. A válasz pedig épp ugyanabban rejlik, mint bármely más felhasználói felületek fejlesztésére szolgáló eszközkészlet esetén, azaz az eseményvezérelt működési modellben. Ez nem jelent más mint, hogy a \textit{GTK+} mindaddig várakozik, amíg valamilyen forrásból létre nem jön egy esemény (pl: egér mozgatása, gombjainak lenyomása, billentyű felengedése, \dots), ha ez megtörténik akkor meghatározza, melyik \textit{widget}et érinti az esemény, meghívja a \textit{widget} megfelelő eseménykezelő függvényét, majd újabb várakozásba kezd.

Ennek a várakozási ciklusnak az implementációja \textit{main loop}. A \textit{GTK+} tulajdonképpeni főciklusa a \textit{Glib} függvénykönyvtárban implementált, a \textit{GTK+} ezt újra felhasználva ciklizál, várva a grafikus szerver -- legyen az a \textit{Linux}, a \textit{Mac OS X}, vagy a \textit{Windows} megfelelő alrendszere -- üzeneteire. Mindezt a \textit{GDK}-n keresztül teszi, mely -- mint azt az előző részben is említettük (\ref{sec:gdk}) -- egy vékony burkoló réteg az ablakozó rendszer köré. A \textit{main loop} tehát az aki az imént említett kapcsolaton át eljuttatja az ablakozó rendszer alacsony szintű eseményeit a \textit{GDK} által standardizált formában az egyes \textit{widget}ekhez, hogy ezek a feldolgozást követően egy magasabb szintű eseményt váltsanak ki a többi \textit{widget}, illetve az applikáció más részei felé.

\subsubsection{Referencia-számlálás}
\index{referencia-számlálás}
\index{lebegő referencia@''lebegő'' referencia}

A \textit{GTK+} segítségével létrehozott felületek -- ahogy azt a későbbiekben látni fogjuk -- nem \textit{widget}ek szórvány halmazát, hanem egymással szoros összefüggésben álló (szülő-gyerek, modell-nézet- vezérlő kapcsolat) láncolatát jelentik. Ennek okán az megoldandó feladat, hogy az egymáshoz valamilyen szempont alapján kötődő elemek egymás oly módon tudják hivatkozni, hogy hivatkozások létrejötte, illetve megszűnése egyúttal a \textit{widget}ek memória menedzsmentjére is megoldást adjon. Ennek bevett módszere a referenciák tartása a hivatkozott elemekre, mellyel a \textit{GTK+} is él.

Minden \texttt{GObject}ből származó osztály -- így a \texttt{GtkWidget} is -- rendelkezik referencia-számmal, mely tulajdonképpen azt fejezi ki, hogy hányan hivatkoznak az adott elemre. A \textit{GTK+} -- pontosabban ez esetben a \textit{GLib} -- ''lebegő'' referenciát (\textit{''floating'' reference}) alkalmaz, mely azt jelenti, hogy az objektum létrejöttekor annak referenciája 1 lesz, bár a \textit{widget}re ekkor még nem hivatkozik semelyik másik \textit{widget} sem, azaz ezt a referenciát úgymond nem birtokolja senki. Amikor egy \textit{widget}re megszületik ez első valódi hivatkozás, például egy konténer osztályba -- mint amilyen egy közönséges ablak is -- tesszük a \textit{widget}et, vagyis megjelenik az első valódi hivatkozás az elemre, akkor az a hivatkozó birtokába kerül. A referencia-érték változatlanul 1 marad, viszont a lebegő referencia elsüllyesztésre (\textit{sink}) kerül. Minden ezt követő esetben a konténerből történő eltávolítás csökkenti, ahhoz való hozzáadás pedig növeli a referencia értékét. Érdemes felhívni a figyelmet arra, hogy az elmondottak alapján, ha hozzáadtuk \textit{widget}ünket egy konténerhez, majd pedig eltávolítjuk belőle azt, akkor annak referenciája 0-ra csökken, ami maga után vonja a \textit{widget} destruktorának lefutását. Ezt elkerülhetjük, ha az eltávolítás előtt explicit módon növeljük a referenciát, amit aztán csökkentenünk kell, ha egy másik osztály ``birtokába'' adjuk a \textit{widget}et.

\subsubsection{Szülő-gyerek kapcsolat}
\index{szülő-gyerek kapcsolat}

A szülő-gyerek kapcsolat a már említett konténerek -- azaz a \texttt{GtkContainer}, illetve az abból származó osztályok -- viszonylatában merül fel. Ezen elemek teszik lehetővé a \textit{widget}ek felületen való elrendezését (ablakok, táblázatok, gombok, \dots), egymásba ágyazását. A konténer tehát az a \textit{widget}, amely további \textit{widget}et, vagy \textit{widget}eket tartalmazhatnak. Ilyen értelemben tehát egy szülő-gyerek kapcsolatot valósítanak meg, ahol minden szülőnek lehetnek gyermekei, de egy -- ebben az értelemben -- gyermek \textit{widget}nek minden esetben csak egy szülője lehet, vagyis a szülők és a gyerekek egy fa hierarchiát alkotnak.

Ez a szerkezet több szempontból is fontos szerepet játszik a \textit{GTK+} működése során. Egyrészről a referencia-számlálásnál már említett módon, azaz ha egy \textit{widget}et hozzáadunk egy konténerhez, akkor az úgymond tart rá egy referenciát -- vagy a referenciaszám növelésével, vagy ''lebegő'' referencia elsüllyesztésével --, majd elereszti azt a konténerből való eltávolításakor. Másrészről egy még nem ismertetett -- a szülő- és gyerek\textit{widget}ek viszonyának tulajdonságait rögzítő -- mechanizmust tesz lehetővé, melyről a későbbiekben még részletesebben esik szó. Elöljáróban csak annyit, a \textit{widget}ek saját tulajdonságain (\textit{property}) túl, léteznek olyanok is melyek szülő \textit{widget}ekkel való kapcsolatára jellemzőek, mint például a gyerek \textit{widget}ek elhelyezkedése a konténerben (pozíció, térköz, kiterjedés, \dots).

\subsubsection{Interfészek}
\index{interfész}

A \textit{GTK+}, erősítve az objektum-orientált megközelítést, olyan absztrakciós rétegeket definiál, melyek adott funkciótat (szöveg bevitel, aktiválhatóság, igazítás) betöltő \textit{widget}ek implementálnak. Az ilyen típusú általánosítások komoly haszonnal bírnak, mikor az adott funkciót egységes felületen keresztül szeretnénk elérni.

Függetlenül attól, hogy például egy, vagy többsoros beviteli mezőről legyen szó, a karaktereket épp úgy szeretnénk kiolvasni mindkét esetben. Éppúgy igaz ez az elrendezés (\textit{orientation}) tekintetében, hisz amennyiben a megfelelő \textit{widget}ek -- jellemzően a konténerek -- megvalósítják ezt a felületet, a vízszintes, illetve függőleges orientáció futás közben is könnyedén váltható (\textit{flip}). Van egy olyan interfész, amit minden egyes felülettel rendelkező \textit{widget} (\texttt{GtkWidget}) és számos felülettel nem rendelkező objektum (\texttt{GObject}) is implementál (\texttt{GtkBuildable}). Ezen osztályon keresztül valósulnak meg a legalapvetőbb funkciók, mint amilyenek a az objektumok nevének, tulajdonságok értékének lekérdezése, beállítása, a gyerek \textit{widget}ek létrehozása\footnote{mára a \textit{Glade} elnevezésű felhasználói felületek tervezésére szolgáló alkalmazás is ezt az interfészt használja}.

\section{A tesztelés fogalmai}

A tesztelési feladatok ellátása kapcsán a \textit{GTK} ismerete bizonyos esetekben nem árt, míg más esetekben nem használ, azon egyszerű oknál fogva, hogy a megközelítés itt merőben eltérő, lévén nem közvetlenül a \textit{GTK+}-ra , hanem az \textit{ATK}-ra épít.

\subsection{Az \textit{ATK} koncepciója}

A már említett (\ref{sec:atk}) \textit{ATK} koncepciójában némiképpen különbözik a \textit{GTK+}-től. Lévén ezt az interfészt a fogyatékossággal elő emberek szükségleteinek kielégítésére tervezték, elsősorban nem magukra a \textit{widget}ekre, vagy azok kapcsolataira, felületi megjelenésére, hanem az általuk hordozott információkra koncentrál. Ezen információk rejtőzhetnek természetesen a \textit{widget}ek által megjelenített szövegekben, vagy számszerű értékekben éppúgy, mint a \textit{widget}ek aktuális állapotában (aktív, szerkeszthető, látható, \dots), ugyanakkor persze az egymás közi viszonyok (tartalmazás, vezérlés, \dots) is meghatározóak lehetnek.

Ezen interfészek, állapotok, illetve viszonyok függetlenek a konkrét implementációtól. Bármely \textit{widget}készlet számára implementálhatóak, sőt implementálandóak, ami annyit tesz, hogy az egyes \textit{widget}készletek logikája még ha nagyjából egyezik is az \textit{ATK} logikájával, lévén erre tervezték, számos ponton kisebb-nagyobb eltérések tapasztalhatóak, amiket szükséges valamilyen, a \textit{widget}készlet és az \textit{ATK} között elhelyezkedő, azokat összekötő (\textit{bridge}) implementációval áthidalni.

\subsubsection{Interfészek}

A kifejezetten a funkcionalitáson alapuló megközelítés egyenes következményei az \textit{ATK} interfészei. Ezek gyakorlatilag a grafikus felhasználói felület elemeit legfőbb funkcióik szerint csoportosító eszközök. Egy adott \textit{GTK} \textit{widget} természetesen megvalósíthat több interfész is, lévén többféle funkciót is betölthet. Egy egyszerű példával megvilágítva a helyzetet egy közönséges gomb (\texttt{GtkButton}) egyszerre valósítja meg a szöveg (\texttt{AtkText}), illetve a képek (\texttt{AtkImage}) lekérdezésére szolgáló interfészeket, lévén a gombon kép és felirat egyaránt lehet.

\subsubsection{Állapotok}

Bizonyos esetekben nem a \textit{widget} által tartalmazott adat -- legyen az szám, vagy szöveg, esetleg kép -- hordozza az információt, hanem egy tulajdonság megléte vagy hiánya. Általánosságban véve ilyen lehet egy \textit{widget} láthatósága, egy ablak átméretezhetősége, egy beviteli mező szerkeszthetősége, vagy akár egy rádiógomb aktív mivolta. Ezen tulajdonságok mindegyike -- ahogy a többi állapot is -- bináris, azaz igaz vagy hamis értékkel írható le. Ennek megfelelően az \textit{ATK} által definiált állapotok egy bithalmazt alkotnak, melyek leírják egy adott \textit{widget} konkrét időpillanatban vett állapotát.

\subsubsection{Viszonyok}

Tesztelési szempontokból létezik még egy jelentőséggel bíró tulajdonságtípus, mely azonban nem konkrét \textit{widget}ek paramétereit, hanem azok egymáshoz kötődő viszonyát írják le. Úgyis, mint egy felirat (\textit{label}) és a hozzá kötődő \textit{widget} összetartozását, a vezérlő és vezérelt \textit{widget} között fennálló kapcsolatot, vagy éppen a fák megjelenítésére használt \textit{widget} esetén a az elemek szülő-gyerek kapcsolatai. Az egyes viszonyok (\textit{relation}) -- hasonlóan az imént említett állapotokhoz -- szintén két értékeket vehetnek fel, bár ellentétben azokkal az állapotokkal az egyes viszonyok csak egy másik \textit{widget}tel együtt van értelmük. Az ellentétes előjelű viszonyok, mint a vezérlő és vezérelt \textit{widget} egyidejűleg is igazak lehetnek, más-más \textit{widget}ekkel összefüggésben.

\subsection{A \textit{Dogtail} működése}

A \textit{Dogtail} a szoftverek akadálymentesítésének megvalósítására használta technológiai módszereket (\textit{assistive technologies}) alkalmazza a tesztelendő alkalmazások vezérlésére.

\begin{figure}[h]
\begin{center}
\includegraphics[height=0.5\linewidth]{images/a11y.png}
\caption{Akadálymentesített szoftverek elérése}
\label{fig:a11y}
\end{center}
\end{figure}

A működés modell (\ref{fig:a11y} ábra) végeredményben nem túl bonyolult. Az applikáció közvetlenül nem szólítható meg. Ahogy ezt hang alapú vezérlést megvalósító, illetve képernyőolvasó szoftverek is teszik, az \textit{AT SPI}-n (\textit{Assistive Technologies Service Provider Interface}) keresztül szólítják meg a megfelelő szoftvert. Amennyiben ez egy \textit{GTK+} felhasználásával fejlesztett alkalmazás, akkor a kérésre a \textit{Gail} (\ref{sec:gail}) nevű alrendszert futtatva válaszol, az \textit{ATK} interfészben leírtaknak megfelelően.


\chapter{A fejlesztés menete}
\section{\textit{Gimp Tool Kit}}

\subsection{Beszerzése}
\index{GTK}

A \textit{GTK+} beszerzésére alapvetően két módszer kínálkozik. Az egyik megoldás, hogy hagyatkozunk az általunk használt operációs rendszerre és az általa biztosított, vagy legalábbis arra elérhető változatot telepítjük. A másik lehetőség, hogy letöltjük a \textit{GTK+} és a függőségek forráskódját és némi nehézséget vállalva ezeket fordítjuk le. Előbbi eset bőségesen megfelel amennyiben még csak most ismerkedünk a \textit{GTK+} függvénykönyvtárral, illetve nem akarunk túlságosan belebonyolódni az grafikus alkalmazások fejlesztésébe. Elkerülhetetlenül az utóbbit kell azonban választanunk, ha az átlagnál jobban el szeretnénk merülni a \textit{GTK+} rejtelmeiben, ha esetleg hibákat javítanánk, vagy változásokat eszközölnénk magán a grafikus eszközkészleten.

\subsubsection{Bináris változat}
\index{GTK}
\index{deb}
\index{rpm}
\index{bináris csomag}

A \textit{GTK} változatok beszerzése nem jelent különösebb feladatot, amennyiben valamelyik népszerű \textit{Linux} disztribúciót használjuk, hiszen azok nagy valószínűséggel már amúgy is telepítve vannak az általunk használt rendszeren, lévén vélhetőleg már használunk ezen eszközök segítségével fejlesztett szoftvereket. A fejlesztéshez, illetve teszteléshez a bináris változatokon túl fejlesztői csomagokra is szükséges lesz, amit \textit{rpm}, illetve \textit{deb} alapú disztribúciók esetén rendre az alábbi parancsok kiadásával tehetünk meg:

\lstcommand{sudo yum install gtk-devel-package}\\
\lstcommand{sudo apt-get install gtk-dev-package}

\paragraph{GTK+}
\index{nyelvi változatok!GTK+@\textit{GTK+}}
\index{C}

A \textit{GTK+} fejlécfájlokat és egyéb állományokat -- melyekről a későbbiekben (\ref{sec:compilingandlinking}) még részletesebben is esik szó -- a \textit{Debian}/\textit{Ubuntu}, illetve \textit{Fedora} rendszereken a \texttt{libgtk-3-dev}, illetve a \texttt{gtk3-devel} csomagok tartalmazzák.

\paragraph{gtkmm}
\index{nyelv változatok!gtkmm@\textit{gtkmm}}
\index{C++}

A fenti csomagok természetesen csak a \textit{C} nyelvű változat -- azaz a \textit{GTK+} -- használatához elegendőek, amennyiben a \textit{C++} nyelvet -- ezzel együtt a \textit{gtkmm} függvénykönyvtárat -- kívánjuk használni, további csomagokra (\texttt{libgtkmm-3.0-dev}, vagy \texttt{gtkmm3-devel}) is szert kell tennünk.

\paragraph{PyGObject}
\index{nyelv változatok!PyGObject}
\index{Python}

Amennyiben a \textit{GTK} alapú fejlesztéssel a \textit{Python} nyelv révén ismerkednénk a már említett fordított nyelvek helyett, akkor a fenti csomagokat nem, a \texttt{python-gi}, vagy a \texttt{python-gobject} csomagokat viszont be kell szereznünk.

\paragraph{Dogtail}
\index{Dogtail@\textit{Dogtail}}
\index{Python}

Az automata tesztek készítéséhez szükséges \textit{Python} függvénykönyvtár -- a \textit{Dogtail} -- állományait az azonos nevű csomag tartalmazza, melyek installálása a fentiekhez hasonlóan történik.

\subsubsection{Forráskód}

A \textit{GTK} forrásának beszerzésére szintén több módszer kínálkozik. Egyrészről az általunk használt \textit{Linux} disztribúció biztosít eszközöket forráscsomagok telepítésére. A korábbi \textit{deb}, illetve \textit{rpm} alapú rendszerek példájánál maradva ez rendre az alábbiak szerint történik.

\lstcommand{apt-get source gtk-src-package}\\
\lstcommand{yumloader --source gtk-src-package}\\

Lehetőség van természetesen az egyes verziók letöltésére a \textit{GNOME} projekt weboldaláról is,

\lstcommand{wget http://ftp.gnome.org/pub/gnome/sources/gtk+/major.minor/gtk+-major.minor.micro.tar.gz}\\
\lstcommand{wget http://ftp.gnome.org/pub/gnome/sources/gtk+/major.minor/gtk+-major.minor.micro.tar.bz2}\\

valamint használhatóak e célra egyes projektek verziókelői is, ha szeretnénk mindig az aktuális forráskóddal dolgozni.

\lstcommand{git clone git://git.gnome.org/gtk+}\\

Ha azonban magunk szeretnénk a teljes \textit{GTK}-t fordítani -- legyen szó a \textit{C}, vagy a \textit{C++} nyelvű változatról -- számolnunk kell azzal, hogy számos egyéb komponens (\textit{GLib}, \textit{Pango}, \textit{Cairo}, \textit{ATK}, \dots) fordítására, illetve az frissítéseket követő újrafordítására válik szükségessé, ami meglehetősen időigényes és fáradságos feladat, amit a \textit{Linux} disztribúción összeállítói már megtettek helyettünk. Így célszerű kezdetben ezt kihasználni és a saját fordításba csak akkor belekezdeni, ha arra feltétlenül szükségünk.

\subsection{Fordítása}
\index{Autotools@\textit{Autotools}}
\index{Automake@\textit{Automake}}
\index{Autoconf@\textit{Autoconf}}
\index{Libtool@\textit{Libtool}}

Amennyiben a korábban említett nehézségek ellenére mégis nekivágunk a \textit{GTK} saját fordításának, akkor sem vagyunk magunkra hagyva. A \textit{GNOME} projekt része egy \textit{JHBuild} elnevezésű szoftver\footnote{telepítése a korábbiaknak leírtaknak megfelelőn történik}, amit a \textit{GNOME} projekt moduljaiból összeállított halmazok -- mint amilyen a \textit{GTK} és annak függőségei -- letöltésére, frissítésére, fordítására és fordítás eredményeként létrejött futtatási környezetben való munkára használhatunk.

Mindenekelőtt azonban szükségünk lesz néhány olyan eszközre, amik a \textit{Linux} alapú rendszereken fordítási feladatok ellátására kvázi szabványnak számítanak. A \textit{GNOME} -- sok más fejlesztési projekthez hasonlóan -- az \textit{Autotools}t\footnote{a \textit{GNU} fordítási rendszere, mely tulajdonképpen az \textit{Automake}, \textit{Autoconf}, \textit{Libtool} együttese.} használja moduljainak fordításához. Ennek részleteibe nem célunk ezen dokumentum keretében elmerülni, már csak azért sem, mert a \textit{JHBuild} ezen, fordításhoz szükséges, függőségek telepítését megoldja helyettünk.

\lstcommand{jhbuild sanitycheck}\\
\lstcommand{jhbuild bootstrap}

Az parancs futtatásának hatására ellenőrzésre kerül a konfigurációban megadott könyvtárak írhatósága, a szükséges fordítási eszközök telepített mivolta. Ezt követően kerülhet sor a forráskódok letöltésére, a fordítás előtti konfigurálásra, magára  fordításra, valamint az elkészült bináris állományok telepítésére. Ehhez a következő parancsokat használhatjuk.

\lstcommand{jhbuild update}\\
\lstcommand{jhbuild make}

A fordítás végeztével lehetőségünk van az újólag létrejött környezetbe úgymond belépni, vagy közvetlenül parancsokat futtatni. Ez gyakorlatilag ennyit tesz, hogy számos környezeti változó beállításának eredményeként saját a fordítandó alkalmazásaink és a fordítás eredményeként létrejött futtatandó állományok is a az új környezetben létrehozott függvénykönyvtárakat fogják használni a rendszeren található változatuk helyett.

\lstcommand{jhbuild run}\\
\lstcommand{jhbuild shell}

Ennek akkor van leginkább haszna, ha szeretnénk az általunk fejlesztett alkalmazást a rendszeren elérhető \textit{GTK} változaton kívül a legfrissebb verzión is kipróbálni, \textit{GTK} valamely modulján szeretnénk változtatni és ennek hatását látni alkalmazásunkra, vagy éppen csak nyomon követnénk a \textit{GTK} változásaik, aktuális fejlesztéseit még mielőtt azok az általunk használt disztribúcióban is megjelenik.

\section{Saját alkalmazások}

Mivel a továbbiakban részletesen foglalkozunk majd a \textit{GTK} alapú alkalmazások létrehozásával, itt most csak a legfontosabb parancsokat vesszük számba, melyek révén a \textit{C}, illetve \textit{C++} nyelvű forrásfájlokból futtatható bináris állományokat hozhatunk létre.

\subsection{Fordítás és linkelés}
\label{sec:compilingandlinking}

Akár a rendszeren található, akár \textit{JHBuild}, vagy más eszköz révén létrehozott környezetben lévő \textit{GTK} változatot is használunk, az alábbi parancssorok segítségével fordíthatóak le forrásfájljaink.

\lstcompiles
{gtk\_sourcefile.c}{gtk\_binary}
{gtkmm\_sourcefile.cc}{gtkmm\_binary}

Ahhoz, hogy a \textit{GTK+}, illetve \textit{gtkmm} fordítási függőségeit ne magunknak kelljen megadnunk a \texttt{pkg-config} parancsot hívhatjuk segítségül, hogy a \textit{GCC} részére a megfelelő paramétereket meg tudjuk adni. A \texttt{--cflags} paraméter hatására a fordításhoz, míg a \texttt{--libs} eredményeképp a linkeléshez szükséges opciókat kapjuk vissza. A parancs két \texttt{\`} (backtick) közé zárt. aminek hatására a program kimenete része lesz a fordító parancssorának, amivel pont az kívánt hatást érjük el.

\subsection{Futtatás}

Ezek után már csak az örömteli pillanat van hátra, mikor a két különböző nyelven és függvénykönyvtárral lekódolt teljesen azonos funkciójú programunkat lefuttatjuk a \texttt{./gtk\_binary}, illetve a \texttt{./gtkmm\_binary} paranccsal.

Amennyiben a \textit{Python} nyelvű változat mellett tesszük le voksunkat a fordítás, mint lépés kimarad, a futtatás történhet közvetlenül (\texttt{./gtk\_script.py}), amennyiben van az adott fájlon futtatási jog, vagy a \textit{Python} interpreternek paraméterként (\texttt{python gtk\_sctipt.py}). A továbbiakban ezt az utóbbi sémát követjük.

Ezzel túl is vagyunk azon a rövid áttekintésen ami után már épp ideje nekilátnunk első ablakunk implementálásának, futtatásának és tesztelésének.


\chapter{Első ablakunk}
\section{Kódolási alapismeretek}

A nagyobb nyílt forrású projektek a kódolás, kódszervezés során egy meghatározott konvenciót követnek, bár egy olyan méretű projekt esetén, mint a \textit{GNOME} az egyes részterületeken lehetnek eltérések, azzal együtt is, hogy az azonosságok nyilván erős többségben vannak. Lássuk mik ezek a \textit{GNOME}, illetve a \textit{GTK} projektek esetén.

\subsection{Forráskód formázása}

A \textit{GTK} fejlesztői a \textit{GNU coding standard}\cite{gnucodingstandards}, illetve a \textit{Linux kernel coding style}\cite{linuxcodingstyle} irányelveit alkalmazzák, ami mindaddig csak az olvasást, megértését elősegítő módszertani eszköz, amíg nem áll szándékunkban a \textit{GTK} fejlesztésébe, javításába belefogni, ugyanakkor két oknál fogva mégis érdemes megemlíteni. Ha még nem ismerkedtünk meg egyetlen kódolási konvencióval sem, akkor az említett kettő -- mind népszerűségük, mind letisztult mivoltuk okán -- alkalmas választás lehet. A másik ok, hogy néhány a fejlesztés során hasznos információ ezen konvenciókból következik.

\subsection{Elnevezési konvenciók}
 \index{GTK@\textit{GTK}!kódolási konvenciók}
 \index{GTK@\textit{GTK}!kódolási konvenciók!formázás}
 \index{GTK@\textit{GTK}!kódolási konvenciók!nevezéktan}

A \textit{GNOME} projekten belül nem csak formázási, de elnevezési konvenciók is használatosak. Ezek az egyes funkciót megvalósító szoftverelemekre (pl: függvények, makrók, \dots) vonatkoznak, amik közül a az alábbiak már a legegyszerűbb példák esetén is feltűnnek.

\begin{itemize}
  \item az egyes nevek részekre oszthatóak,
  \item a részek meghatározott sorrendben követik egymást, ahol
  \begin{itemize}
    \item kezdve a \textit{GNOME} megfelelő projektjével (pl: \texttt{atk}, \texttt{gtk}, \dots),
    \item folytatva a vonatkozó osztály nevével (pl: \texttt{entry}, \texttt{label}, \dots),
    \item befejezve a megvalósított művelettel (pl: \texttt{get}, \texttt{set}, \dots),
    \item illetve a művelet tárgyával (pl: \texttt{text}, \texttt{value}, \dots),
  \end{itemize}
  \item a részeket aláhúzásjel ('\texttt{\_}') választja el,
  \item az egyes részek rendszerint vagy csak kis-, vagy csak nagybetűket, illetve számokat tartalmaznak.
\end{itemize}

Fentieknek megfelelően egy -- a \textit{GTK+} által implementált -- rádiógomb aktív mivoltát lekérdező függvény neve a \texttt{gtk} előtaggal kezdődik, amit a osztálynév, vagyis a \texttt{radio\_button}, majd lekérdezésről lévén szó \texttt{get} akciónév követ, végül pedig a tulajdonság neve (\texttt{actvie}) zár. Aláhúzás jelekkel összefűzve \texttt{gtk\_radio\_button\_get\_active}.

A \textit{C++}, illetve a \textit{Python} nyelvű változatok esetén is hasonló az elnevezés módszertana a nyelvből fakadó sajátosságok okozta eltéréssel természetesen. A \textit{gtkmm} esetén a projektek nevét tartalmazó prefixum szerepét a \textit{Gtk} névtér veszi át, míg az osztályok neve a \textit{C++} osztályok neve lesz. A tagfüggvények az előbbi két előtag (pl: \texttt{gtk\_entry}) nélküli nevek lesznek (pl: \texttt{set\_text}). A \textit{PyGobject} esetén a névtér szerepét a \textit{Gtk} modulnév veszi át, a \textit{Python} osztályok nevei ugyanazt a szerepet töltik be, mint a \textit{C++} esetén.

\subsection{Fejlécfájlok és importálás}
 \index{GTK@\textit{GTK}!fejlécfájlok}

Szakítva az előző főverziónál (\textit{2.x}) megszokottaktól az új főverzió (\textit{3.x}) esetén, a \textit{C}, illetve a \textit{C++} változat egyaránt csak egyetlen állomány beszerkesztésére (\textit{include}) van lehetőség és szükség. A korábbiakban az egyes \textit{widget}ekhez tartozó fejlécállományok (pl: \texttt{gtk/gtkentry.h}) beszerkesztésére külön-külön volt lehetőség függően attól, melyekre van, illetve melyekre nincs szükségünk. Most azonban közvetlenül csak a \texttt{gtk/gtk.h} szerkeszthető be, a többi fejlécállomány esetén hibaüzenetet kapunk. Így mind a \textit{C}, mind a \textit{C++} változat azonos módon működik, ami egyébiránt hasonlít a \textit{Python} modul importjára.

\section{Minimálisan alkalmazás}
\label{sec:gtkminimal}
\label{sec:gtkmmminimal}

Ennyi bevezető után lássuk egymás mellett (\ref{lst:gtkminimal}. Kódrészlet) a három nyelvi változat (\textit{C}, \textit{C++}, illetve \textit{Python}) kódját számba véve azok hasonlóságait és különbözőségeit. Ami talán első látásra is feltűnő, hogy messze a \textit{GTK+} változat ''kódsűrűsége'' a legnagyobb, vagyis a \textit{C} nyelvű változat igényli ugyanazon funkcionalitás mellett a legtöbb kódsor (\ref{gtkminimalc:end} sor) begépelését, és egyben a legtöbb munkát is. Ez persze nem jelenthet különösebb meglepetést ha van némi tapasztalatunk az interpretált, illetve a fordított, a procedurális, illetve az objektum központú nyelvek esetén elérhető fejlesztési sebesség terén.

\subsection{Forráskód}

Első közelítésben a már említett formai különbségek lehetnek szembeötlőek, ugyanakkor számos, a tartalmat, megvalósítást érintő eltérés is felfedezhető ebben a még oly kevést kódsort tartalmazó példában. Ezek megértése nagyban könnyíti az egyes nyelvi változatok közötti átjárást és ne utolsó sorban a \textit{GTK} koncepcionális sajátosságaira is rávilágít. Lássuk tehát sorról sorra az imént olvasott kódok magyarázatát.

\lsttriplesource
{sources/gtk_minimal.c}
{sources/gtk_minimal.cc}
{sources/gtk_minimal.py}
{Minimálisan szükséges kód \textit{GTK+}, \textit{gtkmm}, illetve \textit{PyGobject} használata mellett}
{lst:gtkminimal}

\begin{description}
 \index{GTK@\textit{GTK}!fejlécfájlok}
 \item[\ref{gtkminimalc:include}. sor] A \textit{header} fájlok beszerkesztésének különbözőségeiről az előzőekben esett szó, így erre itt csak azt megemlítendő térünk ki, hogy a \textit{Python} változat \texttt{import} parancsának azt a változatát alkalmazzuk, ami a legerősebb hasonlóságot eredményezi a másik két nyelvi változattal, már ami a \texttt{gtk} kulcsszó forráskódban történő megjelenéseinek helyét illeti.

 \item[\ref{gtkminimalc:main}. sor] Ez a sor a programjaink belépési pontja, azaz itt kezdődik meg a futtatás, már legalábbis ami a \textit{C}/\textit{C++} változatot illeti. A \textit{Python} nyelv esetén az \texttt{import} parancs már végrehajtásra került mire ide jutunk, sőt erre a kódsorra voltaképpen nincs is feltétlenül szükséges\cite[fej.~27.4.]{pythonlib}, leginkább csak a másik két példához való még erősebb hasonlóság végett került be ez a kódsor. Fontossá a \texttt{main} függvények tulajdonképpen csak a parancssori paraméterek \textit{GTK}-nak történő átadás\ref{gtkminimalc:gtkmain} szempontjából válnak.

 \item[\ref{gtkminimalc:windowdeclare}. sor] A \textit{C} nyelvi verzióban kénytelenek vagyunk blokk elején deklarálni azt a változót -- ami ez esetben \textit{widget}ünk címét tartalmazza majd -- mivel az \textit{ISO C90} szabvány még nem, majd csak az \textit{ISO C99} támogatja a blokkon belül kifejezés után elhelyezett változó deklarációkat. Ezt viszont sem \textit{3.0}-nál korábbi \textit{GCC}, sem pedig a \textit{Microsoft Visual Studio} \textit{C} fordítója nem támogatja, vagyis problémába ütköznénk a \textit{C++} példában használt módszerrel, így a biztonság kedvéért maradunk a hagyományoknál, azaz lokális változók deklarációja csak blokkok kezdetén szerepel.

 \index{Main@\texttt{Main}}
 \index{Gtk@\texttt{Gtk}!függvények!init@\texttt{init}}
 \item[\ref{gtkminimalc:gtkmain}. sor] Eljutottunk végre az első \textit{GTK} specifikus híváshoz, mely a \textit{Python} változatból teljesen hiányzik, míg a \textit{C}, illetve a \textit{C++} verzióban funkciójuk azonos, mégis van köztük egy árnyalatnyi különbség. A \textit{GTK+} esetén az \texttt{argc}, valamint az \texttt{argv} változók címeit adjuk át, biztosítandó, hogy az \textit{init} függvény a \textit{GTK} saját paramétereit \footnote{a \textit{GTK} által értelmezett parancssori paramétereket összefoglalója \href{http://library.gnome.org/devel/gtk/stable/gtk-running.html}{itt} olvassható.} el tudja távolítani a tömbből és azok számával csökkenteni tudja \texttt{argc} értékét. Erre a \textit{C++}-os változat esetén erre azért nincs szükség, mert még ha nem is látszik, mindkét változóra referencia adódik át a \texttt{Gtk::Main} konstruktorának. A \textit{Python} változat esetén nincs \texttt{init} jellegű függvényhívás, mivel az inicializálás a modul (\textit{gi.repository.Gtk}) importálásával implicit módon megtörténik, másrészről a parancssori paraméterek a \textit{sys} modulon keresztül bárhol hozzáférhetőek, azok paraméterként való átadása így felesleges.

 \index{GtkWindow@\texttt{GtkWindow}!függvények!new@\texttt{new}}
 \item[\ref{gtkminimalcc:windowdeclare}. sor] Első \textit{widget}ünk létrehozása a már említett nevezéktan szerinti függvények meghívásával történik. Minden \textit{widget}típushoz létezik egy, mondjuk úgy konstruktor, melyet meghívva egy új -- az adott típushoz tartozó -- \textit{widget}et kapunk vissza. A hívás mikéntje természetesen függ a nyelvtől magától, illetve attól is, hogy a nyelv esetén használható-e, illetve használjuk-e az objektumok-orientált megközelítést. A \textit{C} nyelvi változat esetén a \texttt{gtk\_widgettípusnév\_new} forma használatos, addig a \textit{C++} esetében a prefixek szerepét a névterek veszik át, tehát általános formában a \texttt{Gtk::WidgetTípusNév::WidgetTípusNév} írható le a konstruktor. A \textit{Python} szintén névterekkel operál, ahol azok határait a modulok jelentik, melyek neveit egymástól, illetve függvényeiktől pont (\texttt{.}) választja el, vagyis a konstruktor \texttt{Gtk.WidgetTípusNév} formában írható le.

 \index{lebegő referencia@''lebegő'' referencia}
 A különbség nem is annyira a nevekben, mint inkább a memória kezelésében rejlik, hiszen egy újólag létrehozott objektum felszabadításáról \textit{C} esetén magunknak kell gondoskodnunk, míg a \textit{C++} a tőle megszokott módon felszabadítja a lokális változókat. Ugyanakkor érdemes itt visszautalni a korábbiakban már említett referencia számlálásra, illetve a ''lebegő'' referenciára, melyek révén a különböző nyelvi változat esetén mód van arra, hogy csak a legfelső szintű elemről kelljen ebben a tekintetben magunknak gondoskodnunk, a \textit{GTK} a többi elem memóriakezelését maga menedzseli.

 Az ablak létrehozásában meg egy különbség fedezhető fel a \textit{C}, illetve a másik két változat között.Ez pedig a paraméterkezelés mikéntje. Előbbi esetben meg kell mondanunk az ablak típusát (toplevel) -- hiszen a \textit{C} nyelv nem tesz lehetővé a paraméterek esetén alapértelmezett értékét --, míg a másik két esetben erre nincs szükség. Ez a paraméter ugyan mindkét esetben létezik alapértelmezett értékük, pont az, amit a \textit{C} változat esetén használunk.

 \index{widget@\textit{widget}!szignálok!delete-event@\texttt{delete-event}}
 \index{Main@\texttt{Main}!függvények!run@\texttt{quit}}
 \item[\ref{gtkminimalc:windowdelete}. sor] Anélkül, hogy a szignálok kezelésének rejtelmeiben elmerülnénk egy gondolatnyi kitérőt érdemes ezen kódsor a kapcsán tenni. Elsőként azt érdemes tisztázni mire szolgál a \texttt{delete-event} szignál. Ez a szignál akkor váltódik ki, amikor az ablakunk felhasználó interakció hatására záródik be. Ez többféle interakciót is jelenthet -- operációs rendszertől és ablakkezelőtől függően --, de alapvetően az ablak jobb felső sarkában\footnote{\textit{Mac OS X}, illetve az újabb \textit{Ubuntu} verziók esetén bal felső sarok} lévő \textit{X} gomb, vagy az \texttt{Alt+F4} billentyűk lenyomására kell gondolnunk.

  A \textit{C}, illetve a \textit{Python} nyelvű változat ezen esemény bekövetkeztekor a \textit{GTK} főciklusát szeretné leállítani, ami annak révén ér el, hogy a \texttt{delete-event} szignálra a nyelvi változatnak megfelelő \texttt{main\_quit} függvényt köti fel. Figyelembe véve, hogy a \texttt{delete-event} alapértelmezett szignálkezelője megszünteti (\textit{destroy}) az ablakot, ez az eljárás logikus, lévén egy programnak főablak nélkül nincs igazán sok értelme. A \textit{C++} változat viszont ugyanebben a tekintetben látszólag semmilyen lépést sem tesz, ugyanakkor megfigyelhetjük, hogy néhány sorral lejjebb (\ref{gtkminimalc:gtkrun}. sor) a \texttt{run} függvények paraméterként átadja azt ablakot, ami nagyon hasonló eredményre vezet. Egészen pontosan nem csak az átadott ablak megszűnésekor, de az eltüntetésekor (\textit{hide}) is ki fog lépni a főciklus.

  \index{GtkWidget@\texttt{GtkWidget}!függvények!show@\texttt{show}}
  \item[\ref{gtkminimalc:windowshow}. sor] Ezek után nem érhet meglepetésként bennünket, hogy miért nincs szükség a \textit{C++} változat esetén az megjelenítő függvény meghívására, ezt az átadott ablakra a főciklust elindító \texttt{run} függvény megteszi, ami logikus is hiszen ha nincs egyetlen látható ablakunk sem, akkor nehéz olyan felhasználó interakció kezdeményezni -- legalábbis a felhasználói felületen keresztül --, ami a főciklus kilépését eredményezné.

 \index{Main@\texttt{Main}!függvények!run@\texttt{run}}
 \index{Gtk@\texttt{Gtk}!függvények!main@\texttt{main}}
 \item[\ref{gtkminimalc:gtkrun}. sor] A hívások a -- korábban már részletezett -- \textit{GTK main loop}ot indítják, azaz itt kezdődik meg az az eseményvezérelt szakasz, mely a választott nyelvtől függően a \texttt{gtk\_main\_quit}, a \texttt{Gtk::Main::quit}, vagy a \texttt{Gtk.main\_quit} meghívásáig tart. Ezekben a minimális példákban erre az egyetlen mód a futtatáskor megjelenő ablak bezárása, hiszen az imént említett függvényeket az ennek hatására kiváltódó \texttt{delete-event} szignálhoz rendeltük.  Miután a szignált ''kezelő'' függvény lefutott a \textit{GTK} főciklusa (\textit{main loop}) kilép, azaz a \texttt{run} függvény futása befejeződik, a program futtatása az azt követő soron folytatódhat.

 \item[\ref{gtkminimalc:return}. sor] Visszatérési értékünk mindhárom esetben 0, amit a rendelkezésre álló nyelvi módszerek legegyszerűbbikével érünk el, ezzel jelezvén a hívó félnek, hogy a futás rendben lezajlott.
\end{description}

\subsection{Fordítás és futtatás}
\index{fordítás}
\index{futtatás}

A korábbiakban már említett metodika mellet a mostani forráskódok\footnote{természetesen csak a \textit{C}, illetve a \textit{C++} változat fordítandó a \textit{Python} nem} fordítása a követlezőképp néz ki:

\lstcompiles
{gtk_minimal.c}{gtk_minimal}
{gtkmm_minimal.cc}{gtkmm_minimal}

A futtatás tekintetében próbálkozzunk a \texttt{./gtk\_minimal}, illetve a \texttt{./gtkmm\_minimal}, illetve a \texttt{python3 gtk\_minimal.py} parancsokkal abban a könyvtárban, ahol a forrásállományaink is találhatóak.

\subsection{Eredmény}

Nagy meglepetést nem várhatunk egy ilyen méretű applikációtól, viszont az azért kell tudnunk értékelni, hogy a legrosszabb esetben is alig másfél tucat kódsorból egy működő grafikus felhasználó felületet lehet létrehozni, melynek eredménye az alábbi képeken látható.

\begin{figure}[ht]
\begin{center}
\includegraphics[height=40mm]{images/screenshot_gtk_minimal.png}\hspace{0.05\textwidth}
\includegraphics[height=40mm]{images/screenshot_gtkmm_minimal.png}\hspace{0.05\textwidth}
\includegraphics[height=40mm]{images/screenshot_gtkpy_minimal.png}
\caption{Minimális mintapéldák képernyőképi}{\textit{C}, \textit{C++}, \textit{Python} változat}
\label{fig:screenshotminimap}
\end{center}
\end{figure}

\index{widget@\textit{widget}!szignálok!delete-event@\texttt{delete-event}}
\index{ablakkezelő}
\index{ablakkezelő!bezárás}
\index{ablakkezelő!minimalizálás}
\index{ablakkezelő!minimalizálás}
Tulajdonképpen csak egy puszta ablakot kapunk, bármilyen gomb, vagy egyéb elem nélkül. Minden díszítés -- az ablak fejléce, a címsor szövege, a minimalizáló, maximalizáló és a bezáró gombok -- egyaránt az ablakkezelőnek és nem a \textit{GTK}-nak köszönhetőek. Ez utóbbi gombhoz kötődik az egyetlen -- \textit{GTK} szempontjából is érdemleges\footnote{a minimalizálása és a maximalizálás az ablakkezelő hatáskörébe tartoznak} -- művelet, ami a már több ízben is említett \texttt{delete-event} szignált fogja kiváltani, ami a végső soron a program futásának befejezéséhez vezet.

\section{Tesztelés}

Elsőre talán azt gondolhatjuk, hogy a fenti ablakon igazán nincs mit tesztelni, egy feladat mégis akad, mégpedig az, hogy a futó applikációt, illetve annak egyetlen ablakát megtaláljuk, majd bezárjuk, ami mint látni fogjuk azért ez is jelent némi feladatot.

\subsection{Forráskód}

\index{Dogtail@\textit{Dogtail}!tree@\texttt{tree}}
\index{Dogtail@\textit{Dogtail}!procedural@\texttt{procedural}}
Az ablakok tesztelésére szolgáló programok forráskódjai (\ref{lst:dogtailminimal}. Kódrészlet) mindössze két sorban térnek el egymástól, ami a gyakorlatban nem jelent érdemi különbséget, ugyanakkor rámutat arra, hogy a \textit{Dogtail} használatával két API is rendelkezésünkre áll az applikációk teszteléséhez. Ezek a \textit{tree}, illetve a \texttt{procedural} API. Előbbi egy objektumorientált megközelítést alkalmazva teszi lehetővé, hogy a felhasználó felület egyes elemeit -- gombok, ablakok, menük -- elérjük, azokon műveleteket végezzünk, illetve a köztük fennálló összefüggéseket feltárjuk. Utóbbi az ablakokra, illetve azok elemeire -- melyeket nevükkel hivatkozhatunk -- ad fókuszt, illetve végez egyéb műveleteket, aminek révén egyszerűen vezérelhetjük a tesztelendő alkalmazást.

\lstdoublepysource
{sources/dogtail_minimal_procedural.py}
{sources/dogtail_minimal_tree.py}
{Minimálisan szükséges teszt \textit{tree}, illetve \textit{procedural} API használata mellett}
{lst:dogtailminimal}

\index{Python@\textit{Python}!unittest@\texttt{unittest}}
A következő kód -- a fejlesztési példától némiképpen eltérően -- nem szorítkozik a minimálisan szükséges kódsorok ismertetésére, ennél egy kicsit tovább megy. Ennek oka kettős. Egyrészről minimálisan alig néhány sorra van szükség, másrészről igyekszünk egy, a valós életben is használható kóddal szolgálni, aminek része egy tesztelési kertrendszer (framework), ebben az esetben a \textit{Python} \textit{unittest} modulja. Nem mellesleg a \textit{Dogtail} saját tesztjei is ezt a modult alkalmazzák, a következőkben ismertetettekhez nagyon hasonló, esetenként teljesen azonos módon. Ahogy a korábbiakban, úgy itt sem ismertetjük a nyelvi eszközökből adódó sajátosságokat, hacsak annak nincs kifejezett hatása a tesztelésre, így a \texttt{unittest} modul is csak olyan mértékben kerül ismertetésre, amennyire az a megértés szempontjából szükséges. Tesztelendő alkalmazásnak a \textit{GTK} demó programját választottuk, ami csaknem minden \textit{widget}re ad példát és része azon fejlesztői korábban már említett csomagnak, mely a \textit{C} nyelvű fordításhoz szükséges.

\begin{description}
 \item[\ref{dogtailminimal:importtree}. sor] A két különbözőséget adó sor egyike, ahol az \textit{Dogtail} előbbiekben említett \texttt{procedural}, illetve \texttt{tree} moduljait importáljuk. Az egyes példákban az importált modulnak megfelelő módszereket, illetve eszközöket alkalmazzuk.

 \index{Python@\textit{Python}!unittest@\texttt{unittest}!TestCase@\texttt{TestCase}}
 \item[\ref{dogtailminimal:testclass}. sor] A tesztelésre szolgáló osztályunk a \textit{Python} beépített \texttt{unittest} nevű moduljának \texttt{TestCase} osztályából származik, vagyis ezen modult használjuk a tesztek implementálására. Maga a modul természetesen nem feltétlenül szükséges a \textit{Dogtail} alapú teszteléshez, ugyanakkor számos olyan eszközt nyújt melynek a későbbiekben még hasznát látjuk.

 \index{Python@\textit{Python}!unittest@\texttt{unittest}!setUp@\texttt{setUp}}
 \index{Dogtail@\textit{Dogtail}!tree@\texttt{tree}!root.application@\texttt{root.application}}
 \index{Dogtail@\textit{Dogtail}!procedural@\texttt{procedural}!focus.application@\texttt{focus.application}}
 \index{Dogtail@\textit{Dogtail}!utils@\texttt{utils}!run@\texttt{run}}
 \item[\ref{dogtailminimal:setup}. sor] A \textit{unittest} modul \texttt{TestCase} osztályából származó osztályok \texttt{setUp}, illetve \texttt{tearDown} (\ref{dogtailminimal:teardown}. sor) nevű függvényei a tesztfuttatása során automatikusan meghívódnak minden egyes teszteset előtt, illetve után, lehetőséget adva a összes tesztesetre nézve közös előkészítő, illetve utómunkák elvégzésére. Ebben az esetben az előkészíts nem áll másból, mint hogy a \texttt{utils} modul megfelelő függvényének segítségével elindítjuk a \textit{GTK+} demó alkalmazását, amit az egyes tesztelési feladatok bemutatására használunk majd fel.

 \index{Python@\textit{Python}!os@\texttt{os}!kill@\texttt{kill}}
 \item[\ref{dogtailminimal:utilsrun}. sor] Itt történik a \textit{GTK+} demó alkalmazásának tényleges futtatása, ahol a visszatérési értékként kapott folyamatazonosítót (\textit{PID}) eltesszük későbbi használatra (\ref{dogtailminimal:kill}. sor).

 \index{Python@\textit{Python}!unittest@\texttt{unittest}!tearDown@\texttt{tearDown}}
 \item[\ref{dogtailminimal:teardown}. sor] Ahogy arról szó esett a \texttt{setUp} függvényhez hasonlóan egy speciális függvény, amit \textit{Python} \texttt{unittest} modulja automatikusan hív meg, minden egyes teszteset lefutását követően.

 \index{Python@\textit{Python}!signal@\texttt{signal}!SIGTERM@\texttt{SIGTERM}}
 \item[\ref{dogtailminimal:kill}. sor] A korábban elmentett (\ref{dogtailminimal:utilsrun}. sor) folyamatazonosítót felhasználva küldünk kilépésre (\textit{termination}) felszólító szignált a \textit{GTK+} demó programjának, mivel annak főablaka nem tartalmaz olyan elemet (pl: menüpont, gomb, \dots), melynek segítségével a kilépést el lehetne érni.

 \index{Python@\textit{Python}!time@\texttt{time}!sleep@\texttt{sleep}}
 \item[\ref{dogtailminimal:sleep}. sor] Levezetésként 5 másodperc várakozás következik minden tesztesetet követően elkerülendő az \textit{AT-SPI} túlterhelését.

 \item[\ref{dogtailminimal:testgtkdemo}. sor] Ez a függvény voltaképpen csak demonstrációs jelleggel kapott helyet ebben a példaprogramban, bemutatandó, hogy a nevükben \texttt{test} előtaggal rendelkező függvényeket a \textit{Python} \texttt{unittest} modulja tesztnek tekinti és ennek megfelelően futtatja őket.

 \index{Python@\textit{Python}!unittest@\texttt{unittest}!main@\texttt{main}}
 \item[\ref{dogtailminimal:unittestmain}. sor] A \texttt{unittest} modul \texttt{main} függvénye példányosítja a \texttt{TestCase} osztály leszármazottjait -- azaz a teszteset implementációkat tartalmazó objektumokat hoz létre --, majd futtatja az azokban lévő -- az előbbiekben említett nevezéktan szerinti -- tesztfüggvényeket. Bizonyos körülmények megfelelően -- például, hogy a függvények futása során keletkezett-e kivétel -- hibásnak, vagy sikeresnek tekinti e teszteket. A \texttt{unittest} modul számos funkció révén könnyíti meg a tesztelői munkát, melynek részleteiről a modul dokumentációjában olvashatunk, illetve a további részekben lesz szó.

\end{description}

\subsection{Futtatás}

\index{teszt szkript!futtatás}
\index{Dogtail@\textit{Dogtail}!procedural@\texttt{procedural}!GtkDemoTest@\texttt{GtkDemoTest}}
Az imént ismertetett tesztfuttatása rendkívül egyszerű eredményre vezet, lévén mindösszesen egy tesztesetet (\texttt{GtkDemoTest}) és azon belül is csak egyetlen tesztet (\texttt{testGtkDemo}) tartalmaz, mely reményeink szerint sikeresen fut majd le. A szkript futtatása még ebben az egyszerű esetben is többféleképp lehetséges, már ami a megadható paramétereket illeti. A paraméterek nélkül a szkript az összes tesztesetének összes tesztfüggvényét futtatja, ugyanakkor lehetőség van paraméterként egy teszteset, vagy akár egy azon belüli tesztfüggvény megadására is. Előbbi révén a megadott teszteset összes tesztfüggvénye futtatható, míg az utóbbi eset egy konkrét tesztfüggvény futtatására használható.

\medskip
\lstcommand{python dogtail_minimal_tree.py}\\
\lstcommand{python dogtail_minimal_tree.py GtkDemoTest}\\
\lstcommand{python dogtail_minimal_tree.py GtkDemoTest.testGtkDemo}
\medskip

A tesztesetek lefutásának mikéntjéről maga a teszt szolgáltat információt futtatáskor megjelenítve az összes futtatott teszteset számát, a futtatás idejét, a sikeresen, a sikertelenül, illetve a hibásan lefutott teszteket. Utóbbi két esetben a hiba okával, illetve a hozzájuk tartozó híváslistával (\textit{backtrace}) együtt, ami alapjául szolgálhat a hibakeresésnek.


\chapter{Szignálkezelés dióhéjban}
Ebben a részben a szignálok kezelésének elméleti kérdéseiről, illetve gyakorlatáról esik szó, amihez egy új példaprogramot veszünk górcső alá, ami forráskódját tekintve némiképp ugyan bonyolultabb a korábbiakban tárgyaltaktól, működésére nézve azonban nem sokban különbözik attól.

\section{Fogalmak}

Először is a korábban már megismert alapfogalmakat vesszük újra elő, most azonban általános ismertetésük helyett a témánkhoz konkrétan kapcsolódó specifikumaikat vázoljuk fel.

\paragraph{Main Loop}
\index{main loop}

Ahogy arról az előző részekben már szó esett -- más felületprogramozási nyelvekhez teljesen hasonlóan -- a \textit{GTK} is eseményvezérelt (\textit{event-driven}) módon működik. Ez annyit tesz, hogy felhasználói interakciók bekövetkeztéig -- figyelmen kívül hagyva az ütemezett eseményeket és néhány, a későbbi részekben részletezendő funkciót --, a \textit{GTK} a saját főciklusában (\textit{main loop}) várakozik, lényegében tehát a szoftver futását maguk a bekövetkező események vezérlik, hiszen ezek híján a várakozás sosem ér véget. Felhasználói interakció lehet például az egér megmozdítása, vagy egy kattintás, esetleg egy billentyű lenyomása, vagy éppen felengedése.

Ahhoz, hog az alkalmazás az eseményvezérelt szakasza megkezdődhessen be kell lépni a \textit{GTK} főciklusába. Erre, az előző részben taglalt minimalista alkalmazások forráskódjából már ismerős \texttt{gtk\_main()}, \textit{gtkmm} esetén a \texttt{Gtk::Main::run()}, illetve a \textit{Python} változat esetén a \texttt{Gtk.main()} függvény szolgál. Ha az imént említett események közül bármelyik bekövetkezik, az addig ``alvó'' főciklus úgymond ``felébred'', azonosítja az eseményhez tartozó felületi elemet, majd a bekövetkezett eseményt továbbítja (\textit{propagate}) az azonosított \textit{widget}, vagy \textit{widget}ek felé. A vezérlés ezt követően a főciklusba tér vissza, ahol folytatódik a várakozás a következő eseményre. 

\paragraph{Signal}
\index{szignál}

A szignál voltaképpen egy névvel azonosított, a bekövetkezett esemény továbbítására szolgáló üzenet, amit az osztály definiál, hogy példányai értesíthessék az események bekövetkeztéről az az iránt érdeklődőket. Ilyen üzenetből számos létezik, hisz az egyes \textit{widget}típusokon különböző események lehetnek értelmezettek. Gomb esetén a rá történő kattintás, egy legördülő menünél az egér menüelem fölé történő mozgatása, míg egy \textit{widget}nél például annak átméretezése. Minden ilyen esemény rendelkezik saját névvel, melynek révén hivatkozni tudunk rá (pl: \texttt{button-pressed}, \texttt{enter-notify-event}, \texttt{size-request}). A \textit{szignál}ok öröklődnek, azaz egy specifikus \textit{widget}, mint amilyen mondjuk egy \textit{RadioButton}, vagy egy \textit{CheckButton}, minden olyan szignállal rendelkezik, amivel őse a \textit{Button}, vagy akár annak az őse az a \textit{Widget} típus rendelkezett.

A \textit{szignál}ok egyrészről arra szolgálnak, hogy a \textit{GTK} rendszerén belül az egyes \textit{widget}ek egymással kommunikálhassanak. Ha például egy gombot lenyomunk, akkor azt (illetve annak részeit) újra kell rajzolni, ha egy menüelemet kiválasztunk, azt át kell színezni, illetve az esetleges almenüpontokat ki kell rajzolni, míg átméretezésnél az egyes \textit{widget}ek helyigényét újra ki kell számolni. Másfelől ha a program írói valamely esemény bekövetkezéséről értesülni szeretnének, megadhatnánk eseménykezelő függvényeket, melyek ezen esetekben meghívódnak.

\paragraph{Callback}

Ezen eseménykezelő függvények elnevezése a \textit{GTK} terminológiában \textit{callback}. Az egyes eseményekhez tartozó kezelőfüggvények prototípusai a \textit{szignál} fajtájától függenek. A \textit{C} nyelvű változat esetén első paraméterük jellemzően az a \textit{Widget} -- pontosabban szólva \textit{Object}, hiszen a \textit{szignál}kezelés ezen a szinten került implementálásra a \textit{Glib}-ben -- melyen az esemény kiváltódott. Ezt a paramétert követik a \textit{szignál}hoz kapcsolódó egyéb jellemzők, az utolsó pedig a szignál bekötésekor megadott, úgynevezett \textit{user data}, amiről a példaprogram kapcsán részletesebben szólunk. Elöljáróban csak annyit, hogy ez egy meglehetősen kényelmetlen és gyakorta nehézkesen használható megoldás, melyre a \textit{C++}, illetve \textit{Python} nyelvű változatok kínálnak kényelmes alternatívát.

\section{Szignálkezelés}

Az előző szám módszertanától eltérve az alábbiak szerint elemezzük a kódokat:

\begin{itemize}
 \item külön-külön vesszük számba ez egyes nyelvi változatok sajátosságait
 \item először a \textit{C}, illetve a \textit{C++} nyelvű verziónak fogunk neki, ezt követően bemutatjuk mennyiben más a helyzet, \textit{Python} nyelvű változatokban
 \item a kódot nem sorfolytonosan, hanem a futás logikája szerint követjük, lévén egy kicsit is bonyolultabb esetben -- mint amilyennek az alábbi példa is mondható -- már ez a logikusabb
\end{itemize}

\subsection{\textit{C}, illetve \textit{C++} nyelvű változat}

\lstdoublecsource
{sources/gtk_signal.c}
{sources/gtkmm_signal.cc}
{Szignálok kezelése \textit{C}, illetve \textit{C++} nyelven}
{lst:gtksignal}

\subsubsection{Általánosságok}

\begin{description}
 \index{GLib@\texttt{GLib}!függvények!print@\texttt{print}}
 \item[\ref{gtksignalc:includebegin} - \ref{gtksignalcc:includeend} sor] Az fejlécállományok beszerkesztésének sajátosságairól már esett szó, így az egyedüli specifikum itt a \textit{C++} változat által beszerkesztett \texttt{iostream} fejlécfájl, amire csupán a képernyőre történő íráshoz lesz szükség. A \textit{GTK+} esetén -- mivel a \texttt{gtk.h} minden szükséges fejléc állományt maga felsorol -- használni tudjuk a \textit{Glib} erre a célra használatos függvénylét (\texttt{g\_print}).

 \item[\ref{gtksignalc:main} - \ref{gtksignalc:end} sor] A \texttt{main} függvényben alkalmazottak gyakorlatilag teljesen azonosak a korábbi minimális példánál ismertetettekkel, így itt erről leginkább csak annyit érdemes megjegyezni, hogy a \textit{C++} változat -- élve a nyelv adta lehetőségekkel -- egy saját osztály segítségével zárja egységbe egy gombbal kiegészített ablakunkat, míg a \textit{C} változatban egy -- a saját ablak létrehozására szolgáló -- függvény segítségével igyekeztünk ezt módszert valamelyest követni. Természetesen a \textit{GTK+} esetén is van mód származtatásra, de nem oly kézenfekvő módon, mint amikor a \textit{C++} nyelvet használjuk, ennél fogva ennek ismertetése még várat magára.
\end{description}

\subsubsection{Szignálok a \textit{GTK+} nyelvű kódban}

\begin{description}
 \index{GtkButton@\texttt{GtkButton}!függvények!new\_with\_label@\texttt{new\_with\_label}}
 \index{GtkWindow@\texttt{GtkWindow}!függvények!new@\texttt{new}}
 \index{típuskényszerítés}
 \item[\ref{gtksignalc:widgetcreatebegin} - \ref{gtksignalc:widgetcreateend} sor] A létrehozott ablak, illetve gomb tárolására szolgáló változók típusa \texttt{GtkWidget}, a specifikusabb \texttt{GtkWindow}, illetve \texttt{GtkButton} helyett. Ennek magyarázata, hogy minden olyan függvény a \textit{GTK+}-ban, melynek segítségével egy új \textit{widget}et hozhatunk létre -- gyakorlatilag a \texttt{\_new} végű metódusok -- egy \texttt{GtkWidget} típusú objektumra mutatóval tér vissza. Ennek több oka is van. Egyrészről kényelmi, hogy elkerülhessük a folytonos típuskényszerítéseket, hisz számos esetben olyan függvényeket használunk, melyek amúgy is \texttt{GtkWidget}eket kezelnek, tehát ilyen típusra mutatót vesznek át első paraméterként. Másrészről ha egy specifikus -- mondjuk \texttt{GtkButton}-t kezelő -- függvényt akarunk hívni, akkor vagy fordítási -- vagy ami inkább javasolt -- futás idejű típuskényszerítés alkalmazandó (pl: \texttt{GTK\_BUTTON}, \texttt{GTK\_WINDOW}), aminek megvan az a komoly előnye, hogy ha sikerült a mutatónkat, vagy a mutatott objektumot valamilyen módon korrumpálni, akkor arra viszonylag hamar fény tud derülni.
\end{description}

\paragraph{Szignálkezelő függvények felkötése}

\begin{description}
 \index{GtkWidget@\texttt{GtkWidget}!szignálok!delete-event@\texttt{delete-event}}
 \index{GtkButton@\texttt{GtkButton}!szignálok!clicked@\texttt{clicked}}
 \index{GObject@\texttt{GObject}!függvények!signal\_connect\_data@\texttt{signal\_connect\_data}}
 \index{GObject@\texttt{GObject}!makrók!signal\_connect@\texttt{signal\_connect}}
 \index{GLib@\texttt{GLib}!makrók!OBJECT@\texttt{OBJECT}}
 \index{öröklődés}
 \index{szignál}
 \index{szignál!eseménykezelő felkötése}
 \item[\ref{gtksignalc:signaldeleteevent}. sor] Az első szignálbekötés. Viszonylagos egyszerűsége ellenére számos apróságra érdemes figyelmet fordítani. Az első maga a \texttt{g\_signal\_connect} kulcsszó, ami függvénynek tűnhet, pedig ugyanúgy, mint a \texttt{g\_signal\_connect\_swapped}, makró, amik a \texttt{g\_sinal\_connect\_data} függvényt burkolják. A soron következő érdekesség a \texttt{G\_OBJECT} makró, ami futás idejű típusellenőrzést hajt végre a neki megadott paraméteren, majd egy \texttt{GObject} típusra mutatóval tér vissza. A megrögzött \textit{C++} programozók joggal kérdezhetik, mi szükség erre, hisz egyfelől majd elvégzi a típusellenőrzést a fordító, meg hát a \texttt{GtkWindow} típus úgy is leszármazottja a \textit{GObject} ''osztálynak``. Ez így is lenne, na de ez itt \textit{C}, tehát ős-, illetve származtatott osztályokról csak logikai értelemben lehet szó, a típusellenőrzés tehát nem végezhető, sőt minden esetben a hívott függvénynek megfelelő típuskényszerítő makrót célszerű alkalmazni.

 A második paraméter a szignál neve, amivel azt adjuk meg, hogy az előző paraméterként megadott \textit{object} melyik szignáljára is szeretnénk kezelő függvényt (\textit{callback}) kötni. A harmadik paraméter azon függvény címe, aminek meghívódását ki szeretnénk váltani az esemény bekövetkezésekor. A függvénynevet itt is egy makró segítségével adjuk át, ami az előzőekhez hasonlóan \textit{C} nyelvi hiányosságokra vezethető vissza. Mivel a meghívandó \textit{callback}ek prototípusai igen sokfélék lehetnek (ami magából a példából is látszik valamelyest és ezek mind külön típusnak minősülnek a \textit{C} nyelvben ezért ahányféle \textit{callback} variáció létezik, annyiféle \textit{g\_signal\_connect} függvényre lenne szükség. Könnyen belátható, hogy a jogos lustaság más irányba vitte a \textit{GTK+} fejlesztőit. A \texttt{G\_CALLBACK} tulajdonképpen egy fordítási idejű típuskényszerítés egy általános függvénytípusra, amivel ugyan megoldottuk, hogy csak egyetlen \texttt{g\_signal\_connect\_data} függvényre legyen szükség, de elvesztettünk minden nemű típusbiztosságot. Ha például egy az adott szignálnak nem megfelelő típusú függvényt adunk meg paraméterként, amit a példabeli függvénynevek felcserélésével könnyen megtehetünk, csúnya meglepetésekben lesz részünk, de csak futásidőben. Nem hagyhatjuk továbbá figyelmen kívül a \textit{C} nyelv azon sajátosságát sem, hogy az átadott függvényparaméterek átvétele nemkötelező, azaz ha kevesebb paraméterrel definiálunk egy függvény, mit amennyivel hívni szeretnénk voltaképpen nem követünk el bűnt, ez viszont tovább bonyolítja a helyzetet.

 \index{GLib@\texttt{GLib}!makrók!NULL@\texttt{NULL}}
 Az utolsó paraméter az úgynevezett \textit{user data}, ami arra szolgál, hogy az eseménykezelő függvényünknek olyan adatokat adjunk át, amik az esemény megtörténtéből nem következnek. Ilyenek lehetnek például más \textit{widget}ek címei, ahogy azt látni is fogjuk. Ez esetben az átadott paraméter \texttt{NULL}, ami szintén egy makró ami egy jól nevelt \texttt{((void*) 0)} kifejezésre fejtődik ki \textit{C} kód esetén. Zárszóként ehhez a sorhoz csak annyit, hogy a \texttt{delete\_event} eseményt az ablakkezelő váltja ki, akkor, amikor az ablakot valamilyen módon (billentyűzet, menü, egér) bezárjuk.
\end{description}

\paragraph{A \texttt{delete-event} szignál blokkolása}
\index{szignál!blokkolása}
\index{GtkWidget@\texttt{GtkWidget}!szignálok!delete-event@\texttt{delete-event}}

\begin{description}
 \index{GLib@\texttt{GLib}!típusok!gboolean@\texttt{gboolean}}
 \index{szignál!alapértelmezett eseménykezelő függvény}
 \item[\ref{gtksignalc:callbackdeleteevent}. sor] Ez a \texttt{delete\_event} szignálkezelő függvénye, aminek -- néhány más szignálkezelő függvényhez hasonlóan -- egy \texttt{gboolean} értékkel kell visszatérnie, ami azt határozza meg, hogy az általunk a szignálkezelőben végrehajtottak után a \textit{GTK} lefuttassa-e saját szignálkezelő rutinját, vagy sem. Jelentése voltaképpen tehát az, hogy a saját magunk a szignált mindenre kiterjedően kezeltük-e. Ennek megfelelően ha a visszatérési érték 0 -- azaz logikai hamis --, akkor végrehajtódik a \textit{GTK+} adott szignálhoz kapcsolódó alapértelmezett kezelő függvénye, ellenkező esetben értelemszerűen arra utasítjuk a \textit{GTK}-t, hogy a szignál további feldolgozásától tekintsen el. Itt érdemes felhívni a figyelmet arra, hogy mivel a \textit{C} nyelvben -- a \textit{C++}-al ellentétben --, nincs \texttt{bool} típus annak analógiájára definiálták a \texttt{gboolean} típust (ami tulajdonképpen egy \texttt{int}) és a két megfelelő logikai értéket makróként (\texttt{TRUE}, \texttt{FALSE}).

 \index{GtkWidget@\texttt{GtkWidget}!függvények!destroy@\texttt{destroy}}
 Ebben a konkrét esetben (\texttt{delete\_event} szignál) az alapértelmezett szignálkezelő a \texttt{gtk\_widget\_destroy} függvény, vagyis ha nem kötünk fel saját szignálkezelő függvényt, vagy logikai hamis értékkel térünk vissza a kezelő függvényből, akkor a \texttt{window} objektum megsemmisül, az ablak bezárul. Logikai igaz érték visszaadásával elérhető, hogy hiába próbáljuk akár a jobb felső sarok \textit{x} gombjának megnyomásával, akár valamilyen billentyűkombináció révén bezárni az ablakot ez a próbálkozás sikertelen lesz, ellenben minden ilyen próbálkozás egy újabb sor kiírást eredményezi. 
\end{description}

\paragraph{Az ablak bezárása}

\begin{description}
 \index{GtkWidget@\texttt{GtkWidget}!szignálok!destroy@\texttt{destroy}}
 \index{GtkWidget@\texttt{GtkWidget}!függvények!destroy@\texttt{destroy}}
 \item[\ref{gtksignalc:signaldestroy}. sor] Az előzőhöz teljesen hasonló módon itt a \texttt{destroy} szignálra kötünk be eseménykezelőt, ami a \textit{widget} életciklusának végén váltódik ki. Egy konténerben lévő \textit{widget} esetén ez a konténerből való eltávolítás következménye -- már ha valaki nem tart külön referenciát az eltávolított \textit{widget}re --, legfelső szintű \textit{widget}ek (\textit{toplvel}) esetén -- mint amilyenek az ablakok is -- ez jellemzően a \texttt{gtk\_widget\_destroy} függvény meghívásának folyománya, lévén az ilyen \textit{widget}ek automatikusan nem semmisülnek meg, erről nekünk explicit módon kell gondoskodnunk (\ref{gtksignalc:connectdestroy}. sor).

  A \textit{destroy} szignál kezelése általánosságban nézve ritka, jelen esetben is csak az a szerepe, hogy a program valamilyen módon ki tudjon lépni. Az ablak bezárása alapértelmezetten ugyan ezt eredményezné, de a \texttt{delete-event} szignálra kötött kezelőfüggvényben nem hogy ezt nem tesszük meg, de még az ablak bezáródásást is meggátoljuk. Mikor a \texttt{destroy} szignálra kötött kezelőfüggvény meghívódik, ablakunk épp megszűnőfélben van, ugyanakkor ha ez az eset áll is fenn a programunk futása annak ellenére sem érne véget, hogy az ablakunk bezáródik, hiszen a \textit{main loop}ból nem lépnénk ki. Ezen helyzet elkerülésére eseménykezelőként a \textit{main loop}ból való kilépésre szolgáló \textit{gtk\_main\_quit} függvényt kötjük fel.

 \index{Gtk@\texttt{Gtk}!függvények!main\_quit@\texttt{main\_quit}}
 Érdemes megjegyezni, hogy bár a \texttt{gtk\_main\_quit} függvény definíciója (\texttt{void (*) ()}) nem felel meg tökéletesen a \textit{destroy} szignál által elvártaknak (\texttt{void (*) (GtkWidget *, gpointer)}) ez voltaképpen nem jelent problémát, hiszen a típusok különbözőségét a \texttt{G\_CALLBACK} makró által alkalmazott típuskényszerítés elrejti a fordító elől, futásidőben pedig a \texttt{gtk\_main\_quit} egész egyszerűen nem veszi át a szignálkezelőt meghívó kódtól a két függvényparamétert.

 \index{GtkWidget@\texttt{GtkWidget}!függvények!destroy@\texttt{destroy}}
 \index{GObject@\texttt{GObject}!makrók!signal\_connect\_swapped@\texttt{signal\_connect\_swapped}}
 \item[\ref{gtksignalc:connectdestroy}. sor] A \ref{gtksignalc:connectclicked}. sortól csak a meghívandó eseménykezelő függvényben, illetve az annak átadandó paraméterekben sorrendjében tér el. Ahogy az a függvény nevéből (\texttt{g\_signal\_connect\_swapped}) következik, arról van szó, hogy a gomb lenyomásakor meghívandó \textit{callback} -- jelen esetben a \texttt{gtk\_widget\_destroy} -- paramétereiben a \textit{user\_data}, illetve az az \textit{object}, amin az esemény kiváltódik, felcserélésre kerül. Kicsit konkrétabban fogalmazva a \textit{user\_data} lesz a \textit{callback} első paramétere és a gomb a második. Mivel itt a \textit{callback} a \texttt{gtk\_widget\_destroy} függvény, ami paraméterként mondjuk úgy, a törlendő \textit{widget}et várja, a \textit{user\_data} pedig az ablakunk, nem nehéz kitalálni, hogy a gombra való kattintás eredményeként az ablak meg fog szűnni, de csak azután, hogy a ``Helló Window!'' üzenet megjelent a konzolban.
\end{description}

\paragraph{Gomb lenyomásának kezelése}

\begin{description}
 \index{GtkButton@\texttt{GtkButton}!szignálok!clicked@\texttt{clicked}}
 \index{típuskényszerítés}
 \item[\ref{gtksignalc:connectclicked}. sor] Eseménykezelő függvény bekötése a gomb \texttt{clicked} szignáljára, ami a gomb lenyomásakor hívódik meg, aminek egyetlen különlegessége, hogy itt a szignálkezelő definíciója pontosan megfelel a szignál által elvártaknak, ugyanakkor a \texttt{G\_CALLBACK} makró mégis szükséges, mivel a \texttt{g\_signal\_connect} azt a típust várja, amire az \texttt{on\_button\_clicked} függvényt a makró kényszeríti.

 \index{szignál!eseménykezelő függvények sorrendje}
 \item[\ref{gtksignalc:callbackbuttonclicked}. sor] A fenti állítás -- miszerint az ablak csak a kiírást követően szűnik meg -- csak azért igaz, mert a \textit{on\_button\_clicked} függvény, mint eseménykezelő előbb kerül felkötésre, mint a \texttt{gtk\_widget\_destroy}, valamint azért, mert az eseménykezelők alapvetően a felkötés sorrendjében kerülnek meghívásra. Fordított esetben előbb hívódna meg a \textit{destroy} az ablakra, ami -- sok egyéb mellett -- leköti az eseménykezelő függvényeket, így a kiírást nem is látnánk.
\end{description}

\paragraph{Egyebek}

\begin{description}
 \item[\ref{gtksignalc:addbutton}. sor] A nyomógomb hozzáadása az ablakhoz.

 \item[\ref{gtksignalc:widgetshowbegin} - \ref{gtksignalc:widgetshowend} sor] A létrehozott \textit{widget}ek megjelenítése.

 \item[\ref{gtksignalc:windowreturn}. sor] Belépés az eseményvezérelt szakaszba.
\end{description}

Fentiek ismeretében nagy biztonsággal jósolhatjuk meg példaprogramunk működését. Az elindított alkalmazás egy ablakot jelenít meg, melyben egy \textit{Helló Window!} feliratú gomb lesz. Az ablak bezárásával hiába próbálkozunk egér, vagy billentyűzet segítségével, ezen kísérletek eredmény csupán egy-egy ''\textit{delete event occurred}`` sor a konzolban. Ha azonban le találnák nyomni gombunkat az ablak hirtelen eltűnik a konzolban egy ''\textit{Helló Window!}`` felirat jelenik meg és a program kilép. Lássuk, hogy érhetünk ehhez teljesen hasonló funkcionalitást \textit{C++}-ban.

\subsubsection{Szignálok a \textit{gtkmm} nyelvű kódban}

\paragraph{Szignálkezelő függvények felkötése}

\begin{description}
 \index{öröklődés}
 \item[\ref{gtksignalc:main} - \ref{gtksignalc:end} sor] Ahogy azt az általánosságokat taglaló részben említettük \texttt{Gtk::Window} helyett \texttt{MyWindow} típust használunk főablakunk létrehozásához. Mivel azonban a \texttt{MyWindow} publikusan származik a \texttt{Gtk::Window} típusból ez a \textit{gtkmm} számára nem jelent különbséget. A \textit{C} változathoz képest a származtatás itt nem csupán ''logikai'', vagyis minden a \textit{C++}-an megszokott előny könnyedén realizálható. Erre példa, hogy a származtatás miatt nincs szükség semmilyen típuskényszerítésre mikor a \texttt{Gtk::Main::run} függvényt hívjuk, ami pedig egy \texttt{Gtk::Window} referenciát vesz át paraméterként.

 \item[\ref{gtksignalcc:mywindowctorbegin} - \ref{gtksignalcc:mywindowctorend} sor] Saját osztályunk konstruktorában megtehetjük mindazokat a lépéseket, melyeket a \textit{C} nyelvű változat esetén a \texttt{my\_window\_new} függvényben implementáltunk. Úgy is mint a szignálok felkötése, a gomb hozzáadása az ablakhoz, a \textit{widget}ek megjelenítése. Az egységbezárás ezen előnyén túl a származtatásból fakadó örömöket is élvezhetjük, ugyanakkor persze az ebből fakadó kötelességeknek is eleget kell tenni. Ez esetben ez a konstruktor meghívását jelenti, ami rejtett módon megy végbe. Az ősosztály konstruktorának explicit hívásának hiányában a \texttt{Gtk::Window} azon konstruktora fut le, ami paraméterek nélkül is hívható. Másrészről viszont az adattagként tárolt \textit{GtkButton}t (\texttt{button}) is inicializálnunk kell. Itt is lehetne közvetve, implicit módon hívni a paraméter nélküli konstruktort, azonban kézenfekvőbb azt a változatot használni, amivel egyszerre a gomb feliratát (\textit{label}) is megadhatjuk, így egy hívás a későbbiekben megspórolható.

 \index{szignál!eseménykezelő felkötése}
 \index{Gtk@\texttt{Gtk}!tagfüggvények!main\_quit@\texttt{main\_quit}}
 \index{SignalProxyBase@\texttt{SignalProxyBase}!tagfüggvények!connect@\texttt{connect}}
 \index{libsigc++@\texttt{libsigc++}!típusok!slot@\texttt{slot}}
 Külön szót érdemelnek a szignálok bekötései. Különösebb programozó géniusz nem kell, hogy felfedezzük a szignálok eléréséhez egy \textit{signal\_}\textit{szignálnév} szerkezetű hívását használjuk fel. Az ilyen hívások egy, a \texttt{Glib::SignalProxyBase} osztályból származó objektumot adnak vissza, amik \texttt{connect} nevű metódusai valósítják meg azt, amit a \textit{GTK+} esetén a \texttt{g\_signal\_connect} makró tett meg, vagyis egy adott \textit{widget}, adott \textit{szignál}jára eseménykezelő felkötését. Előnye ennek a módszernek, hogy típusbiztos, azaz a \texttt{connect} paraméterként csak olyan függvényt (\textit{slot}) fogad el, melynek típusa megfelel az adott szignálnál leírtakkal. További előny, hogy a \textit{slot}okhoz nem csupán egy \textit{user data} csatolható, hanem tetszés szerinti számú, s ezek típusa is ellenőrzésre kerül fordításkor. Amennyiben azonban sikerül csupán egy apróságot is elírnunk a szignál bekötésénél, vagy a \textit{slot} típusának megadásánál -- a \textit{sablon}okkal (\textit{template}) történő megvalósításnak hála --, akkor jellemzően több oldalas, nehezen kibogarászható hibaüzenettel találhatjuk szemben magunkat.

 \index{libsigc++@\texttt{libsigc++}!függvények!ptr\_fun@\texttt{ptr\_fun}}
 \item[\ref{gtksignalcc:sigcptrfun} - \ref{gtksignalcc:sigcmemfun} sor] Lássuk akkor miként is érhető el ugyanaz \textit{gtkmm} esetén, mint ami korábban \textit{GTK+} használatával. Első pillantásra is szembeszökő, hogy mindkét sorban találunk olyan hívást, amik nem a \texttt{Gtk} névtérben definiáltak. Ennek az az oka, hogy a \textit{gtkmm} a szignálkezelést egy külső -- \textit{libsigc++} nevű -- függvénykönyvtárral valósítja meg. A két eseménykezelő felkötése közötti különbséget az eseménykezelő függvények típusa adja lássuk ezt részletesebben.

 \index{libsigc++@\texttt{libsigc++}!függvények!bind@\texttt{bind}}
 \item[\ref{gtksignalcc:sigcptrfun}. sor] Ha a megadni kívánt függvény nem kötődik objektumhoz -- legyen ez egy osztály statikus tagfüggvénye, vagy akár egy tisztán \textit{C} nyelvű kódból származó függvény -- \textit{slot} létrehozásához a \textit{sigc::ptr\_fun} alkalmazandó. Ebben a konkrét esetben a \textit{slot} létrehozásán túl, paramétereket is hozzákapcsolunk a \textit{clicked} esemény bekövetkeztekor meghívandó függvényhez. Ennek eszköze a \textit{sigc::bind}, melynek első paramétere egy \textit{slot}, a továbbiak pedig a csatolandó paraméterek. Itt csupán egy ilyen van, a gomb lenyomásának hatására kiírandó üzenet szövege. Ez persze kissé kényszeredett, hiszen a paraméter értéke soha nem változik, így ennek igazi hasznát ezen a példa alapján még nehéz belátni.

 \index{GtkButton@\texttt{GtkButton}!szignálok!clicked@\texttt{clicked}}
 \item[\ref{gtksignalcc:slotbuttonclicked}. sor] Eseménykezelő függvényünk a lehető legegyszerűbb, csupán azt szemlélteti miként is kell az átadott paramétereket használni. Ez esetben annak értékét a standard kimenetre kiírni. Működését és funkcióját tekintve a \textit{C}-s változat azonos nevű függvényével analóg.

 \index{libsigc++@\texttt{libsigc++}!függvények!mem\_fun@\texttt{mem\_fun}}
 \item[\ref{gtksignalcc:sigcmemfun}. sor] Ha a megadni kívánt eseménykezelő egy osztály tagfüggvénye, akkor a \textit{sigc::mem\_fun} használható arra, hogy \textit{slot}ot hozzunk létre az osztály egy példányából, illetve az osztály tagfüggvényéből, ebben a sorrendben átadva őket a függvénynek, utóbbit a teljes névtérlistával együtt. Természetesen az imént említett \texttt{sigc::bind}, az előbbiekhez hasonlóan módon itt is alkalmazható. 

 \index{GtkWidget@\texttt{GtkWidget}!függvények!destroy@\texttt{destroy}}
 \index{GtkWidget@\texttt{GtkWidget}!függvények!hide@\texttt{hide}}
 \index{GtkWidget@\texttt{GtkWidget}!szignálok!delete-event@\texttt{delete-event}}
 Ez a hívás épp ugyanazt a célt szolgálja, mint a \textit{GTK+} változat azonos sorszámú sora, azaz hogy az alkalmazásunkból annak ellenére is ki lehessen lépni, hogy az ablak bezáró gombjának hatására a itt sem történik semmi egyéb, mint kiíródik a ''\textit{delete event occurred}`` szöveg a standard kimenetre. Míg legutóbb a \texttt{gtk\_widget\_destroy} függvényt kötöttük fel eseménykezelőként, itt a \texttt{Gtk::Widget} osztály \texttt{hide} függvényét használjuk, aminek hatására a \textit{GTK} főciklusa kilép, mivel annak futtatásakor (\ref{gtksignalcc:gtkmainrun}. sor) megadtuk ablakunkat paraméterként. Átgondolva a működést jogosnak tekinthető, hiszen látható főablak hiányában a további működésnek nem sok értelme van, ugyanakkor a \textit{C} változat \texttt{gtk\_widget\_destroy} függvénye helyett a \textit{C++} változatban a \texttt{delete} hívással lehetne úgymond jelezni, hogy az ablakunkra nincs tovább szükség, viszont ez nem célszerű, hiszen az a \texttt{main} függvényben egy lokális változó.
\end{description}

\paragraph{A \texttt{delete-event} szignál blokkolása}
\index{szignál!blokkolása}
\index{GtkWidget@\texttt{GtkWidget}!szignálok!delete-event@\texttt{delete-event}}

\begin{description}
 \index{szignál!alapértelmezett eseménykezelő függvény}
 \item[\ref{gtksignalc:methoddeleteevent}. sor] A \textit{GTK+} változathoz képesti komoly különbség, hogy itt a \texttt{delete-event} szignál blokkolása nem egy felkötött eseménykezelőn keresztül valósul meg -- ezért is nincs a kódban olyan sor, ami erre a szignálra vonatkozna --, hanem az alapértelmezett eseménykezelő kerül felülírásra. A működés megértésének kulcsa a \texttt{virtual} kulcsszóban rejlik. Minden szignálhoz tartozik ugyanis egy -- az adott \textit{widget} által implementált -- alapértelmezett eseménykezelő függvény, ami alkalmasint felülbírálható (\textit{ovverride}). Ha ezt megtesszük, azzal a szignál kezelésének teljes folyamatát mi irányítjuk, ami mellett komoly érvek szólhatnak, de nem árt körültekintőnek lenni. Ám ebben az esetben a cél pont annak a demonstrálása, hogy a \textit{gtkmm} szabad kezet ad abban, hogy egy származtatott \textit{widget} miként kívánja kezelni az ősosztály eseményeit. A visszatérési érték szerepe ugyanaz, így a működés is azonos az előző -- \textit{GTK+} nyelvű -- példáéval.
\end{description}

A szignálkezelésről összegzésképpen annyit, hogy alapvetően két lehetőség kínálkozik arra, hogy az egyes \textit{widget}ek eseményei kezeljük:

\begin{itemize}
 \item \textit{Callback}eket kapcsolni azon \textit{widget}ek azon eseményihez, melyek számunkra érdekesek és ezekben megtenni a megfelelő lépéseket

 \item Felülbírálni a \textit{widget} saját eseménykezelőjét az öröklődés mechanizmusai útján. Erre mindkét változat esetén van lehetőség, ám a \textit{GTK+} megoldása kissé körülményes és nehezebben megérthető, így annak ismertetése valmely későbbi részre marad. A \textit{C++} nyelvi eszközeit kihasználva a \textit{gtkmm} viszont ezt oly könnyedén oldja meg, hogy kár lett volna kihagyni a bemutatást annak ellenére is, hogy a módszerre ritkán van szükség, hiszen többnyire arról van szó, hogy a különböző \textit{widget}példányok azonos szignáljainak kiváltódásakor más-más irányba szeretnénk terelni a program futását. A felülbírálás révén viszont arra nyílik lehetőség, hogy a szignál kezelésének módját változtassuk meg. Ha nem kívánunk egyebet tenni, mint ami amúgy is történne, hívjuk meg a felülbírált függvény szülőosztálybeli változatát. Ha azonban ez előtt, vagy után még valami másra is szükségünk van, megtehetjük, hogy csak a függvény közepéről hívjuk a szülő metódusát, vagy akár el is hagyhatjuk az ha tudjuk mit és főként hogyan szeretnénk kezelni.
\end{itemize}

\subsection{\textit{Python} nyelvű változatok}

\lstdoublepysource
{sources/gtk_signal.py}
{sources/gtk_signal_oo.py}
{Szignálok kezelése \textit{Python} nyelven}
{lst:gtksignalpy}


Elöljáróban annyit érdemes megjegyezni a \textit{Python} változat kapcsán, hogy az itt alkalmazott megoldások a \textit{C}, illetve \textit{C++} változatból már ismertek, logikájuk hol a \textit{GTK+}, hol pedig a \textit{gtkmm} megoldásaira emlékeztetnek -- függően természetesen attól is, hogy kihasználjuk-e a \textit{Python} nyelv adta objektum-orientáltság lehetőségeit --, ötvözve azok előnyeit. Kihasználva a korábbiakban leírtakat, itt már csak a kifejezett nyelv specifikus részeket ismertetjük, a csupán szintaktikai elemekben eltérő részletekre nem térünk ki.

\paragraph{Általánosságok}

\begin{description}
 \item[\ref{gtksignalpy:include}. sor] A \texttt{Gtk} szimbólumok importálása, csakúgy, mint a korábbi, minimális példában.

 \item[\ref{gtksignalpy:main} - \ref{gtksignalpy:end} sor] Ez a rész gyakorlatilag azonos a korábbi, minimális példánál ismertetettekkel, azzal a különbséggel, hogy a \texttt{delete-event} szignál helyet az ablakunk \texttt{destroy} szignáljára kötjük rá a programból való kilépéshez vezető \texttt{main\_quit} függvényt, melynek magyarázat épp az, mint a korábban \textit{C} változatnál, erről azonban még esik szó.

\end{description}

\paragraph{A \texttt{delete-event} szignál blokkolása}
\index{szignál!blokkolása}
\index{GtkWidget@\texttt{GtkWidget}!szignálok!delete-event@\texttt{delete-event}}

\begin{description}
 \item[\ref{gtksignalpy:signaldeleteevent}. sor] A korábbiakhoz hasonlóan a \texttt{delete-event} szignál elnyomására -- függően attól, hogy objektum-orientált megközelítést alkalmazunk-e vagy sem -- két lehetőség kínálkozik. Vagy egy szignálkezelő függvényt kötünk fel, amiben logikai igaz értékkel (\texttt{True}) térünk vissza, azt mondva ezzel a \textit{GTK} szignált feldolgozó kódjának, hogy ezt az eseményt kezeltük, vagy saját osztályunkban írjuk felül az alapértelmezett eseménykezelőt (\texttt{do\_delete\_event}), amiben hasonlóképpen járunk el. Mivel a \textit{Python} nyelvben gyakorlatilag minden függvény virtuális ezt minden további nélkül megtehetjük.

 \item[\ref{gtksignalpy:callbackdeleteevent}. sor] A kezelő függvény a nyelvi különbségektől eltekintve teljesen azonos a \textit{C}, illetve \textit{C++} nyelvű változatokéval, vagyis mindkét függvény paraméterként megkapja a kezelt eseményt leíró adatstruktúrát, illetve azt az objektumot amin az esemény kiváltódott. Kicsit korrektebbül fogalmazva az objektum-orientált változat valójában azon objektumra kap referenciát, melynek osztályához a kezelő függvény tartozik, de ebben az esetben ez a kettő egybeesik.
\end{description}

\paragraph{Gomb lenyomásának kezelése}

\begin{description}
 \index{GtkButton@\texttt{GtkButton}!szignálok!clicked@\texttt{clicked}}
 \item[\ref{gtksignalpy:connectclicked}. sor] Mivel a \textit{Python} esetén a \textit{widget}ek -- a \textit{C++} változathoz hasonlóan --  valódi objektumok, az eseménykezelő függvények bekötése az objektumon keresztül történik. Az eseménykezelő függvények első paramétere maga az objektum lesz. A szignál bekötésekor megadott további paraméterek a kezelő függvénynek adódnak át. A \textit{C} változattal ellentétben -- ahol csak egy paraméter adható meg -- a \textit{Python} képes egyszerre több paraméter átadására is -- éppúgy, mint a \textit{gtkmm} -- ugyanakkor itt természetesen olyan szigorúan vett típusellenőrzésről nem beszélhetünk, mint a \textit{C++} nyelv esetén.
\end{description}

\paragraph{Az ablak bezárása}

\begin{description}
 \index{GtkButton@\texttt{GtkButton}!szignálok!destroy@\texttt{destroy}}
 \index{Gtk@\texttt{Gtk}!tagfüggvények!main\_quit@\texttt{main\_quit}}
 \item[\ref{gtksignalpy:connectdestroy}. sor] Az ablak bezárásának logikája azonos a \textit{C}  változatnál elmondottakkal, vagyis az ablak \texttt{destroy} szignáljának kezelőjeként a \texttt{main\_quit} függvényt adjuk meg, azaz az esemény bekövetkeztekor a \textit{Gtk} főciklusa, egyszersmind a programunk is kilép.

 \index{GtkButton@\texttt{GtkButton}!szignálok!clicked@\texttt{clicked}}
 \index{GObject@\texttt{GObject}!tagfüggvények!connect@\texttt{connect}}
 \index{GObject@\texttt{GObject}!tagfüggvények!connect\_object@\texttt{connect\_object}}
 \index{szignál!eseménykezelő felkötése}
 \item[\ref{gtksignalpy:signaldestroy}. sor] Ezt úgy érjük el, hogy gombunk \texttt{clicked} szignáljának kiváltására meghíjuk az ablak \texttt{destroy} függvényét. Hasonlóan a \textit{C} változathoz, itt is alkalmazunk némi cselt. Mivel ott csak egy paraméter adható át -- pontosabban kettő, hiszen az első maga az objektum, aminek a szignáljára a kezelőfüggvényt felkötöttük --, a paramétereket meg kell fordítanunk, hogy az általunk megadott adat kerüljön az első helyre, vagyis ez legyen a \texttt{gtk\_widget\_destroy} által megkapott egyetlen paraméter. Voltaképpen a \texttt{connect\_object} ugyanezt a célt szolgálja, így ellentétben a \texttt{connect} függvénnyel ennek csak egy paramétere van, mely ez esetben maga az ablakunk, amit a \texttt{destroy} függvénynek adunk át.
\end{description}

\subsection{Fordítás és futtatás}
\index{fordítás}
\index{futtatás}

A korábbiakhoz hasonlóan az alábbi parancssorok segítségével fordíthatóak elemzett programjaink:

\lstcompiles
{gtk_signal.c}{gtk_signal}
{gtkmm_signal.cc}{gtkmm_signal}

Próbálkozzunk ezúttal a \texttt{./gtk\_signal}, \texttt{./gtkmm\_signal}, illetve \texttt{python gtk\_signal.py} parancsokkal abban a könyvtárban, ahol a fordítást elkövettük, illetve a \textit{Python} forrást tartjuk.

\subsection{Eredmény}

Bármily hihetetlen ezúttal sem történik sok egyéb, mint a korábbi minimális példa esetén. A különbség remélhetőleg annyi, hogy a meglepetéssel teli borzongást legutóbb ablakunk váratlan felbukkanása, míg most a bennünk szikraként felvillanó megértés okozza.


\part{Alapvető \textit{widget}ek}

\chapter{Ablakok}
\section{Bevezetés}
\index{GtkWindow@\texttt{GtkWindow}}
\index{GtkDialog@\texttt{GtkDialog}}

Ebben a részében a \textit{GTK}-ban létrehozható különböző ablaktípusok közös vonásait, valamint eltéréseit, illetve ezek okait vesszük sorra. Kitérünk egyrészről az egyes ablaktípusok létrehozásának sajátosságaira, azok \textit{widget}ekkel való feltöltésére, másrészről a felhasználó interakciók kezelésére, ezzel együtt az ablakok bezárásának módjaira is, szem előtt tartva természetese a \textit{C}, \textit{C++}, illetve \textit{Python} nyelvű változat azonosságait, különbözőségeit.

\subsection{\textit{Popup} és \textit{toplevel} ablakok}
\label{sec:windowtype}
\index{ablak!típus!popup}
\index{ablak!típus!toplevel}
\index{GtkWindow@\texttt{GtkWindow}!tulajdonságok!type@\texttt{type}}

A \textit{popup} ablakra, mint típusra ugyan ritkán lesz közvetlenül szükségünk, érdemes tudni, hogy a \textit{GTK} ebben a tekintetben két fajta ablakot különböztet meg. A \textit{popup} (felugró, felbukkanó) ablakokat, melyekre -- valamilyen speciális célt szolgáló saját készítésű \textit{widget}ektől eltekintve -- csak néhány példa létezik (\textit{menu}, \textit{tooltip}), valamint a \textit{toplevel} (legkülső, legfelső szintű) ablakokat, melyek csaknem minden \textit{GTK}-s, illetve saját fejlesztésű ablak alapjául szolgálnak. Ha tehát az ablakra, illetve a hozzá kapcsolódó fogalmakra gondolunk, többségében egy \textit{toplevel} ablakra gondolunk, és nem a \textit{popup}\footnote{Más eszközkészletek a ``popups'' gyűjtőfogalom alá sorolják a dialógusokat, a \textit{GTK} esetén azonban egy dialógus ablak mindig egy \textit{toplevel}} típusúakra, melyekről talán nem is feltételeznénk első ránézésre, hogy ablakok.

\index{GtkWindow@\texttt{GtkWindow}!tulajdonságok!type@\texttt{type}}
Az ablakkezelő ezt az információt használja fel annak eldöntésére, hogy az adott ablakot milyen kerettel, dekorációval lássa el, illetve hogy általánosságban menedzselje e ablakot. Utóbbi -- azaz a \textit{toplevel} ablakok -- esetben alapértelmezetten az ablakkezelő keretet, illetve a beállításoktól függően azon például bezáró, teljes mértre váltó, minimalizáló gombot jelenít meg. A \textit{popup} típusú ablakokat az ablakkezelő nemcsak hogy nem dekorálja, de nem is menedzseli azokat, következésképp tehát számos -- az ablakkezelő hatáskörébe tartozó -- funkció, mint amilyen például a minimalizálás, vagy a maximalizálás nem is érhető el. Bár kézenfekvő megoldásnak látszik a \textit{popup} típus arra, ha egy dekoráció nélküli ablakot készítsünk, mégse ezt tegyük, az ilyen típusú megjelenésbeli sajátosságok beállítására léteznek külön függvények.

\index{GtkBin@\texttt{GtkBin}}
Minden \texttt{GtkWindow} egyben konténer is, pontosabban fogalmazva egy \textit{GtkBin}, azaz tartalmazhat egy további elemet gyerekként, ami természetesen szintén lehet egy konténer, így biztosítva, hogy számos elemet helyezhessünk el az elkészített ablakon belül.

\subsection{\textit{Window} és \textit{dialóg}}
\index{GtkWindow@\texttt{GtkWindow}}
\index{GtkDialog@\texttt{GtkDialog}}
\index{GtkBox@\texttt{GtkBox}!függvények!pack\_end@\texttt{pack\_end}}
\index{GtkBox@\texttt{GtkBox}!függvények!pack\_start@\texttt{pack\_start}}

\label{par:dialogbox}
\index{ablak!típus!toplevel}
\index{GtkBox@\texttt{GtkBox}}
\index{GtkButtonBox@\texttt{GtkButtonBox}}
\index{GtkSeparator@\texttt{GtkSeparator}}
\index{GtkDialog@\texttt{GtkDialog}!belső elemek!action\_area@\texttt{action\_area}}
\index{GtkBox@\texttt{GtkBox}!függvények!pack\_start@\texttt{pack\_start}}
\index{GtkBox@\texttt{GtkBox}!függvények!pack\_end@\texttt{pack\_end}}
A \textit{window} típus -- azon belül is ahogy tárgyaltuk a \textit{toplevel window} -- közvetlen szülője a \textit{dialog} típusnak, számottevő különbség tulajdonképpen nincs is a kettő között. Egy \textit{dialog} nem más, mint egy olyan \textit{window}, melybe a \textit{GTK+} fejlesztői néhány hasznos elemet helyeztek el. Konkrétabban fogalmazva minden dialógba egy függőleges elrendezésű konténer \textit{widget} (\texttt{GtkBox}), abba pedig egy, a gombok elhelyezésére szolgáló konténer (\texttt{GtkButtonBox} típusú \texttt{action\_area}), valamint egy szeparátor (\texttt{GtkSeparator}) kerül, ebben a sorrendben mindkét esetben a konténer aljára helyezve\footnote{A \texttt{GtkBox} típus \texttt{pack\_end()} függvényét hívva.}. Ebből következik, hogy minden, amit egy dialógusba -- annak elkészült után -- tenni akarunk az a gombsor, valamint a vízszintes szeparátor fölött jelenik meg, függetlenül attól, hogy azt a \texttt{pack\_start()}, vagy a \texttt{pack\_end()} függvény segítségével helyezzük el a konténerben.

\index{GtkDialog@\texttt{GtkDialog}!belső elemek!content\_area@\texttt{content\_area}}
A \textit{dialog} típus tehát -- a szeparátor által -- függőlegesen ketté osztott \textit{window}, ahol az alsó rész (\texttt{action\_area}), ami általában a gombokat tartalmazza (pl.: Ok, Mégse, Súgó, \dots), a felső (\texttt{content\_area}) pedig azokat az elemeket tartalmazza, amik a felhasználói számára a szükséges akcióhoz (pl.: adatbevitel, hibaüzenet megjelenítése, \dots) szükséges.

\subsection{Modalitás}
\label{sec:windowmodal}

\index{ablak!modalitás}
\index{GtkWindow@\texttt{GtkWindow}!tulajdonságok!modal@\texttt{modal}}
A több ablakkal történő párhuzamos interakció tiltására szolgál a \textit{window} \textit{modal} tulajdonsága. Amennyiben egy ablak ``modális'' csak az abban az ablakban elhelyezkedő \textit{widget}ekbe történhet például bevitel, csak azokon váltódhat ki valamilyen felhasználó által kezdeményezett esemény. Ezt kihasználva biztosíthatjuk például, hogy egy adatbeviteli ablak\footnote{Ilyen lehet például a \textit{szerkesztés} menüpontok \textit{beállítások} almenüjének hatására megjelenő ablak.} bezárásáig ne változzon semmilyen, a felhasználó által módosítható, \textit{widget} tartalma a háttérben.

Modalitásnak két formáját különböztetjük meg. Egyrészről -- amiről eddig is szó esett -- a csak az applikációra vonatkozó modalitást, mely lehetővé teszi, hogy más applikációk ablakaihoz minden további nélkül hozzáférhetünk. Másrészről a teljes rendszerre érvényes modalitást, ahol a modális ablakon kívüli ablakokkal folytatott minden nemű felhasználó interakció tiltott. Ez utóbbi módszert csak a legszükségesebb esetben -- már ha van ilyen -- célszerű alkalmazni és az előbbi is csak akkor fogadható el felülettervezési szempontból, ha az applikáció egyéb részeihez való hozzáférés adatvesztést, vagy más komoly hibát okozna. Amennyiben mégis a modalitás mellett döntünk, ami nem ritka, hiszen az adatbevitelre, módosításra használt ablakok majd mindegyike ilyen, fontos egyértelművé tenni a felhasználó számára, hogy miként hagyhatja el azt az ablakot, ami korlátozza az applikáció más részeihez való hozzáférését. Egy ilyen menekülő útvonal biztosításának kézenfekvő módja lehet például egy \textit{Mégse} feliratú gomb.

\subsection{Tranziencia}
\label{sec:windowtransientfor}

\index{ablak!tranziencia}\index{GtkWindow@\texttt{GtkWindow}!tulajdonságok!transient-for@\texttt{transient-for}}
A dialógusok rendszerint ``tranziensek'' arra az ablakra, melyből származnak, azaz arra az ablakra melyen azt a műveletet váltottuk ki, aminek hatására a dialógus megjelent. Ezen beállítás alapján az ablakkezelő képes a dialógusunkat előtérben, a szülőablak fölött tartani\footnote{Helytelen beállítások esetén -- ha rosszul, vagy egyáltalán nem adjuk meg a szülőablakot -- előfordulhat, hogy egy újonnan létrehozott és megjelenített dialógusunk a már létező ablakok alatt, vagy között kerül megjelenítésre, ami felhasználói szempontból roppant zavaró}, valamint ha arra kérjük, akkor a szülő ablakhoz képest középen megjeleníteni (\ref{sec:windowpos}).

\index{GtkWindow@\texttt{GtkWindow}!tulajdonságok!destroy-with-parent@\texttt{destroy-with-parent}}
Ez a funkció azonban nem csak az ablakok helyes megjelenítéséhez szükséges, a megszüntetésükkor is hasznos, hiszen a \texttt{destroy-with-parent} tulajdonságon (\textit{property}) keresztül lehetőség van arra utasítani a \textit{GTK}-t, hogy egy ablak megszűnésekor azokat az ablakokat is szüntesse meg, melyek erre a szülőablakra nézve tranziensek. Ez leginkább akkor hasznos, hogy ha egy bizonytalan ideig létező ablakra szeretnénk tranziensek lenni\footnote{Erre lehet példa egy nem modális ablak, ami a programfutása során is megszűnhet}. Így nem kell törődnünk azzal, hogy ablakaink esetleg ``árván'' maradnak.\label{par:windowdestroywithparent}

\section{Használat}

\subsection{Létrehozás}
\index{GtkWindow@\texttt{GtkWindow}}
\index{GtkDialog@\texttt{GtkDialog}}

Mind a \texttt{GtkDialog}, mind pedig a \texttt{GtkWindow} típus létrehozása -- már ami formai részt illeti --, teljesen hasonló az összes többi \textit{widget}éhez, van azonban egy érdemi különbség, amire érdemes kitérni. Az ablakok -- igaz ez természetesen az összes többi \texttt{GtkWindow} típusból származó \textit{widget}re is (pl.: \texttt{GtkMessageDialog}, \texttt{GtkAboutDialog}, \dots) -- természetüknél fogva nem kerülnek bele más konténerbe, hiszen pont ezek azok a típusok, amik \textit{widget}eket tartalmaznak. Ellentétben azonban a többi típussal, ahol a referenciaszámlálás megoldja a problémát, itt a létrejött \textit{widget}ek felszabadításáról magunknak kell gondoskodnunk.

\subsubsection{Paraméterek}

A \texttt{GtkDialog} létrehozásában már valamivel nagyobb a különbség, függően attól, hogy \textit{GTK+}-t, vagy \textit{gtkmm}et használunk, bár így sem számottevő. Minkét esetben meg kell adnunk a címsor szövegét, valamint azt az ablakot, amire tranziensek kívánunk lenni. \textit{GTK+} esetén -- ahogy látszik -- lehetőségünk van \texttt{NULL} érték megadására, ami azt jelenti, hogy nem kívánunk ezzel a lehetőséggel élni. A \textit{gtkmm} is lehetséges ez, ha az alább látható konstruktort helyett azt hívjuk, amelyből hiányzik a \texttt{parent} paraméter.

\lsttriplesourcev
{sources/window_create.h}
{sources/window_create.hpp}
{sources/window_create.py}
{\textit{Window} létrehozása}
{lst:windowcreate}

\index{ablak!típus}
\index{ablak!típus!popup}
\index{ablak!típus!toplevel}
Ahogy arról szó esett (\ref{sec:windowtype}) a \textit{GTK} két típust különböztet meg -- a \textit{popup} és \textit{toplevel} \textit{window} -- melyek közül az előbbi olyannyira ritkán használt, hogy a \textit{C++} nyelvű változat esetén az alapértelmezett paramétere is van a típus konstruktorának, ahol a típus alapértelmezett értéke \textit{toplevel}.

\index{GtkWindow@\texttt{GtkWindow}!tulajdonságok!modal@\texttt{modal}}
A \textit{C}, illetve a \textit{Python} változat \texttt{flags} paramétere egyben tartalmazza a \textit{gtkmm} \texttt{modal} (\ref{sec:windowmodal}) és a \texttt{destroy-with-parent} (\ref{par:windowdestroywithparent}) értéket egy \textit{bitmask} értékben. Ez utóbbi beállítására ugyan van lehetőség \textit{gtkmm} esetén is, a \textit{C++} nyelvi eszközeinek korrekt használata mellett nemigen van szükség\footnote{Ha egy ablakra egy tranziens dialógust akarunk megjeleníteni, az nyugodtan lehet adattag, aminek megszüntetéséről az ablak destruktorában gondoskodhatunk.}.

\lsttriplesource
[numbers=none]
{sources/dialog_create.h}
{sources/dialog_create.hpp}
{sources/dialog_create.py}
{\textit{Dialog} létrehozása}
{lst:dialogcreate}

\subsubsection{Pozíció}
\label{sec:windowpos}
\index{ablak!pozíció}

Egy ablak képernyőn elfoglalt pozíciójának megadására alapvetően két lehetőség kínálkozik. Az egyik, ha előre -- még az ablak megjelenítése előtt -- megadjuk, a kívánt elhelyezkedést. Ehhez a \textit{GTK} annyiban tud segítségünkre lenni, hogy választhatunk néhány előre definiált elhelyezkedési pozíció közül, így nem szükséges a pixelben megadott koordináták kiszámítására időt és energiát pazarolni\footnote{Ami nem minden esetben kézenfekvő feladat, hiszen adott esetben nem csak a képernyő felbontásával, saját ablakunk méreteivel, de a szülőablak, vagy éppen a desktop szélesség és magasság értékeivel is foglalkozni kell.}.

\index{GtkWindow@\texttt{GtkWindow}!függvények!set\_position@\texttt{set\_position}}
\index{GtkDialog@\texttt{GtkDialog}!függvények!run@\texttt{run}}
\index{GtkWidget@\texttt{GtkWidget}!függvények!show@\texttt{show}}
A \texttt{set\_position} függvény még a megjelenítést megelőzően -- azaz \textit{window} esetén a \texttt{show}, \textit{daialog} esetén pedig a \texttt{run} meghívása előtt -- módunkban áll az alábbi elhelyezkedési sémák közül a megfelelőt kiválasztani.

\begin{description}
 \index{Gtk@\texttt{Gtk}!konstansok!WIN\_POS\_NONE@\texttt{WIN\_POS\_NONE}}
 \item[\texttt{WIN\_POS\_NONE}] Nincs befolyással a megjelenítést ablak pozíciójára nincs.

 \index{Gtk@\texttt{Gtk}!konstansok!WIN\_POS\_CENTER@\texttt{WIN\_POS\_CENTER}}
 \item[\texttt{WIN\_POS\_CENTER}] A megjelenítendő ablak a teljes képernyőhöz képest középen jelenik meg.

 \index{Gtk@\texttt{Gtk}!konstansok!WIN\_POS\_MOUSE@\texttt{WIN\_POS\_MOUSE}}
 \item[\texttt{WIN\_POS\_MOUSE}] A megjelenítendő ablak az egér aktuális pozíciója alatt jelenik meg.

 \index{Gtk@\texttt{Gtk}!konstansok!WIN\_POS\_ALWAYS@\texttt{WIN\_POS\_ALWAYS}}
 \item[\texttt{WIN\_POS\_CENTER\_ALWAYS}] A megjelenítendő ablak a teljes képernyőhöz képest középen jelenik meg és átméretezést követően is ott marad\footnote{A legtöbb esetben ez a választás nem szerencsés, lévén nem feltétlenül működik ez a mód minden ablakkezelő rendszer esetén.}.

 \index{ablak!tranziencia}
 \index{GtkWindow@\texttt{GtkWindow}!függvények!set\_transient\_for@\texttt{set\_transient\_for}}
 \index{Gtk@\texttt{Gtk}!konstansok!WIN\_POS\_CENTER\_ON\_PARENT@\texttt{WIN\_POS\_CENTER\_ON\_PARENT}}
 \item[\texttt{WIN\_POS\_CENTER\_ON\_PARENT}] A megjelenítendő ablak -- a \texttt{set\_transient\_for} függvénnyel beállított -- szülőjéhez képest középen jelenik meg.
\end{description}

\index{ablak!pozíció}
\index{GtkWindow@\texttt{GtkWindow}!függvények!move@\texttt{move}}
\index{GtkWindow@\texttt{GtkWindow}!tulajdonságok!gravity@\texttt{gravity}}
Ha úgy látjuk a fenti lehetőségek nem felelnek meg maradéktalanul céljainknak, akkor lehetőségünk van arra, hogy ablakunkat a kívánt pozícióra mozgassuk. Megjegyzendő, hogy ez a mozgatás csupán egy kérés az ablakkezelő felé, amit az figyelmen kívül is hagyhat. Az ablakkezelők jelentékeny része ezt meg is teszi amennyiben ezzel a módszerrel kívánjuk az ablak kezdeti pozícióját meghatározni, viszont honorálja kérésünket, ha az ablak korábban már megjelenítésre került. Az ablak elhelyezkedésének megadása \textit{x}, \textit{y} koordinátákkal történik egy választott referenciaponthoz képest, mely lehet az ablak bármely sarokpontja, az élek középpontja és az ablak középpontja egyaránt. A mozgatás maga a \texttt{move} függvénnyel történik, a referenciapontot pedig a \texttt{gravity} értéke határoz meg.

\index{GtkWindow@\texttt{GtkWindow}!függvények!get\_position@\texttt{get\_position}}
Egy ablak pozíciójának nem csak a beállítására, de lekérdezésére is szükség lehet, ugyanakkor ebben a tekintetben adott egy komoly megszorítás, amivel mindenképp szükséges számolni. A \texttt{get\_position} függvény által visszaadott értékek a már korábban említett módon függenek egyrészről a \texttt{gravity} értékétől, másrészről pedig az ablakkezelőtől. Elméletben, ha a visszakapott \textit{x}, illetve \textit{y} értéket átadnánk a \texttt{set\_position} függvények azt kellene tapasztalnunk, hogy az ablak egy helyben marad, gyakorlatban viszont azt tapasztalhatjuk, hogy az ablak valamennyit elmozdul. Ennek oka az ablakkezelő által az ablak köré rajzolt dekoráció, illetve annak geometriája, amit a \textit{GTK+} csak jó közelítéssel tud becsülni. Ez például akkor okozhat gondot, ha programunk ablakainak méretét és elhelyezkedését menteni szeretnénk, majd azt visszaállítanánk a következő futtatásnál. Érdemes tehát körültekintőnek lenni.

\subsubsection{Méret és arány}

A konténerek méretének meghatározásáról és elemeik (\textit{children}) elhelyezkedéséről leírtak (\ref{sec:packing}) a belső elrendezésük arányait átméretezéskor is megtartó ablakok kialakításakor válnak igazán fontossá.

\index{GtkWindow@\texttt{GtkWindow}!függvények!set\_default\_size@\texttt{set\_default\_size}}
\index{GtkWindow@\texttt{GtkWindow}!tulajdonságok!size-request@\texttt{size-request}}
\index{GtkWidget@\texttt{GtkWidget}!függvények!hide@\texttt{hide}}
\index{konténer!size request@\textit{size request}}
Egy ablak méretét, ha más erre vonatkozó beállítást nem teszünk -- épp úgy mint mint minden más konténerét -- a benne lévő elemek méretigény határozza meg, ugyanakkor lehetőség van ennek az alapértelmezés szerinti működésnek a módosítására. Egyrészről a \texttt{set\_default\_size} függvény révén, melynek megadható az ablak alapértelmezett vízszintes és függőeges mérete pixelben. Ennek hatása, hogy az ablak első megjelenítéskor\footnote{Egy esetleges eltüntetéskor \textit{hide} az ablak mérete úgymond mentésre kerül, azaz az alapértelmezett méret nem kerül újra alkalmazásra, ha az ablakot eltüntetjük \textit{hide}, majd újra megjelenítjük.} legalább ilyen méretű lesz. Ha azonban az ablakban tárolt \textit{widget}ek méretigénye azt indokolja, akkor a megadott szélesség és magasság értékeknél nagyon méretben kerül megjelenítésre. Ezt a működést természetesen a \textit{size request} megadása révén is elérhetnénk, de az alapértelmezett méret beállítása esetén a felhasználó csökkenteni tudja az ablak méretét, ha megadottak szerinti méretre nincs feltétlenül szükség.

\index{GtkBox@\texttt{GtkBox}!gyerek tulajdonságok!expand@\texttt{expand}}
\index{GtkBox@\texttt{GtkBox}!gyerek tulajdonságok!fill@\texttt{fill}}
\index{GtkWindow@\texttt{GtkWindow}!függvények!set\_resizable@\texttt{set\_resizable}}
\index{GtkWindow@\texttt{GtkWindow}!függvények!resize@\texttt{resize}}
Az ablak átméretezése kapcsán két felhasználói szempontból érdemleges kérdés merül fel. Az egyik, hogy engedjük-e az ablak átméretezését a felhasználónak. Erre a kérdésre adott válasz leginkább azon múlik, hogy mennyi időt kívánunk az ablak tervezésével tölteni, illetve mennyire van a felhasználónak igénye az átméretezésre. Ha időnk korlátos és valójában nincs szükség a méret megváltoztatására, akkor szerencsénk van, nyugodtan tilthatjuk ezt az interakció. Ha viszont ez nem lehetséges -- például azért, mert az alkalmazásunk főablakáról, vagy egy olyan dialógusról van szó, melyben egy lista jelenik meg, aminek hasznos lehet a lehető legtöbb helyet biztosítani --, akkor időt és energiát kell szálnunk az egyes widgetek viselkedésének megtervezésére. Át kell gondolnunk mely elemeknek foglalják el (\textit{expand}, \textit{fill}) azt a helyet, ami az átméretezés révén rendelkezésre áll majd. Egy-egy feleslegesen megnyúló \textit{widget} -- mondjuk egy teljes képernyőt elfoglaló beviteli mező, amibe mondjuk csak egy IP címet szeretnénk írni -- épp annyira szerencsétlenül mutat, mint amennyire zavaró az, ha hiába növeljük az ablak méretét, a lista, amiből több sort szeretnénk látni, mégsem nő. Ha mégis az átméretezés tiltása mellett döntenénk, akkor ezt a \textit{set\_resizable} függvény hívásával tudjuk elérni. A másik érdemleges kérdés a programból történő átméretezés, amire ugyan megoldható (\textit{resize}), de erősen ellenjavallt usability szempontból.

 \index{GtkWindow@\texttt{GtkWindow}!függvények!set\_geometry\_hints@\texttt{set\_geometry\_hints}}
A fentieknél is részletesebb beállítások a \textit{set\_geometry\_hints} függvénnyel tehetők meg. Az ablak minimális és a maximális vízszintes, illetve függőeges irányban külön-külön állíthatóak, ahogy az átméretezés lépésköze is, sőt az ablak méretarányának \footnote{$\mbox{szélesség} / \mbox{magasság}$ lebegőpontos számként adott értéke} (\textit{aspect ratio}) lehetséges legkisebb és legnagyobb értéke is megadható.

\subsection{Minimális példa}

Ennyi bevezető után lássuk egy olyan példát, ami a lehető legkevesebb kódsor mellet, még ha meglehetősen korlátozott funkcionalitás bíró, de működő alkalmazást eredményez. Az alábbi \textit{C}, \textit{C++}, illetve \textit{Python} nyelvű kód nem tesz egyebet, létrehoz egy ablakot, amit meg is jelenít azáltal, majd átadja a vezérlést a \textit{GTK}-nak azáltal, hogy futtatja a \textit{main loop}ot, 

\lsttriplesource
{sources/window_minimal.c}
{sources/window_minimal.cc}
{sources/window_minimal.py}
{Minimál példa \texttt{GtkWindow}hoz}
{lst:windowminimal}

A változatok bár egyformának tűnnek, néhány apróságban mégis eltérnek egymástól. Ezek egy része, mint a \texttt{window} változó deklarálásának helye, vagy paraméterezése, a programozási nyelv sajátosságaiból következik. Mások viszont, mint a \textit{main loop} futtatásának módja, illetve a függvény elnevezése már a nyelvi változat megalkotóinak belátásán műlik. A programok működése azonos, csupán a technikai megvalósításban vannak eltérések.

Ezekről korábban már szó esett, így itt ezeket nem részletezzük, inkább sorra vesszük miként vehetünk használatba egy frissen létrehozott ablakot, mit kell tennünk, ha elemeket szeretnénk elhelyezni az ablakban, ha vezérlő gombokkal szeretnénk látnánk el, majd a megjelenítést után a felhasználó interakciókat szeretnénk követni, illetve azokra reagálni.

\subsection{Tartalmi elemek}

\index{GtkBin@\texttt{GtkBin}}
Egy ablak típusa szerint nem más, mint egy konténer, pontosabban fogalmazva egy \texttt{GtkBin} (\ref{sec:bin}), amibe további elemeket tehetünk. Praktikusan ez az elem egy újabb konténer, rendszerint egy \texttt{GtkBox}. A \texttt{GtkDialog} esetén, mint azt a korábbiakban (\ref{par:dialogbox}) kifejtettük, a konténerbe helyezett \texttt{GtkBox} már adott.

Az így egy ablakba kerülő \textit{widget}ekre nem csak abban az értelemben tekintünk csoportként, hogy mindegyikük szülője -- \textit{toplevel widget} szinten -- ugyanaz az ablak, de vannak bizonyos tulajdonságok, melyek ugyan konkrétan a \textit{widget}re vonatkoznak, de összefüggésben állnak az ablak más \textit{widget}einek bizonyos tulajdonságaival, de mindig csak az azonos ablakban lévő más \textit{widget}ekéivel, vagyis erre csoport nézve zártak.

\subsubsection{Fókusz \textit{widget}}
\label{sec:widgetfocus}

\index{ablak!típus!toplevel}
\index{ablak!típus!popup}
\index{fókusz!billentyűzet}
Egy adott ablakban\footnote{Itt ablak alatt a \textit{toplevel} és nem a \textit{popup} ablakokat értjük.} egy adott pillanatban legfeljebb egy olyan \textit{widget} lehet, melyen fókuszban (\textit{keyboard focus}) van. Ha van ilyen \textit{widget}, akkor minden -- az ablak által fogadott -- billentyűzet esemény (billentyű lenyomása, felengedése, \dots) hozzá kerül továbbításra. Így érthető is, hiszen ha például gépelünk valamit a billentyűzeten, akkor annak eredményét értelemszerűen csak egy beviteli mezőben  szeretnénk látni.

\index{GtkEntry@\texttt{GtkEntry}}
\index{GtkTextView@\texttt{GtkTextView}}
\index{GtkContainer@\texttt{GtkContainer}}
\index{GtkContainer@\texttt{GtkContainer}!függvények!set\_focus\_chain@\texttt{set\_focus\_chain}}
A \textit{widget}ek többsége valamilyen látható módon is jelzi azt az állapotot, hogy aktuális fókuszban van. Ez az szövegbevitelre szolgáló \textit{widget}ek (\texttt{GtkEntry}, \texttt{GtkTextView}, \dots) esetén abban nyilvánul meg, hogy a kurzort látjuk villogni a beviteli mezőben, egyéb esetekben ezt egy vékony fekete keret jelzi. A fókusz egyik \textit{widget}ről a másikra történő mozgatására a szokások módszer, azaz a tab, illetve a kurzormozgató billentyűk használhatóak. Az egyes \textit{widget}ek között történő váltás is testre szabható a \texttt{GtkContainer} \texttt{set\_focus\_chain} függvényével, de erre valóban ritkán lehet szükség.

\begin{figure}[H]
\begin{center}
\includegraphics[height=15mm]{images/widget-keyboard-focus.png}
\caption{Billentyűzet fókusz jelzése a \textit{widget}en\cite{gnomehig}}
\end{center}
\end{figure}

\index{GtkWidget@\texttt{GtkWidget}!tulajdonságok!can-focus@\texttt{can-focus}}
\index{GtkWidget@\texttt{GtkWidget}!függvények!grab\_focus@\texttt{grab\_focus}}
Az egyes \textit{widget}ekre külön-külön engedhető, vagy tiltható, hogy fókuszba kerülhessenek, a \textit{can-focus} tulajdonság állításával. Az egyes \textit{widget}típusok esetében az alapértelmezés szerinti érték rendszrint megfelel a céljainknak\footnote{Ez az érték egy \texttt{GtkEntry} esetén igaz, míg egy \textit{Gtkabel} esetén hamis alapértelmezés szerint.}. Ha kódból szeretnénk átmozgatni a fókuszt az egyik \textit{widget}ről a másikra, vagy csak azt kívánjuk elérni, hogy az ablak megjelenítésekor legyen olyan \textit{widget} ami fókuszban van, akkor a \texttt{grab\_focus} függvényt kell alkalmaznunk, aminek előfeltétele a \textit{can-focus} tulajdonság igaz értéke, azaz fókuszálhatónak kell lennie, ami viszont bizonyos \textit{widget}ek (pl.: \texttt{GtkFrame}) esetén nem lehetséges.

\subsubsection{Alapértelmezett \textit{widget}}

\index{GtkDialog@\texttt{GtkDialog}
\index{gomb!jóváhagyó}
\index{gomb!ok}
\index{GtkWidget@\texttt{GtkWidget}!tulajdonságok!has-default@\texttt{has-default}}
A \texttt{GtkWindow} estén -- amit rendszerint olyan ablakhoz használunk aminek nincsenek gombjai -- ritkábban, míg \texttt{GtkDialog}} esetén csaknem mindig használt tulajdonság az alapértelmezett \textit{widget}. Ez ellentétben a fókusszal, ami inkább egy logikai tulajdonság, a felületre nincs, csak a működésre van hatással. Egy ablakon belül -- nevéből is következően -- legfeljebb egy olyan \textit{widget} lehet, mely rendelkezik ezzel a tulajdonsággal, ez rendszerint a dialógus jóváhagyó (\textit{affirmative}) gombja, jellemzően az \textit{Ok} gomb. 

\begin{figure}[H]
\begin{center}
\includegraphics[height=15mm]{images/button-affirmative.png}
\caption{Gombok szokások sorrendje egy dialógusban\cite{gnomehig}}
\end{center}
\end{figure}

\index{billentyű!tab}
\index{billentyű!enter}
\index{GtkEntry@\texttt{GtkEntry}}
\index{GtkTextView@\texttt{GtkTextView}}
\index{GtkWidget@\texttt{GtkWidget}!tulajdonságok!can-default@\texttt{can-default}}
\index{GtkWidget@\texttt{GtkWidget}!tulajdonságok!has-default@\texttt{has-default}}
\index{GtkEntry@\texttt{GtkEntry}!tulajdonságok!activates-default@\texttt{activates-default}}
Hatása abban áll, hogy az alapértelmezett \textit{widget} -- vagyis ami esetén a \texttt{has-default} tulajdonság értéke igaz -- aktiválódik akkor, ha egy egysoros beviteli mező van fókuszban (\texttt{GtkEntry}) és akkor \textit{Enter}t nyomunk, ez is csak akkor, ha az \texttt{GtkEntry} \texttt{activates-default} tulajdonsága szintén igaz értékű. Többsoros beviteli mező (\texttt{GtkTextView}) esetén ez nem működőképes, hiszen ott az \textit{Enter} lenyomása soremelést jelent. Az alapértelmezett \textit{widget}nek a billentyűzetről való használat kényelmesebbé tételében van szerepe, hiszen ha minden szükséges mezőt kitöltöttünk egy dialógusban, akkor nem kell a megfelelő gombig -- egérrel, vagy \textit{Tab} billentyű(k) lenyomásával -- elnavigálnunk, csak egyszerűen az \textit{Enter} leütésére aktiválódik az alapértelmezett \textit{widget}.

Ahhoz, hogy egy \textit{widget} egyáltalán számításba kerüljön, mint potenciális alapértelmezett \textit{widget}, ahhoz először a \texttt{can-default} tulajdonságának kell igaznak lenni, ilyen \textit{widget} több is lehet egy ablakon belül, melyek közül aktuálisan alapértelmezetté \texttt{GtkWidget} osztály \texttt{garb\_default} függvényével tehetünk egyet, így annak \texttt{has-default} értékkel igazzá válik, míg az ablak korábbi alapértelmezett \textit{widget}e, már ha volt ilyen, esetén a tulajdonság értéke természetesen hamis lesz.

\subsection{Vezérlő elemek}
\label{sec:windowvsdialog}

Amennyiben a szükséges tartalmi elemeket elhelyeztük az ablakban, a vezérlő elemekkel is hasonlóan kell eljárnunk. A különböző célokra használt ablakok különböző vezérlő elemeket kívánnak meg, amik az egyes típusokként erős hasonlóságot mutatnak. Egy főablak csak kivételes esetekben tartalmaz az eddigiekben tárgyalt gombokat, a vezérlés általában menükkel, a \textit{toolbar}on elhelyezett gombokkal történik (\ref{fig:windowprimary} ábra).

\includetwingraphics
{Főablak}
{window-primary.png}
{windowprimary}
{Dialógus}
{window-dialog.png}
{windowdialog}
{Tipikus ablakszerkezetek\cite{gnomehig}}
{windowtypes}

A főablakból nyíló különböző célú ablakok, melyek például egy adott elem tulajdonságainak beállítására, egy bonyolultabb funkció lépésenként történő megvalósítására, az alkalmazás egészének konfigurálására, egy folyamat nyomkövetésére, esetleg a felhasználó informálására, figyelmeztetésére szolgálnak, közvetlenül, vagy közvetve a \texttt{GtkDialog} típusból szármáznak, saját gombsorral látjuk el őket. Tipikus példa erre egy elem tulajdonságainak szerkesztésére használt dialógus (\ref{fig:windowdialog} ábra).

\subsubsection{\texttt{GtkDialog}}
\label{sec:dialogbuttonadd}
\index{GtkDialog@\texttt{GtkDialog}}

A \textit{C}, illetve a \textit{Python} nyelvű változat mutat némi különbözőséget a \textit{C++} változathoz képest a gombok hozzáadásának mikéntjében. Előbbiek esetén ugyanis több gombot is hozzáadhatunk egyszerre a dialógoshoz, egymás után sorolva a \texttt{button\_text} és a \texttt{response\_id} paramétereket. A paraméterlistát a \textit{C} változat esetén \texttt{NULL} értékkel kell zárnunk, különben változatos programhibákkal fogunk szembesülni, erre a \textit{Python} esetén nincs szükség, mivel itt a hívott fél tisztában van az átadott paraméterek számával.

\lsttriplesource
[numbers=none]
{sources/dialog_button_add.h}
{sources/dialog_button_add.hpp}
{sources/dialog_button_add_init.py}
{Gombok hozzáadása \texttt{GtkDialog}hoz\cite{gnomehig}}
{lst:dialogbuttonaddh}

\index{GtkStockId@\texttt{GtkStockId}}
A választott nyelvi változattól függetlenül igaz, hogy a \texttt{button\_text} paramétere vagy az általunk vágyott felirat szövege lehet, vagy egy \textit{stock ID} (\textit{Ok} gomb esetén például \texttt{GTK\_STOCK\_OK}).

\begin{figure}[H]
\begin{center}
\includegraphics[height=13mm]{images/button-alternate.png}
\caption{Egy tipikus gombsor}
\end{center}
\end{figure}

Egy fenti elrendezésű -- amúgy meglehetősen szokványos -- gombsor az alábbi kódrészletekkel hozható létre az egyes nyelvek esetén. Az eltérés nem számottevő, nem tartalmaz semmilyen olyan különbözőséget, ami a korábbiakban már ne került volna ismertetésre.

\lsttriplesource
[numbers=none]
{sources/dialog_button_add.c}
{sources/dialog_button_add.cc}
{sources/dialog_button_add.py}
{Tipikus gombsor hozzáadása \texttt{GtkDialog}hoz\cite{gnomehig}}
{lst:dialogbuttonadd}

\subsubsection{\texttt{GtkMessageDialog}}
\label{sec:messagedialog}
\index{GtkMessageDialog@\texttt{GtkMessageDialog}}

Létezik a \texttt{GtkDialog} típusnak egy -- a gombok hozzáadása szempontjából érdekes sajátossággal bíró -- specializált változata, melyet a felhasználóval történő kommunikáció céljaira használunk, s mely ennek megfelelően rendszerint csak az üzenet szövegét, illetve a válasz megadásához szükséges vezérlő elemeket tartalmazza.

\includetwingraphics
{Információs üzenetablak}
{message-dialog-information.png}
{messagedialoginformation}
{Hiba üzenetablak}
{message-dialog-error.png}
{messagedialogerror}
{Tipikus üzenetablakok}
{messagedialogtypes}

Az ábrákon látható dialógusokat -- a szövegezéstől eltekintve -- csak típusuk különbözteti meg egymástól. A bal oldali (\ref{fig:messagedialoginformation}) egy információs ablak, míg a jobb oldali (\ref{fig:messagedialogerror}) egy hibadialógus. Az egyik szembeötlő különbség az ablakok között az ikon, amit az üzenetablak típusa határoz meg. A fenti két típuson kívül még két saját ikonnal rendelkező típus (\textit{question} és \textit{warning}) létezik, illetve készíthetünk ikon nélküli változatot, aminél módunk van saját ikon megadására.

Az üzenetablak típus eltérése, vagy inkább specialitása a \texttt{GtkDialog} típushoz képest nem csak a típus megadásának lehetőségére korlátozódik. A kifejezetten a szoftver és a felhasználó közötti ``üzenetváltás'' célját szolgáló \textit{widget} rendelkezik beépített elemekkel az üzenet megjelenítésére. A felhasználóval közölni kívántakat egy elsődleges és egy másodlagos (\texttt{primary-text}, \texttt{secondary-text}) részre bonthatjuk, ahol az előbbi egy rövid, csak a helyzet leglényegesebb elemit tartalmazó, egy mondatos összefoglalója a közölni kívánt információnak, vagy a javasolt kívánt műveletnek, míg az utóbbi ennek mélyebb, részletekbe menő kifejtése leírása, ami tájékoztatja a felhasználót a felmerült helyzet okairól, esetleges mellékhatásairól. Az esetek többségében a felhasználónak már az elsődleges szöveg elolvasását követően meg kell tudni hoznia döntését, a másodlagos szöveg a döntés alátámasztására, az esetleges kétségek eloszlatására szolgál.

A harmadik specialitás -- a gombok felhelyezésének mikéntje -- a vezérlő elemek szempontjából is említésre méltó. Azzal együtt, hogy \textit{dialog} típusnál ismertetett módszer az öröklődés okán természetesen itt is használható, mivel azonban a leggyakrabban használt gombkombinációk száma erősen korlátos, így ezek közül létrehozáskor választhatunk. Lehetséges értékek eredményeként vagy egyedüliként a \textit{Bezárás}, \textit{Ok}, \textit{Mégse} gombok, vagy a \textit{Ok}/\textit{Cancel}, \textit{Igen}/\textit{Nem} párosok kerülnek a dialógusra.

\lsttriplesource
[numbers=none]
{sources/message_dialog_create.h}
{sources/message_dialog_create.hpp}
{sources/message_dialog_create.py}
{\textit{MessageDialog} létrehozása}
{lst:messagedialogcreate}

 \index{GLib@\texttt{GLib}!makrók!G\_GNUC\_PRINTF@\texttt{G\_GNUC\_PRINTF}}
A \textit{C}, illetve a \textit{Python} nyelvű változatok, ellentétben a \textit{C++}-os megvalósítással nem egyetlen paraméterként várják az üzenetablak elsődleges szövegét, hanem egy \texttt{printf}-stílusú formátumleírót és az annak megfelelő paramétereket vesznek át. Ennek típusbiztosságáról a \textit{C} változat esetén a \texttt{G\_GNUC\_PRINTF} makró gondoskodik, már amennyiben a \textit{GNU C} fordítót használjuk. Ebben az esetben fordítási idejű figyelmeztetést kapunk ha paramétereink nem felelnek meg a formátumleíróban megadottaknak.

\subsection{Megjelenítés}

Több lehetőség kínálkozik, ha egy ablakot, illetve annak tartalmát szeretnénk megjeleníteni. Használhatjuk egyrészről, a \texttt{GtkWidget}, a \texttt{GtkWindow}, illetve a \texttt{GtkDialog} típus által adott módszereket.

\subsubsection{\texttt{GtkWidget}}

\index{GtkWidget@\texttt{GtkWidget}!függvények!show@\texttt{show}}
\index{GtkWidget@\texttt{GtkWidget}!függvények!show\_all@\texttt{show\_all}}
A korábban már tárgyalt \texttt{show} függvényt, ami megjeleníti a \textit{window}t, de csak a \textit{window}t, annak gyerekeit nem. Lévén a \textit{window} egy konténer típus, helyezhetünk el további \textit{widget}eket benne, amiknek a megjelenítéséről vagy már korábban gondoskodnunk kell -- mondjuk egy \texttt{show} hívással --, vagy megtehetjük a szülő és az összes gyerek megjelenítését egyszerre a \texttt{show\_all} függvénnyel.

\subsubsection{\texttt{GtkWindow}}

\index{GtkWindow@\texttt{GtkWindow}!függvények!present@\texttt{present}}
Egy ablak esetén nem csupán a puszta megjelenítés lehet szempont, hanem az is, hogy a felhasználó az ablakot észre is vegye. Ez az estek túlnyomó többségében adott, hiszen valamilyen felhasználói interakció révén jelenik meg az új ablak. Ha viszont nem erről van szó akkor szükséges lehet az megjelenítésen túl más ablakok általi takarás megszüntetésére, a tálcáról való felhozatalra, az aktuális desktopra  történő mozgatásra, a fókusz (\ref{sec:widgetfocus}) átadására, mely műveletek mind függhetnek mind a platformtól, mind az ablakkezelőtől, mind pedig a felhasználói beállításoktól. Erre használható a \texttt{present} függvény. Kezeljük azonban kellő óvatossággal ezt a függvényt, hiszen mindannyian bosszankodtunk már egy kellő indok nélkül, váratlanul megjelenő ablak miatt.

\subsubsection{\texttt{GtkDialog}}

\index{main loop}
\index{GtkDialog@\texttt{GtkDialog}!függvények!run@\texttt{run}}
\index{GtkWindow@\texttt{GtkWindow}!tulajdonságok!modal@\texttt{modal}}
Egy \textit{window} típusú ablak esetén -- mivel jellemzően nincsenek az ablakon gombok és nem modálisak -- nincs igazán szükségünk arra, hogy -- a kód futásának szempontjából helyben -- kivárjuk a felhasználó reakciót. A \textit{dialog} típus ezzel szemben általában egy felhasználói interakció révén jelenik meg (pl.: egy elem tulajdonságait, vagy az applikáció beállításait szerkesztő ablak) és jelentős részben valamilyen döntés elé állítja a felhasználót (különösen igaz ez az üzenet ablakoknál, ahol kérést teszünk fel), melynek eredményéről szeretnénk értesülni. A \texttt{GtkDialog} típus \texttt{run} függvénye -- ahogy azt a következőekben (\ref{sec:dialogresponse}) részletesen is tárgyaljuk -- pontosan ezt a célt szolgálja. Egyrészről várakozik a felhasználó interakció -- amihez természetesen szükséges a \textit{main loop} futtatása -- majd visszatérési értékként az ablak ``futtatásának'' eredményét, vagyis a kiválasztott vezérlő elem -- az ablak elkészítésekor megadott -- \textit{response id}-t adja vissza. Mint látható, a függvény nem kifejezetten a dialógus megjelenítését szolgálja, az hasznos mellékhatásként mégis megtörténik.

\subsection{Bezárás}

A \textit{window} tehát leginkább a bezárás kapcsán állít kihívást elénk, melyet függően attól, mit szeretnénk elérni annak hatására, hogy a felhasználó az ablak bezárását kezdeményezte, több lehetséges megoldás, az egyes megoldásokra, pedig több módszer is kínálkozik.

\subsubsection{Blokkolás}

Kezdjük a legegyszerűbbnek látszó esettel, vagyis azzal, hogy semmilyen hatása ne legyen annak ha felhasználó az ablak bezárását kezdeményezi. Nyilván megfontolandó, hogy ezt tegyük, hiszen a felhasználó sem véletlenül akarja, amit akar, de ha mondjuk azt szeretnénk kikényszeríteni, hogy a főablakunkból csak a Fájl menüpont ``Kilépés'' almenüpontjára kattintva lehessen bezárni, akkor ez egy lehetséges megoldás\footnote{Ne becsüljük azonban alá se a felhasználói találékonyságot, se a felhasználói környezetek változatosságát. Semmiképp ne hagyatkozzunk arra, hogy egy adott módszer a felhasználó számára véleményünk szerint nem érhető el.}.

A feladat megoldása a korábban már tárgyalt \textit{delete-event} szignál kezelésében rejlik. Ahogy arról szó esett ez a szignál váltódik minden ablakon (\textit{toplevel window}), illetve minden az ablakban lévő \textit{widget}en, mikor a felhasználó az ablak bezárását kezdeményezi. A szignál két külön említésre is méltó sajátossággal is rendelkezik:

\begin{enumerate}
 \item A szignált kezelni kívánó függvénynek egy \textit{bool} értékkel kell visszatérnie, ami azt jelzi a \textit{GTK} felé, hogy az adott függvény kezelte-e az eseményt, egyszersmind nincs szükség a további kezelő függvény meghívására. A \textit{GTK} következésképp addig hívja sorra az szignálra ``feliratkozott'' függvényeket (\textit{signal handler}) -- köztük ha van ilyen, akkor az alapértelmezett szignálkezelő függvényt (\textit{default signal handler}) amíg egyikük \textit{true} értékkel nem tér vissza.
 \item Ha minden ilyen függvény \textit{false} értékkel tér vissza, azaz a szignál további propagálását kéri a \textit{GTK}-tól, akkor konkrétan a \texttt{delete-event} esetében a \textit{GDK} alapértelmezett eseménykezelője (\textit{default event handler}) fut le, ami meghívja az ablak destruktorát.
\end{enumerate}

Fentiek alapján ahhoz, hogy megelőzzük az ablak bezárását -- vagyis hogy ne történjen semmi -- el kell érnünk, hogy az alapértelmezett eseménykezelő ne hívódjon meg, azaz az általunk felkötött szignálkezelő \textit{true} értékkel kell hogy visszatérjen, jelezve azt, hogy az eseményt kezeltük, a további szignálkezelő függvények meghívására nincs szükség. Ebben az esetben ez azt is jelenti, hogy az alapértelmezett eseménykezelő sem hívódik meg. Ehhez definiálnunk kell a szignálkezelő  függvényeket, ami a korábbi (\lstref{lst:windowminimal}\footnote{A sorszámozás az eredeti példába való beillesztés pontját mutatja.}) példát kiegészítve az alábbihoz hasonló módon tehetünk meg,

\lsttriplesource
[firstnumber=2]
{sources/window_persistent_callback.c}
{sources/window_persistent_callback.cc}
{sources/window_persistent_callback.py}
{Szignálkezelő függvény perzisztens ablakhoz}
{lst:windowpersistentcallback}

majd ezeket a függvényeket a \texttt{delete-event} szignálra be is kell kötnünk\footnote{A sorszámozás az eredeti példába való beillesztés pontját mutatja.}.

\lsttriplesource
[firstnumber=12]
{sources/window_persistent_connect.c}
{sources/window_persistent_connect.cc}
{sources/window_persistent_connect.py}
{Szignálkezelő függvény bekötése perzisztens ablakhoz}
{lst:windowpersistentconnect}

Ha az adott szignál alapértelmezett szignálkezelő függvénnyel is rendelkezik -- ami a \textit{widget} osztály-leírójában kerül megadásra -- és magunk akarjuk a szignált kezelni, akkor szükségessé válik, hogy a saját kezelő függvényünk még az alapértelmezett előtt hívódjék meg, ami további praktikák bevetését igényli, amiről egy másik részben esett részletesebben szó.

\index{Gtk@\texttt{Gtk}!függvények!true@\texttt{true}}
\index{Gtk@\texttt{Gtk}!függvények!false@\texttt{false}}
Ha a saját szignálkezelő függvény írását kissé túlzónak találjuk egy olyan egyszerű feladat ellátására, mint egy \textit{true} értékkel való visszatérés, akkor nem tévedünk, sőt a \textit{GTK+} fejlesztői is gondoltak erre és megalkották a \texttt{gtk\_true}, illetve a \texttt{gtk\_false} nevű függvényeket, melyek semmi egyebet nem tesznek, mint a nevüknek megfelelő értékkel térnek vissza, így a fenti példa szignálkezelő függvényei elhagyhatók, hiszen azok \textit{GTK+} által adott -- ekvivalens funkciójú -- függvényekkel helyettesíthetőek.

\lstinputsource
[language=C]
{sources/window_persistent_simple.c}
{Egyszerűsített szignálkezelő függvény bekötése}
{lst:windowpersistentsimple}

\subsubsection{Eltüntetés}

Folytassuk másodikként azzal az eshetőséggel, ha el szeretnénk tüntetni az ablakot, aminek a bezárását a felhasználó kezdeményezte. Az előző szituációhoz hasonlóan most is több módszer adódik a feladat megoldására.

\index{GtkWidget@\texttt{GtkWidget}!függvények!hide@\texttt{hide}}
\index{GtkWidget@\texttt{GtkWidget}!szignálok!delete-event@\texttt{delete-event}}
A fent tárgyalt perzisztens ablakot eredményező szignálkezelők (\lstref{lst:windowpersistentcallback}) ismeretében a legnyilvánvalóbb megoldás, hogy még mielőtt visszatérnénk az adott függvényekből a \texttt{hide} függvény segítségével eltüntetjük az adott ablakot. A megoldás jó és működőképes megoldás lehet, ugyanakkor számolnunk kell azzal, hogy az ablak csak eltűnik, de nem feltétlenül semmisül meg. A már korábban is használt minimális példa azon kiegészítését is figyelembe vesszük, ahol a \texttt{delete-event} szignálra a \textit{main loop} futását megszakító függvényt hívjuk, akkor azt fogjuk tapasztalni, hogy az ablakunk eltűnik ugyan, de a mögöttes működés már nagyban attól függ a két implementációt (\lstref{lst:windowpersistentcallback}) miként vontuk össze. Ha a saját szignálkezelő függvényünket kötjük be előbb a \texttt{delete-event} szignálra, akkor az -- a visszatérési értéke révén -- leállítja a szignál további kezelését, vagyis a szignálkezelő függvények sorának meghívását is, így az ablak a saját kezelő függvényünk (\texttt{on\_delete\_event}), a \texttt{hide} hívással kiegészítve eltünteti az ablakot, de a következő szignálkezelő -- ami kiléptetné a \textit{main loop}ot -- már nem hívódik meg. Ha a felkötés sorrendje fordított, akkor előbb kilép a \textit{main loop} és csak ezután hívódik meg saját függvényünk, ami egyrészről elrejti az ablakot, másrészről blokkolja a további kezelést, aminek végső soron nem lehet hatása, hiszen a \textit{main loop} már kilépett.

Amennyiben a célunk csupán az ablak eltüntetése egyszerűbben is elérhetjük ugyanezt, hiszen a \texttt{hide} függvény közvetlenül -- pontosabban egy beépített \textit{GTK+} keresztül -- is beköthető a \texttt{delete-event} szignálra. Ezt kényelmi funkciót a \textit{gtkmm} esetén elveszítjük -- hasonlóan az előző példához --, mivel az általunk használni kívánt szignálkezelő függvény deklarációja nem egyezik meg az előírttal, így tehát ezt a ``kényelmet'' a típusbiztosság oltárán fel kell áldozni.

\lstinputsource
[language=C]
{sources/window_hide_on_delete_event_simple.c}
{Egyszerűsített szignálkezelő függvény bekötése}
{lst:windowpersistentsimple}

Ha nem ragadunk le a könnyen érthető, ám nem túl életszerű minimális példánknál, akkor az mondható el, hogy ablakokat -- legyenek azok \texttt{GtkWindow}, vagy \texttt{GtkDialog} típusúak --, valamilyen felhasználói interakcióra reagálva hozunk fel, valamilyen kezelő függvényben. Az bezáráskori elrejtés akkor lehet hasznos számunkra, ha nem akarjuk újra és újra elkészíteni az ablakot az adott felhasználói műveletre. Erre lehet példa mondjuk egy névjegy (\textit{about}) ablak, aminek a tartalma nem változik a program futása során, így azt az alkalmazás indulásakor, vagy az első megtekintéskor létrehozzuk, utána már elegendő csak elrejteni, vagy újra megjeleníteni. Másik példa lehet egy státusz jellegű információkat megjelenítő ablak, amiben úgymond gyűjteni tudjuk az adatokat és ha a felhasználó be is zárja az ablakot, mi az elrejtés után az adatgyűjtést és az ablak tartalmának frissítését tovább fojtatjuk, majd az újbóli megjelenítéskor már a naprakész információk jelennek meg.

\subsubsection{Megszüntetés}

\index{GtkWidget@\texttt{GtkWidget}!szignálok!delete-event@\texttt{delete-event}}
Mint az a fentiekből már kiderült -- függetlenül attól, hogy \texttt{GtkWindow}, \texttt{GtkDialog}, vagy ezekből származó típusokról van-e szó --, a \textit{delete-event} szignál kiváltódását -- ha egyebet nem teszünk -- az ablak destruktorának meghívás fogja automatikusan követni, ami az esetek túlnyomó többségében meg is felel a céljainknak.

\subsection{Eseménykezelés}
\label{sec:dialogresponse}

\subsubsection{Szinkron}

A \texttt{GtkDialog} és a \texttt{GtkWindow} típusok közötti eltérések (\ref{sec:windowvsdialog}) közül a legszámottevőbb -- mivel ehhez tartozik a legtöbb beépített szolgáltatás --, a vezérlő elemek, azaz a gombok kezelése. A \texttt{GtkDialog} nem csupán arra ad lehetőséget, hogy a gombokat egy erre a célra készült konténerbe helyezzük el -- ezt magunk is megtehetnénk minden különösebb erőfeszítés nélkül --, hanem az ezeken végzett felhasználói interakciókat is egyszerűen nyomon követhetjük. Választhatunk az adott (kód)környezetben számunkra kényelmesebb -- szinkron, illetve aszinkron eseménykezelés közül. Előbbi esetén helyben\footnote{Az ablak létrehozásának helyén.} tudjuk kezelni az eseményeket, utóbbi viszont eseménykezelő függvények megírását és bekötését teszi szükségessé.

\lsttriplesource
{sources/dialog_minimal.c}
{sources/dialog_minimal.cc}
{sources/dialog_minimal.py}
{Minimál példa \texttt{GtkDialog}hoz}
{lst:dialogminimal}

\index{main loop}
\index{GtkDialog@\texttt{GtkDialog}!függvények!run@\texttt{run}}
\index{GtkDialog@\texttt{GtkDialog}!szignálok!response@\texttt{response}}
\index{GtkWidget@\texttt{GtkWidget}!szignálok!unmap@\texttt{unmap}}
\index{GtkWidget@\texttt{GtkWidget}!szignálok!destroy@\texttt{destroy}}
\index{GtkWidget@\texttt{GtkWidget}!szignálok!delete-event@\texttt{delete-event}}
A \texttt{run} függvény (\ref{dialogminimalc:dialogrun}.) egyrészről függvényeket köt be a szükséges szignálokra (\texttt{response}, \texttt{unmap}, \texttt{delete-event}, \texttt{destroy}), majd egy saját \textit{main loop}ot futtatásán belül kezeli az említett eseményeket. Ha ezek közül bármelyik bekövetkezik, akkor a \textit{main loop} futása megszakad, és a \texttt{run} függvény a megfelelő értékkel visszatér. Ennek kezelése tipikusan egy \texttt{if}, vagy egy \texttt{switch} szerkezeten belül történik.

Ha a fenti minimális példát a korábbi, gombok hozzáadását tartalmazó forrással (\lstref{lst:dialogbuttonadd}) egészítjük ki\footnote{A változók deklarációinak hozzáadása szükséges a fordíthatóság érdekében.}, mondjuk az alábbihoz hasonló módon kezelhetjük a felhasználói döntés eredményeként kapott választ (\texttt{response}).

\lsttriplesource
[firstnumber=12]
{sources/dialog_run.c}
{sources/dialog_run.cc}
{sources/dialog_run.py}
{Minimál példa \texttt{GtkDialog}hoz}
{lst:dialogrun}

\index{Gtk@\texttt{Gtk}!konstansok!RESPONSE\_DELETE\_EVENT@\texttt{RESPONSE\_DELETE\_EVENT}}
Itt az egyszerűség kedvéért csak az \textit{Ok}, illetve a többi gomb kerültek megkülönböztetésre, így ha a \textit{Mégse}, illetve a \textit{Súgó} gomb aktiválódott, akkor is a \texttt{switch} szerkezet \texttt{default} ága (\ref{dialogrunc:switchcasedefault}) fut le. Kérdés azonban, hogy ha az imént megismert \texttt{delete-event} szignál váltódik ki az ablak bezáródásának hatására, annak mi lesz az eredménye. A válasz egyszerű, hiszen a \texttt{switch} első ágára (\ref{dialogrunc:switchcaseok}) nem futhatunk rá, tehát marad itt is a \texttt{default} ág. Ha külön szeretnénk kezelni ezt az esetet, akkor a \texttt{RESPONSE\_DELETE\_EVENT} konstans kezelésére kell egy új ágat beillesztenünk.

\index{GtkDialog@\texttt{GtkDialog}!függvények!run@\texttt{run}}
\index{Gtk@\texttt{Gtk}!konstansok!RESPONSE\_NONE@\texttt{RESPONSE\_NONE}}
Nem minden esetet fedtünk azonban le, elképzelhető ugyanis, hogy a dialóg felszabadul, míg a \texttt{run} függvény fut. Ebben az esetben a visszatérési érték \texttt{RESPONSE\_NONE} lesz, így ezt az esetet is meg lehet különböztetni a többitől, azonban mégsem tanácsos. Helyette, ha mindenképpen meg akarjuk szakítani a \texttt{run} futását, akkor a \texttt{resposne} függvényt hívhatjuk, aminek eredményeként a \texttt{run} a \texttt{resposne}-nak paraméterként átadott értékkel tér vissza.

\index{GtkWidget@\texttt{GtkWidget}!függvények!show@\texttt{show}}
Az élelmesebbek megfigyelhetik, hogy \textit{dialog} példaprogramból (\lstref{lst:dialogminimal}) a \textit{window} hasonló példájához (\lstref{lst:windowminimal}) kimaradt a \texttt{show} függvény hívása, ami annak tudható be, hogy a \texttt{run} ezt megteszi helyettünk, ahogy a dialógusunkat is modálissá teszi a \texttt{run} futásának idejére.

\subsubsection{Aszinkron}

\index{GtkDialog@\texttt{GtkDialog}!szignálok!response@\texttt{response}}
Ha valamilyen oknál fogva lemondanák a \texttt{run} adta kényelemről, lehetőségünk van az aszinkron kezelésre. Ebben az esetben a \texttt{run} helyett a \textit{show} függvényt hívjuk, illetve egy saját függvényt -- melyben elvégezzük a számunkra szükséges műveleteket -- kötünk be a \texttt{response} szignálra. Ezt a módszert használva ebben a függvényben kell gondoskodnunk az ablak felszabadításáról is, már amennyiben nem csak a \texttt{delete-event} esemény hatására szeretnénk, hogy ez megtörténjen. A \texttt{response} szignál paraméterei között a \texttt{response\_id} is szerepel, így a függvény tartalma hasonló lehet a \texttt{run} hívást követő kódhoz (\lstref{lst:dialogrun}). Működésben azonban van némi különbség, hiszen a \texttt{delete-event} alapértelmezett működésének megváltoztatására például nincs mód.

Az aszinkron a megoldásra lehet egy másik példa, ha a lehetséges választások mindegyike ugyanazzal az eredménnyel kell járjon, például be kell záródjon az ablak. Ennek olyan leginkább helyzetekben van realitása, ahol csupán egy (pl.: \textit{Bezárás}) gomb jelenik meg az ablakon. Erre lehet példa egy olyan üzenetablak (\textit{MessageDialog}), amiben nem egy kérdést teszünk fel, hanem csupán informáljuk a felhasználót.

\lstinputsource
[language=C]
{sources/dialog_destroy_on_response_simple.c}
{Egyszerűsített \textit{response}-kezelő függvény bekötése}
{lst:windowpersistentsimple}

\subsection{Saját eseménykezelő}

Ha a fentiekben vázoltak valamilyen oknál fogva nem elégítenék ki igényeinket, vagy már egyébként is alkalmasabb módszernek látszik egy saját \textit{widget} implementálása akkor nem kell egyebet tennünk, minthogy az ősosztály eseménykezelő függvényét felüldefiniáljuk a nyelvi változatnak megfelelő módon.

\lsttriplesource
{sources/mywindow_part.c}
{sources/mywindow_part.cc}
{sources/mywindow_part.py}
{Eseménykezelő felüldefiniálása saját \textit{widget}osztályban}
{lst:mywindowpart}

Az objektum-orientált megvalósítások esetén ez egy lényegesen kisebb erőfeszítést igénylő feladat. Ahogy ezt a fenti kódrészlet is mutatja a \textit{C} változatból épp csak a leglényesebb rész emelhető ki egy jó tucatnyi sorban\footnote{A teljes \textit{C} kód egy \ref{mywindowc:lastline} soros \texttt{.c}, illetve egy \ref{mywindowh:lastline} soros \texttt{.h} állományból áll.}, addig a csaknem teljes értékű \textit{C++}, \textit{Python} kódok ennek a felét sem teszi ki.

\section{Platformfüggő sajátosságok}

Bár a \textit{GTK} -- a \textit{GDK}\footnote{GIMP Drawing Kit}, illetve a \textit{Glib} függvénykönyvtárakon keresztül -- komoly platformfüggetlenséget biztosít, mégis számolnunk kell a különbségekkel, különösen az ablakok kezelése kapcsán. Ezek közül itt csak a két legfontosabbat emeljük ki.

\subsection{Ablakkezelő}

Bizonyos értelemben maga a az ablakkezelő is egy platform, hisz ahogy az operációs rendszerek a tőlük elvárt funkciókat -- mint amilyen például a fájlkezelés -- a maguk módján valósítják meg, úgy az ablakkezelő rendszerek is saját szisztémájuk szerint teszik ezt. Bizonyos funkciók csak néhány ablakkezelő implementál, addig másokat csaknem minden ilyen rendszer megvalósít, bár arra nem számíthatunk, hogy az egyes megvalósítások minden részletben megegyeznek.

Az ablakkezelők különbözőségei fejlesztési oldalról azzal a következménnyel járnak, hogy még akkor is szembe kell néznünk a platformok sajátosságai által okozott nehézségekkel, ha egyébként alkalmazásunkat nem használjuk több különböző\footnote{Ebben a tekintetben a \textit{Linux} alapú rendszereket közel azonosnak tekinthetjük} operációs rendszer alatt.

\index{GtkWindow@\texttt{GtkWindow}!függvények!stick@\texttt{stick}}
\index{GtkWindow@\texttt{GtkWindow}!függvények!iconify@\texttt{iconify}}
\index{GtkWindow@\texttt{GtkWindow}!függvények!maximize@\texttt{maximize}}
\index{GtkWindow@\texttt{GtkWindow}!függvények!fullscreen@\texttt{fullscreen}}
\index{GtkWindow@\texttt{GtkWindow}!függvények!set\_keep\_above@\texttt{set\_keep\_above}}
Még az olyan egyszerű és széles körben megvalósított funkciók kapcsán, mint amilyen maximalizálás, vagy a minimalizálás (\texttt{maximize}, \texttt{iconify}) kételkednünk kell a mögöttes implementáció meglétében, illetve figyelembe kell vennünk az egyes megvalósítások különbözőségeit, vagyis nem alapozhatunk ezen függvények meghívását követően arra, hogy az ablak abba az állapotba kerül, amire számítottunk. Akár az előbbiek említett funkciókra, akár mondjuk az ablak előtérben tartására (\texttt{keep\_above}), akár teljes képernyőméretre nagyításra (\texttt{fullscreen}), akár az összes munkaterület (\textit{desktop}) való egyszerre történő meg megjelenítésre (\texttt{stick}) kérjük az ablakkezelőt, nem bizonyos, hogy kérésünk teljesül. Ennek csak az egyik oka az, hogy az ablakkezelő nem támogatja az adott funkciót, a másik pedig, hogy kérésünket követően valaki más az előző, vagy éppen mindkettőtől eltérő állapot állít be. Ennél fogva ügyelnünk kell arra, hogy egy ilyen helyzetre programunk fel legyen készülve.

\index{GtkWindow@\texttt{GtkWindow}!függvények!set\_deletable@\texttt{set\_deletable}}
\index{GtkWindow@\texttt{GtkWindow}!függvények!set\_skip\_taskbar\_hint@\texttt{set\_skip\_taskbar\_hint}}
Az ablak dekorációja a másik olyan terület, ahol kéréseket (\textit{hint}) intézhetünk az ablakkezelő felé. Hasonlóan azonban a fent tárgyaltakhoz itt is igaz, hogy célszerű ellenőrizni feltételezéseinket arra vonatkozólag, hogy kérésünk úgy és abban a formában hajtódik végre, ahogy azt mi elgondoltuk. Számos olyan eset lehetséges az ablak dekorációján található bezáró gomb megjelenítésének tiltásától (\texttt{deletable}), az ablak \textit{taskbar}ból történő kihagyásáig (\texttt{skip\_taskbar\_hint}) melynek implementációs részletei teljes egészében az ablakkezelőtől függenek.

\subsection{Vezérlő elemek}

\index{gomb!menekülő}
\index{gomb!jóváhagyó}
\index{GtkDialog@\texttt{GtkDialog}!függvények!set\_alternative\_button\_order@\texttt{set\_alternative\_button\_order}}
\index{GtkDialog@\texttt{GtkDialog}!függvények!set\_alternative\_button\_order\_from\_array@\texttt{set\_alternative\_button\_order\_from\_array}}
A \textit{GTK} alapértelmezés szerint a gombokat a \textit{GNOME Human Interface Guideline}\cite{gnomehig} (\textit{HIG}) által javasolt elrendezését alkalmazza, vagyis a jóváhagyó (\textit{affirmative}) gomb a jobb oldalon a szélső pozícióba kerül, míg a menekülő gomb (\textit{cancel}) ettől balra kerül. Ha eltérő -- úgymond alternatív elrendezést -- szeretnénk, akkor azt a \texttt{set\_alternative\_button\_order} függvény segítségével egy korábbi példát (\lstref{lst:dialogbuttonadd}) kiegészítve az alábbi módon állíthatjuk be.

\lsttriplesource
[numbers=none]
{sources/dialog_button_order.c}
{sources/dialog_button_order.cc}
{sources/dialog_button_order.py}
{Alternatív gombsorrend beállítása}
{lst:dialogbuttonorder}

Amennyiben az üzenetablakoknál megismert (\ref{sec:messagedialog}) páros gombok (\textit{Ok}/\textit{Cancel}, \textit{Igen}/\textit{Nem}) valamelyikét használjuk, akkor ezt a sorrendállítást a \textit{GTK} megteszi helyettünk.

\section{A kód}

\subsection{Fordítás és linkelés}

A korábbiakhoz hasonlóan az alábbi parancssorok segítségével fordíthatóak programjaink.

\lstcompiles
{gtk_sourcefile.c}{gtk_binary}
{gtkmm_sourcefile.cc}{gtkmm_binary}

\subsection{Futtatás}

A futtatással ezúttal is a forrásfájlok -- egyszersmind a fordítás -- könyvtárában érdemes próbálkoznunk, a példaprogram nevétől függően a \texttt{./futtatható\_bináris} parancs kiadásával.

\subsection{Eredmény}

Bármily hihetetlen ezúttal sem történik semmi egyéb, mint a korábbiakban. Remélhetőleg azonban a különbség mégis érzékelhető annyiban, hogy legutóbb a meglepetéssel teli borzongást ablakunk váratlan felbukkanása, míg most a bennünk szikraként felvillanó megértés okozza.

\section{Tesztelés}

\subsection{Keresés}

\subsubsection{Applikáció keresése}

Fejlesztés közben -- már amennyiben a felhasználói felületet kódból és nem egy felülettervező program segítségével hozzuk létre -- az egyes \textit{widget}ek létrehozáskor módunkban áll egyúttal valamilyen változóval hivatkoznunk rájuk, így a későbbiekben nincs szükségünk arra, hogy az egyes \textit{widget}eket a felhasználói felületen belül keresgessük. Teszteléskor ugyanakkor a felületet készen kapjuk így az első feladat azon elemek megtalálása, amiken később valamilyen műveleteket szeretnénk végezni.

\lstdoublepysource[firstline=5, lastline=9, numbers=none]
{sources/dogtail_minimal_procedural.py}
{sources/dogtail_minimal_tree.py}
{Alapvető elemek keresése teszt \textit{tree}, illetve \textit{procedural} API használata mellett}
{lst:dogtailminimal}


\index{dogtail.tree@\texttt{dogtail.tree}!függvények!application@\texttt{application}}
Ahogy az a korábbi minimális tesztelési példában is látszott az első lépés a tesztelés során, hogy a kívánt applikációt megtaláljuk a sok egyéb futó alkalmazás között. Ha ezzel megvagyunk, akkor az applikáción belül kereshetjük meg annak ablakait, azokon belül pedig az egyes \textit{widget}eket. Előbbire a célra szolgál a \textit{Dogtail} \texttt{tree} moduljának \texttt{application} függvénye.

\index{dogtail.tree@\texttt{dogtail.tree}!függvények!application@\texttt{application}}
\index{dogtail.tree@\texttt{dogtail.tree}!függvények!application@\texttt{applications}}
A függvény mindössze egy paraméter vesz át, a keresett applikáció nevét. Ez a név jellemzően -- bár nem minden esetben -- megegyezik annak az applikáció indításához futtatott állomány nevével. Mint a későbbekben is csaknem mindig, a kísérletezés segít leginkább az applikáció felderítésében, a kereséshez használandó nevek meghatározásában. Ehhez egyrészről használhatjuk a korábban már említett \textit{Accerciser}t, ami voltaképpen egy grafikus felhasználói felület, ami az akadálymentesítéhez használt \textit{API} által megszerezhető információkat jeleníti meg struktúrált formában. Másrészről használhatjuk a \textit{Dogtail}t, a konkrét esetben a \texttt{Root} osztály \texttt{applications} függvényét, ami az összes -- az \textit{accessibilty} alrendszer számára látható -- applikációval tér vissza.

\index{dogtail.Config@\texttt{dogtail.Config}!tulajdonságok!searchCutoffCount@\texttt{searchCutoffCount}}
\index{dogtail.Config@\texttt{dogtail.Config}!tulajdonságok!searchBackoffDuration@\texttt{searchBackoffDuration}}
Az \texttt{application} függvény a \texttt{tree} modul \texttt{Application} osztályának egy, az applikációhoz tartozó példányával tér vissza, vagy ha az applikációt az újrapróbálkozásokat követően -- melyek számát a \texttt{Config} osztály \texttt{searchCutoffCount} tulajdonsága, míg a próbálkozások között tartandó szünet mértékét ugyanezen osztály \texttt{searchBackoffDuration} tulajdonsága határoz meg -- nem találja \texttt{SearchError} kivételt dob.

\subsubsection{Általános keresés}

\index{dogtail.Node\texttt{dogtail.Node}!függvények!findChild@\texttt{findChild}}
A \texttt{findChild} függvény első paramétere egy feltételhalmaz (\texttt{predicate}). Amennyiben a keresés során ennek a faltételhalmaznak megfelelő elemet találunk a közvetlen vagy közvetett leszármazottak között -- függően a \texttt{recursive} paraméter értékétől --, akkor a függvény az első ilyennel visszatér. Amennyiben nem talál megfelelő elemet és a \texttt{retry} paraméter értéke \texttt{True}, akkor újra próbálkozik a \texttt{config} objektumban foglalt beállításoknak megfelelően. Amennyiben a \texttt{retry} értéke \texttt{False} értelem szerűen összesen egy próbálkozás történik. Ha a \texttt{requireResult} paraméter értéke \texttt{True}, akkor \texttt{SearchError} kivételt dob, ha nem akkor egyszerűen \texttt{None} értékkel tér vissza. Megjegyzendő, hogy mindkét paraméter alapértelmezett értéke \texttt{True}, azaz a \textit{Dogtail} többször is próbálkozik és sikertelenség esetén \texttt{SearchError} kivételt dob.

\lstinputsource
[language=Python]
{sources/dogtail_predicate_generic.py}
{Node keresése \texttt{GenericPredicate}, illetve \texttt{findChild} függvény segítségével}
{lst:windowpersistentsimple}

\index{dogtail.Node\texttt{dogtail.Node}!függvények!findChild@\texttt{findChild}}
Ahogy látszik a \texttt{findChild} függvénynek átadott \texttt{GenericPredicate} objektum voltaképpen a keresési feltételeket zárja egy egységbe. Ezek a feltételek a név (\texttt{name}), a szerep neve (\texttt{roleName}) , leírás (\texttt{description}), illetve felirat (\texttt{label}). A működésről fontos megjegyezni, hogy a feltételhalmazban az utolsó feltétel (\texttt{label}) előnyt élvez a többivel szemben, vagyis amennyiben ezt feltételt megadjuk a másik három nem érvényesül. Ha viszont csak az első három paraméter használjuk azok egymással és kapcsolatba kerülnek, vagyis csak olyan elem lehet a keresés eredmény, ami minden feltételnek megfelel.

A feltételhalmazban megadott paraméterek értékeinek kísérleti úton meghatározását a már több alkalommal említett \textit{Accerciser} alkalmazás tud segíteni. Egyrészről megjeleníti az egyes applikációkat, azok ablakait, illetve a további gyerekelemeket fa hierarchiába szervezve. Másrészről az egyes -- amik mind egy-egy \texttt{Node} típusú objektumot jelentenek -- kapcsán megmutatja mik az objektum tulajdonsági, állapotai, a rajta végrehajtható akciók. Természetesen a későbbiekben az egyes \textit{widget}típusok ismertetésekor ezen értékekre mi is ki fogunk térni.

\index{GtkWidget@\texttt{GtkWidget}!függvények!get\_accessible@\texttt{get\_accessible}}
\index{AtkObject@\texttt{AtkObject}}
\index{AtkObject@\texttt{AtkObject}!függvények!set\_name@\texttt{set\_name}}
\index{dogtail.Node@\texttt{dogtail.Node}!tulajdonságok!roleName@\texttt{roleName}}
\index{dogtail.Node@\texttt{dogtail.Node}!tulajdonságok!name@\texttt{name}}
Az általános keresés szempontjából az imént említett néhány paraméter, illetve a \texttt{Node} osztály ezeknek megfelelő paraméterei érdemlegesek. A név attribútum (\texttt{name}) a \textit{widget} típusától -- amit a \texttt{role}, illetve annak szöveges formája a \texttt{roleName} reprezentál --, függően vesz fel értéket. Ablakok esetén annak címsorát, címkék esetén azok szövegét, olyan \textit{widget}ek esetén, amikhez címkét kapcsoltunk szintén a címke szövegét. A nevek felvett értékeinek részleteire, valamint a típusnevekre az egyes \textit{widget}típusok tárgyalásakor térünk. Maga a név explicit módon is beállítható a \textit{GTK}, pontosabban az \textit{ATK} révén, hiszen ez utóbbi interfészen keresztül történik az \textit{accessibilty} réteggel történő kapcsolattartás. Minden \texttt{GtkWidget} objektumhoz tartozik egy \texttt{AtkObject} objektum, amit a \texttt{get\_accessible} függvény segítségével kérhetünk. Az \texttt{AtkObject} a \texttt{set\_name} függvény révén adható olyan név, amire a \textit{Dogtail} használata során is hivatkozni tudunk.

\index{dogtail.Predicate@\texttt{dogtail.Predicate}}
\index{dogtail.Node@\texttt{dogtail.Node}!függvények!satisfies@\texttt{satisfies}}
\index{dogtail.Predicate@\texttt{dogtail.Predicate}!függvények!satisfiedByNode@\texttt{satisfiedByNode}}
A már említett \texttt{Predicate} osztálynak példányait felhasználhatjuk arra is, hogy csupán annyit tudjuk egy adott \texttt{Node} gyerekeként, ami megfelel a \textit{predicate} által megfogalmazott feltételhalmaznak. Ehhez a \texttt{Predicate} osztály \texttt{satisfiedByNode} függvényét kell hívnunk, arra azzal a \textit{Node} objektummal paraméterként, amire a vizsgálatot el szeretnénk végezni. Egy adott \texttt{Node} objektumról ugyanez a döntés a \texttt{satisfies} függvény hívásával hozható meg aminek viszont a \textit{predicate} a paramétere.

\subsubsection{Specializált keresés}

Létezik számos specializációja, leszármazottja a \texttt{Predicate} osztálynak, amik a gyakran használt keresések egyszerűsítésére szolgálnak. Ezek közül minden részben azokat vesszük sorra, amik az adott rész szempontjából érdekesek. Itt tehát az applikációk és ablakok keresésére szolgálok származtatásokat ismertetjük.

\paragraph{Applikáció}

\index{dogtail.Root\texttt{dogtail.Root}!függvények!application@\texttt{application}}
A \texttt{Root} osztály \texttt{application} függvénye az \texttt{Application} osztály a \texttt{tree} modul \texttt{Node} osztályából származik, ami gyakorlatilag minden olyan osztálynak őse, melyet egy keresés eredményeként visszakaphatunk (\texttt{Application}, \texttt{Root}, \texttt{Window}).

\lstinputsource
[language=Python]
{sources/dogtail_find_application.py}
{Applikáció keresése \texttt{application} függvénnyel, illetve \texttt{IsAnApplicationNamed} objektummal}
{lst:windowpersistentsimple}

\index{dogtail.Node\texttt{dogtail.Node}!függvények!findChild@\texttt{findChild}}
\index{dogtail.Predicate@\texttt{dogtail.Predicate}}
Ahogy látszik az \texttt{application} függvény nem más, mint egy specializáció, ami tulajdonképpen \texttt{Node} osztály \texttt{findChild} általános keresőfüggvényét hívja paraméterként egy \texttt{IsAnApplicationNamed} osztály egy objektumával, ami a \texttt{Predicate} osztályból származik és példányosításkor a keresendő applikáció nevét veszi át paraméterként.

\paragraph{Ablak}

Hasonlóan az applikáció kereséséhez az ablakok keresésére is létezik a \texttt{Predicate} osztályból származó saját osztály. Mind a \texttt{IsAWindowNamed}, mind a \texttt{IsADialogNamed} konstruálásához csak az ablak címsorát kell megadnunk. A két külön osztályra mindössze azért van szükség, mert a \texttt{GtkWindow} és a \texttt{GtkDialog} típusok a \textit{Dogtail} reprezentáció más \texttt{roleName} értékkel rendelkeznek (\texttt{frame}, \texttt{dialog}), ahogy az a egyezést vizsgáló függvénnyek implementációjából is látszik.

\lstinputsource
[language=Python]
{sources/dogtail_find_window.py}
{Ablak keresése specializált \texttt{Predicate} objektummal}
{lst:windowpersistentsimple}

\subsection{Státuszok}

\index{Atspi.Accessible\texttt{Atspi.Accessible}!függvények!getState@\texttt{getState}}
Az ablakok kapcsán -- függetlenül attól, hogy \texttt{GtkWindow}, vagy \texttt{GtkDialog} típusról van szó -- van néhány tulajdonság, amit ebben a részben a fejlesztési oldalról már tárgyaltunk és most a visszaellenőrzésükre is kitérünk. Az státuszok alapvetően két értékűek és mint ilyenek egy állapot halmazt alkotnak, ami egy adott \texttt{Node} objektumra egy adott pillanatban jellemző. Ez az állapot-, vagy státuszhalmaz a \texttt{getState} függvény segítségével kérdezhető le, majd ezt kell megvizsgálnunk, hogy a számunkra aktuálisan érdekes állapot része-e a halmaznak.

\lstinputsource
[language=Python]
{sources/dogtail_get_state.py}
{\texttt{Node} státuszvizsgálatához szükséges függvény sémája}
{lst:windowpersistentsimple}

Létezik néhány olyan -- a későbbiekben részletezett -- státusz, amikre a \texttt{Node} osztály implementálja a megfelelő lekérdező függvényt, azon esetekben azonban, ahol ez nem áll rendelkezésre magunknak kell a megoldásról gondoskodnunk. Az ablakok szempontjából az átméretezhetőség (\textit{resizable}), a modalitás (\textit{modal}), valamint az az állapot, hogy a vizsgált ablak aktuálisan az aktív ablak (\textit{active}) ablak-e az érdemleges állapotok.

\subsection{Interfészek}

A \textit{GTK} koncepcióját taglaló részben esett néhány szó róla, a \textit{GTK} akadálymentesítéhez szükséges implementációt az \textit{ATK} által definiált interfésznek megfelelően nyújtja. A \textit{Dogtail} voltaképpen egy \textit{Python} nyelven íródott absztrakció ezen réteg fölé, ami elrejti annak részleteit és egy magasabb szinten, a felhasználói felületek teszteléséhez megfelelő módon kezeli az akadálymentesítési réteg által nyújtott funkcionalitásokat.

\subsubsection{Komponens}

Az egyik \textit{ATK} által definiált interfész a \texttt{AtkComponent}, aminek segítségével az egyes objektumok pozíciójáról, illetve méretéről szerezhetünk információkat. A \textit{Dogtail} ezen interfész részleteit elrejti elölünk, kihasználva a \textit{Python} nyelv adottságait egy-egy tulajdonság (\textit{property}) kiolvasásával férhetünk hozzá az \texttt{AtkComponent} osztály által szolgáltatott értékekhez. Ezek az értékek a \texttt{Node} pozíciója (\texttt{position}), mérete (\texttt{size}), illetve ezek összességét jelentő kiterjedés (\texttt{extents}), ami mind az \textit{x}, \textit{y}, mind pedig a szélesség, magasság értékeket tartalmazza. Bár ezek az értékek minden \texttt{Node} esetén elérhetőek, az ablakokon kívül -- ahol ezen értékek révén ellenőrizni tudjuk az alapértelmezett méretet, illetve a szülő ablakhoz képesti pozíciót --, különösebb jelentőségük nincs.

\subsubsection{Akciók}

Vannak esetek, amikor sem nem információkat kinyerni, sem nem információkat bevinni nem akarunk a tesztelendő alkalmazásba, ehelyett inkább a mozgásban szeretnénk azt tartani, bizonyos akciók révén. Példának okáért ilyen lehet egy gomb megnyomása, ami tovább mozdítja a tesztelendő alkalmazást. A végrehajtható akciókat, illetve azok kezelését az \texttt{AtkAction} interfész fogja össze. Ezt az interfészt a \texttt{queryAction} függvény segítségével kérdezhető le.

Az interfész abban nyújt segítséget, hogy az \texttt{nActions} attribútumból az adott elemen keresztül kiolvashatjuk a végrehajtható akciók számát, bár ez következik az akciókat tartalmazó \texttt{dict} (\texttt{actions}) számosságából is, ami az akciók neveihez magukat az akciókat reprezentáló osztályok (\texttt{dogtail.tree.Action} példányait rendeli. Ezek tartalmazzák a az akció nevét (\texttt{name}), leírását (\texttt{description}) attribútumként, illetve egy paraméter nélküli függvényt (\texttt{do}), ami révén az akciót végrehajthatjuk. Ez utóbbi a \texttt{Node} osztály \texttt{doActionNamed} függvénye révén is megtehető, ha ismerjük az akció nevét, mivel ez a függvénynek átadandó egyetlen paraméter. A függvények visszatérési értéke, hogy az akciót sikerült-e végrehajtani.

A \texttt{GtButton} típus objektumaira, azaz jelen esetben a dialógusok kapcsán tárgyalt gombokra igaz, hogy végrehajtható rajtuk a \texttt{click} akció, ami a gombra való kattintás kiváltódását eredményezi. Ennek hasznát természetesen akkor látjuk, amikor egy dialógus ablak beviteli mezőinek kitöltésével végeztünk és szeretnénk a normál ügymenetnek megfelelően az \textit{Ok} gombot megnyomni, ekkor használhatjuk a \texttt{doActionNamed('click')}, vagy a \texttt{do('click')} függvényhívást függően attól, hogy a gombhoz tartozó \texttt{Node}, vagy annak \textit{action} interfésze áll rendelkezésünkre.

\subsection{Tulajdonságok}

Azon információk számára, amik sem a különböző \textit{ATK} által definiált interfészeken, sem az státuszok révén, sem más módokon nem érhetőek el, adott egy név érték párokat tartalmazó adatszerkezet. Az adatszerkezet lekérdezhető \texttt{dict} (\texttt{get\_attributes}), illetve \texttt{list} (\texttt{getAttributes}) formájában is, ahol a a lista elemei a nevek és az értékek szöveges változatainak kettősponttal (\texttt{:}) elválasztott értékei, míg a szótár értelemszerűen a neveket kulcsként, az értékeket a kulcsokhoz rendelt értékként tartalmazza.

Az attribútumlista minden \texttt{Node} esetén tartalmazza annak a grafikus eszközkészletnek nevét, aminek révén a tesztelt alkalmazást létrehozták. A \textit{GTK} esetén tehát a \texttt{toolkit} névhez nem meglepő módon a \texttt{gtk} érték párosul. Egyébiránt az eszközkészlet neve elérhető a \texttt{Node} \texttt{toolkitName} nevű tulajdonságán keresztül, ehhez hasonlóan az eszközkészlet verziója pedig a \texttt{toolkitVersion} tulajdonságon keresztül.


\chapter{Konténerek}
\section{Bevezetés}

Már a legegyszerűbb felhasználói felület láttán is könnyen belátható, hogy a korábbi példaprogramjaink kissé sántítanak, mégpedig abban a tekintetben, hogy nincs olyan valós életben is használt program amiben egy-egy \textit{widget} lenne csupán. Ha viszont több \textit{widget}et szeretnénk elhelyezni egy ablakban kézenfekvő kérdés, hogy miként tudnák őket a felületen csoportokba rendezni. Erre a kérdésre keressük a választ ebben a részben.

\section{Fogalmak}

A korábbiakhoz hasonlóan itt is érdemes először tisztázni néhány alapfogalmat és csak utána kezdeni bele az érdemi munkába.

\subsection{Konténerek}
\index{konténer}

A \textit{widget}ek a felületen történő csoportokba csoportokba rendezése konténerek (\textit{container}) segítségével valósul meg. Ezek olyan láthatatlan \textit{widget}ek, melyekbe más \textit{widget}eket helyezhetünk (\textit{pack}).

\index{GtkContainer@\texttt{GtkContainer}}
A \texttt{GtkContainer}, avagy \texttt{Gtk::Container} egy önmagában nem használható absztrakt osztály, mely csupán ősül szolgál minden olyan származtatott osztálynak, melyet widgetek tárolására lehet használni.

\index{GtkBin@\texttt{GtkBin}}
\index{GtkBox@\texttt{GtkBox}}
Alapvetően két ilyen leszármazott osztálytípussal találkozhatunk a későbbiekben. Ezek a \textit{bin} és \textit{box} ősosztályok, melyek maguk is absztraktak és abban különböznek egymástól, hogy hány elem tárolására alkalmasak. Előbbiek összesen egyére, míg az utóbbiak akárhányéra.

\subsubsection{Egy elemű konténerek}

\label{sec:bin}
\index{GtkBin@\texttt{GtkBin}}
A \texttt{GtkBin} jelentősége a további származtatásoknál jelentkezik majd, hisz az olyan nélkülözhetetlen típusoknak, mint az ablak (\texttt{GtkWindow}), a gomb (\texttt{GtkButton}), vagy a frame (\texttt{GtkFrame}) mind a \texttt{GtkBin} az ősosztálya.

\subsubsection{Több elemű konténerek}

\index{GtkBox@\texttt{GtkBox}}
\index{GtkHBox@\texttt{GtkHBox}}
\index{GtkVBox@\texttt{GtkVBox}}
\index{GtkTable@\texttt{GtkTable}}
A felületi elrendezés kialakításakor játszik fontos szerepet, hisz a benne található \textit{widget}ek -- amit a \textit{GTK} konténer gyerekeinek (\textit{children}) nevez -- elrendezésén túl azok méretét és konténeren belüli pozícióját is meghatározza. Ilyen típusok például a horizontális, vagy vertikális rendezést biztosító boxok (\texttt{GtkHBox}, \texttt{GtkVBox}), vagy a táblázatos megjelenítést szolgáló \texttt{GtkTable}.

\subsection{Méretezés}

\index{konténer!size request@\textit{size request}}
\label{par:widgetsizerequest}
A \texttt{GtkContainer} osztály legfontosabb funkcionalitása -- melyet minden származtatott osztály is felhasznál -- az, hogy meg tudja határozni a benne található elemek méretét. Ezt persze nem teljesen önállóan teszi, hanem megkérdezni a benne található \textit{widget}eket, hogy mekkora helyre lenne szükségük. Minden egyes \textit{widget} saját hatáskörben állapíthatja meg, hogy mekkora az a vízszintes, illetve függőleges kiterjedés, ami az igényeinek legjobban megfelelne. Ezt az méretigényt nevezi a \textit{GTK} \textit{size request}nek. Ez a mechanizmus fentről lefelé, azaz a gyökértől a levelek felé terjed abban a fa hierarchiában, melynek gyökere az ablak, közbülső elemeit a konténerek, leveleit pedig a widgetek alkotják.

\index{konténer!size allocation@\textit{size allocation}}
\label{par:widgetsizeallocation}
A legegyszerűbb és legáltalánosabb eset tehát az, hogy a konténerek -- amilyen maga az balak is -- összeadják a gyermekeik (a fában közvetlenül alattuk lévő elemek) méretigényét és azt sajátjukként propagálják. A valóság azonban nem ilyen demokratikus. Minden konténernek lehetősége van arra, hogy a benne lévő elemek felé -- egy az eredeti igénytől esetlegesen eltérő -- méretet (\textit{size allocation}) adjon vissza mellyel gazdálkodniuk kell és amely rájuk nézve kötelező érvényű.

\subsection{Elrendezés}

Normális esetben egy ablak rendelkezésre bocsájtja azt a területet, melyet a benne lévő \textit{widget}ek igényeltek. A kérdés nem is abban  áll, hogy mi legyen akkor ha pont annyi hely van amennyi kell, hanem sokkal inkább abban, miként jelenítse meg a konténer a saját \textit{widget}jeit, ha több a hely, mint amennyire feltétlenül szükség volna. Ilyen eset például akkor állhat elő, ha ez ablak átméretezhető és azt a felhasználó nagyobbra nyújtja, mint amekkora hely a benne lévő widgetek kirajzolásához minimálisan elégséges.

\index{GtkBox@\texttt{GtkBox}!gyerek tulajdonságok!expand@\texttt{expand}}
\index{GtkBox@\texttt{GtkBox}!gyerek tulajdonságok!fill@\texttt{fill}}
\index{GtkBox@\texttt{GtkBox}!gyerek tulajdonságok!pack-type@\texttt{pack-type}}
Ezt az esetet szabályozzák azok a paraméterek (pl: \textit{fill}, \textit{expand}, \textit{pack-type}, melyek tulajdonképpen sem a konténerhez, sem pedig a benne tárolt \textit{widget}hez nem tartoznak, hisz a kettejük viszonyát határozzák meg. Megadásuk akkor történik, amikor egy \textit{widget}et szeretnénk egy konténerben -- például egy \textit{box}ban -- elhelyezni.

A konténereket és ezen belül a \textit{box}okat leginkább úgy képzelhetjük el, mint egy dobozt -- vagy ha programozási szakszóval akarunk élni vermet -- melynek mindkét végére lehet pakolni. A verem hasonlat már csak azért is helytálló, mert az egyes elemek verembe történő elhelyezése csak egymást követően, csak egymásra lehetséges. Annyiban viszont sántít a példa, hogy a \textit{GTK} esetén rendkívül ritka -- bár egyáltalában nem lehetetlen -- hogy elemeket vegyünk ki egy konténerből.

\section{Alapműveletek}

Nyilvánvaló igény, hogy ha már vannak tárolóink és azokhoz kapcsolódóan \textit{widget}eink, akkor azokkal műveleteket lehessen végezni. Az sem túl meglepő, hogy ezt a különböző típusú konténerek esetén másként kell megtenni. Itt csak a legalapvetőbb műveleteket és típusokat vesszük sorra, azokat is csak abban a mértékben mely a koncepció megértéséhez szükséges.

\subsection{Létrehozás}

A \texttt{GtkContainer}, \texttt{GtkBin}, illetve a \texttt{GtkBox} absztrakt osztályok, így ebben a formájukban nem, csak a származtatott osztályok révén példányosíthatóak. Ezek közül ebben a részben a vízszintes, illetve függőleges elrendezésű \textit{box}okat (\texttt{GtkHBox}, \texttt{GtkVBox}), illetve a táblázatot (\texttt{GtkTable}) tárgyaljuk.

\subsubsection{\texttt{GtkVBox} és \texttt{GtkHBox}}

\index{GtkBox@\texttt{GtkBox}!tulajdonságok!homogeneous@\texttt{homogeneous}}
\index{GtkBox@\texttt{GtkBox}!tulajdonságok!spacing@\texttt{spacing}}
Függetlenül az iránytól a létrehozáskor két paramétert kell megadunk (\textit{homogeneous}, \textit{spacing}), ahol az egyik egyik egy \textit{bool}, melynek \textit{true} értéke esetén minden egyes elem azonos helyet foglal majd el a konténerben, míg a másik az egyes elemek között üresen hagyandó részt adja meg pixelben.

\subsubsection{\textit{Table}}

Létrehozáskor a táblázat sorainak, illetve oszlopainak kezdeti száma, valamint a korábbról ismert \textit{homogeneous} érték adandó meg. Míg a vízszintes elrendezésű \textit{box}oknál a \textit{widget}ek szélessége, a függőlegeseknél a magassága, addig a táblázatoknál mindkettő azonos ha a \textit{homogeneous} paraméter értéke \textit{true}.

\subsection{Elem hozzáadása}

Ezen függvények közös sajátossága, hogy paraméterként átveszik azt a \textit{widget}et, melyet a konténerbe kívánunk helyezni. A korábban említett fa hierarchiából következik, hogy egy elem nem lehet több szülőnek gyermeke (különben erdő szerkezetről beszélnénk), azaz egy \textit{widget}et összesen egy konténerben helyezhetünk el. Ha esetleg ezt másodszor is megpróbálnánk -- még mielőtt a korábbi konténeréből eltávolítottuk volna -- akkor futás idejű hibaüzenetet kapunk.

\index{referencia számlálás}
Ne feledjük, ahogy arról már említést tettünk, a \textit{GTK} rendelkezik referenciaszámlálási metódussal, azaz minden egyes objektum (\texttt{GtkObject}) -- esetünkben \texttt{GtkWidget} -- rendelkezik egy referenciaszámmal. Ha egy \textit{widget}et egy konténerbe helyezünk, annak referenciáját a konténer, annak rendje és módja szerint, növeli eggyel. Ez a referencia mindaddig megmarad, míg a \textit{widget}et el nem távolítjuk, vagy a konténer valamilyen oknál fogva meg nem szűnik, ami jellemzően akkor következik be ha az egész ablakot megszüntetjük (\textit{destroy}).

\subsubsection{\textit{Container}}

Az \texttt{add} nevű függvény egyetlen paramétert, az elhelyezni kívánt \textit{widget}et veszi át. Ritkán, leginkább csak egyszerű konténereknél alkalmazott hívás, lévén olyan alapértelmezett paraméterek használ a \textit{widget} elhelyezésére, melyek a felhasználó céljainak a legtöbb esetben nem felelnek meg.

Használható ugyan a származtatott, bonyolultabb konténerek esetén is (pl: \texttt{GtkBox}, \texttt{GtkTable}), de célszerűbb ilyen esetekben az azokhoz tartozó, specifikus függvényt alkalmazni, lévén az sokkal rugalmasabban paraméterezhetőek.

\subsubsection{\textit{Bin}}

Ebbe a típusba elemet csak a \texttt{GtkContainer} \texttt{add} függvényével tehetünk. Ha többször hívjuk meg a függvényt anélkül, hogy a \texttt{GtkBin} gyerekét eltávolítanánk futási hibát kapunk, hiszen a \texttt{GtkBin} csak egyetlen elem tárolására képes.

\subsubsection{\textit{Box}}

Ahogy arról a bevezetőben szó volt a \texttt{GtkBox} típus olyan, mint egy két bemenetű verem. Ennek megfelelően két függvény van, amivel elemeket (egyet, vagy többet) lehet helyezni a konténerbe. A \texttt{pack\_start} függőleges elrendezés (\texttt{GtkVBox}) esetén felülről lefelé haladva tölti meg a konténert úgy, hogy az egymást után elhelyezett elemek egymás alatt jelennek meg, míg a vízszintes elrendezést biztosító változat (\texttt{GtkHBox}) balról jobbra haladva teszi ugyanezt. A \texttt{pack\_end} hívás ezekkel ellentétesen működik, tehát alulról, illetve jobbról balra haladva helyez elemeket a tárolóba.

\index{GtkBox@\texttt{GtkBox}!gyerek tulajdonságok!expand@\texttt{expand}}
\index{GtkBox@\texttt{GtkBox}!gyerek tulajdonságok!fill@\texttt{fill}}
\index{GtkBox@\texttt{GtkBox}!gyerek tulajdonságok!padding@\texttt{padding}}
Ahogy a \texttt{GtkContainer} esetén, itt is megadandó a \textit{widget}, de ezen túl itt a konténeren belüli elhelyezkedést meghatározó értékek is paramméterek. Az \textit{expand} és \textit{fill} \textit{bool} típusú paraméterek, melyek a korábban már említett ''felesleges'' hely kitöltésére vonatkoznak. Előbbi azt határozza meg, hogy a \textit{widget} a konténeren belül rendelkezésre álló helyet kitöltse-e, azaz ha van ilyen akkor az igényelje-e magának (\textit{ture}), vagy lemondjon róla (\textit{false}) a többi -- a konténerben lévő -- \textit{widget} javára. Utóbbi paraméter annak beállítására szolgál, hogy a rendelkezésre álló -- illetve az \textit{expand} okán elnyert -- hellyel mi történjék. Ha a paraméter értéke \textit{true}, akkor a \textit{widget} maga tölti ki ezt a helyet, azaz a feltétlenül szükségesnél nagyobb helyen rajzolódik ki, míg ha az érték \textit{false}, akkor csak a minimálisan szükséges helyre rajzolódik és a maradék részt úgymond üresen hagyja. A \textit{pack} függvények utolsó paramétere a \textit{padding}, mely a \textit{widget} körül (\texttt{GtkVBox} esetén felül és alul, \texttt{GtkHBox} esetén jobbról és balról) hagyandó üres hely értékét adja meg pixelben.

Ha elsőre nem is teljesen világos, mit jelent ez a gyakorlatban, a következő fejezet illusztrációjából minden világossá válik.

\subsubsection{\textit{Table}}

Az \textit{attach} függvény -- mely a táblázatok esetén elem elhelyezésére szolgál -- kissé összetettebb, mint a korábbiak, lévén egy táblázatnál vízszintesen és függőlegesen egyaránt szükséges megadni a \textit{fill}, illetve \textit{expand} paramétereket, melyek kiegészülnek egy \textit{srink} opcióval is, ez azonban már túlmutat ennek a résznek a keretein.

\subsection{Elem eltávolítása}

A származtatott típusok, legalábbis azok, amelyekkel ebben a részben foglalkozunk (\texttt{GtkTable}, \texttt{GtkBin}, \texttt{GtkBox}) nem igényelnek az eltávolítás során semmilyen extra műveletet, így a \texttt{GtkContainer} funkcionalitására támaszkodnak.

\subsubsection{\textit{Container}}

\index{referencia számlálás}
A \texttt{remove} függvény értelemszerűen az eltávolítani kívánt \textit{widget}et várja paraméterként és ahogy azt említettük az általa tartott referenciát meg is szünteti. Ez jellemzően (a referenciaszámlálás \textit{GTK}-beli sajátosságairól egy késöbbi részben szólunk) azt is jeleni egyben, hogy az eltávolított \textit{widget}re az utolsó (hisz többnyire csak a konténere tart referenciát egy \textit{widgetre}) referencia és ezzel maga a \textit{widget} is megszűnik.

Ha ezt az esetet el akarjuk kerülni, akkor még az eltávolítás előtt a referenciaszám növeléséről magunknak kell gondoskodnunk. Vagyis, ha két lépésben (\texttt{remove} és \texttt{add}) akarunk egy \textit{widget}et áthelyezni egyik konténerből a másikba, akkor az eltávolítás előtt növelnünk, a hozzáadás után pedig csökkentenünk kell a referenciát. Utóbbira azért van szükség, mert az új szülőelem maga is növel egyet a referencián, így ha mi, a magunk által korábban megnövelt referenciát nem csökkentenénk, a widget soha nem szűnne meg.

Hasonlóan a \texttt{GtkBin}hez itt sincs specifikus függvény az eltávolításra, hanem a \texttt{GtkContainer} \texttt{remove} függvényét hívjuk.

\section{Pa(c)koljunk}
\label{sec:packing}

Lássuk mire jó végül is ez a három opció (\textit{homogeneous}, \textit{expand}, \textit{fill}) és mikét függenek össze egymással.

Azoknak, akik már jártasak valamilyen -- a \textit{GTK+}-tól eltérő -- felületprogramozási nyelvben bizonyosan találkoztak már olyan eszközökkel (pl: \textit{QtDesigner}, vagy jelen esetben a \textit{Glade}), melyek egy konkrét felület elkészítésére alkalmasak. A koncepciók különbözőek ugyan, de mindegyiknél joggal merülhet fel a kérdés, hogy miért is nem lehet egész egyszerűen fix koordinátákkal megadni, hol legyenek az egyes widgetek. Nos lehet, sőt vannak is ilyen eszközök, ugyanakkor az elsőre talán kissé zavaró egyveleg komoly flexibilitást nyújt.

\subsection{Homogenitás}
\index{GtkBox@\texttt{GtkBox}!tulajdonságok!homogeneous@\texttt{homogeneous}}
\index{GtkBox@\texttt{GtkBox}!gyerek tulajdonságok!fill@\texttt{fill}}
\index{GtkBox@\texttt{GtkBox}!gyerek tulajdonságok!expand@\texttt{expand}}

Azaz egyenlőség, abban az értelemben, hogy minden egyes elem a konténerben pontosan ugyanakkora helyet foglal el. A következő ábra azt szemlélteti, hogy miként változtatja meg ez az érték -- a másik kettő függvényében (\textit{expand}, \textit{fill}) -- az elemek elhelyezkedését a konténeren belül.

\vspace{12 pt}
\begin{figure}[H]
\begin{center}
\includegraphics[height=50mm]{images/packbox1.png}
\end{center}
\end{figure}

Ahogy korábban a kódsorokat, most a \textit{widget}sorokat vesszük sorra a minél jobb megértés kedvéért.

\subsubsection{Méretarányos elhelyezés}

\begin{description}
 \item[\textit{expand} = \textit{false}, \textit{fill} = \textit{false}] A konténerben lévő elemek -- ahogy fentiekben fogalmaztunk -- nem akarnak egymás rovására helyet szerezni (\textit{epxand}), így a rendelkezésre álló vízszintes helyet nem is töltik ki, vagyis ezzel a megoldással az egész elemsorra nézve egyfajta balra (\textit{pack\_end} esetén jobbra) zártság alakítható ki.

\index{konténer!size allocation@\textit{size allocation}}
 \item[\textit{expand} = \textit{true}, \textit{fill} = \textit{false}] Az összkép -- az alatta található sor miatt -- kissé csalóka, mivel az elemek kissé rendezetlennek tűnnek, ugyanakkor arról van szó, hogy minden egyes elem megszerezte magénak -- a saját eredeti méretigényének arányában -- a rendelkezésre álló plusz helyet és az így allokált (\textit{size allocation}) térrészen belül középen helyezkedik el.

 \item[\textit{expand} = \textit{true}, \textit{fill} = \textit{true}] Az egyes elemek nem csak hogy kiterjeszkedtek (\textit{expand}) a korábban fel nem használt területre, de ki is töltik (\textit{fill}) azt, azaz annak terjedelmében rajzolják meg magukat.
\end{description}

Ahogy az az ábrából -- és talán a magyarázatból is -- kitűnik az \textit{fill} opció állításának semmi teteje anélkül, hogy az \textit{expand} be ne lenne kapcsolva, hisz e nélkül nincs semmilyen plusz terült, amire a \textit{widget} magát megnagyobbítva rajzolhatná.

\subsubsection{Homogén elhelyezés}
\index{GtkBox@\texttt{GtkBox}!tulajdonságok!homogeneous@\texttt{homogeneous}}
\index{GtkBox@\texttt{GtkBox}!gyerek tulajdonságok!fill@\texttt{fill}}
\index{GtkBox@\texttt{GtkBox}!gyerek tulajdonságok!expand@\texttt{expand}}

\begin{description}
 \item[\textit{expand} = \textit{true}, \textit{fill} = \textit{false}] A konténerben lévő elemek -- a már használt kifejezéssel élve -- nem akarnak egymás rovására helyet szerezni (\textit{epxand}), így a rendelkezésre álló vízszintes helyet nem is töltik ki, vagyis ezzel a megoldással az egész elemsorra nézve egyfajta balra (\textit{pack\_end} esetén jobbra) zártság alakítható ki.

\index{konténer!size allocation@\textit{size allocation}}
 \item[\textit{expand} = \textit{true}, \textit{fill} = \textit{true}] Az összkép -- az alatta található sor miatt -- kissé csalóka, mivel az elemek kissé rendezetlennek tűnnek, ugyanakkor arról van szó, hogy minden egyes elem megszerezte magénak -- a saját eredeti méretigényének arányában -- a rendelkezésre álló plusz helyet és az így allokált (\textit{size allocation}) térrészen belül középen helyezkedik el.
\end{description}

\subsection{Térköz}
\index{GtkBox@\texttt{GtkBox}!tulajdonságok!spacing@\texttt{spacing}}
\index{GtkBox@\texttt{GtkBox}!gyerek tulajdonságok!padding@\texttt{padding}}

Térköz megadására két lehetőség is kínálkozik a ha elemeket szeretnénk elhelyezni egy \textit{box}ban. Az egyik, hogy magának a konténernek állítunk be létrehozáskor -- vagy akár később -- \textit{spacing}et, vagy az egyes elemek hozzáadásakor adunk meg \textit{padding}et. Hogy mi a különbség a két eset között az alábbi ábra -- ahol az fenti blokkban az első, míg az alsó blokkban a második esetre látunk példát -- jól illusztrálja.

\vspace{12 pt}
\begin{figure}[H]
\begin{center}
\includegraphics[height=50mm]{images/packbox2.png}
\end{center}
\end{figure}

\subsubsection{Tér az elemek között}
\index{GtkBox@\texttt{GtkBox}!tulajdonságok!spacing@\texttt{spacing}}

\begin{description}
 \item[\textit{expand} = \textit{true}, \textit{fill} = \textit{false}] Ez a példa nem mutatja igazán jól meg azt, hogy az elemek között jelenik meg az a térköz, amit a konténer létrehozásakor megadtunk.

 \item[\textit{expand} = \textit{true}, \textit{fill} = \textit{true}] Mivel itt mindkét érték \textit{true}, a \textit{widget}ek a rendelkezésre álló teret teljes egészében kihasználják maguk megrajzolására, eltekintve természetesen a közöttük megjelenő 10 pixel \textit{spacing}től. Érdemes külön megfigyelni a két szélső elemet, azoknak is az ablak széléhez közelebb eső részét a másik megoldással való összehasonlításhoz.
\end{description}

\subsubsection{Tér az elemek körül}
\index{GtkBox@\texttt{GtkBox}!gyerek tulajdonságok!padding@\texttt{padding}}

\begin{description}
 \item[\textit{expand} = \textit{true}, \textit{fill} = \textit{false}] A \textit{padding} megadásával a térköz nem az elemek között, hanem azok körül jelenik meg. Ez azt jelenti, hogy minden elem jobb és bal oldalán (\texttt{GtkVBox} esetén felül és alul) egyaránt jelentkezik a megadott térköz, ennek okán közöttük annak minimum (függően a \textit{fill} értékétől) a kétszerese.

 \item[\textit{expand} = \textit{true}, \textit{fill} = \textit{true}] Ez az az eset amikor igazán jól látható a \textit{widget}ek között és az azok mellett megjelenő térköz 2:1 aránya. Az előbb -- a szélső widgetek elhelyezkedésénél megfigyelteket -- most hasznosíthatjuk, ha észrevesszük itt a szélső \textit{widget}ek nem tudnak a konténer széléig kiterjeszkedni, lévén két oldalról ki vannak párnázva (\textit{pad}) 10-10 pixellel.
\end{description}

\section{A kód}

A fenti példaprogramok forrása, illetve azok eredetijei, a \textit{FLOSSzine}, valamint a \textit{GTK+} oldalain az alábbi linkeken érhetőek el:
\ \\\\
\url{http://www.flosszine.org/sources/gtk_packbox.c}\\
\url{http://library.gnome.org/devel/gtk-tutorial/2.17/x387.html}

\subsection{Fordítás és linkelés}

A korábbiakhoz hasonlóan az alábbi parancssorok segítségével fordíthatóak elemzett programjaink:

\lstccompile{gtk_packbox.c}{gtk_packbox}

\subsection{Futtatás}

Próbáljuk ezúttal a \texttt{./gtk\_packbox 1|2|3}, illetve a \texttt{./gtkmm\_packbox 1|2|3} parancsokkal abban a könyvtárban, ahol a fordítást elkövettük, ahol a paraméter a teszt sorszáma, abban a sorrendben, ahogy azokat itt is ismertettük (a 3. természetesen ráadás).

\subsection{Eredmény}

Ha netán úgy érezzük mégsem világos mi is történik, mikor és miért a konténerekbe pakolás kapcsán ne adjuk fel. Elsőre talán az egész mechanizmus jelentősége sem szembetűnő, ugyanakkor érdemes próbálkozni, azaz venni a forrást és játszani a különböző értékekkel (\textit{fill}, \textit{expand}, \textit{spacing}, \textit{padding}), illetve a létrehozott ablak átméretezésével.


\chapter{Megjelenítő eszközök}
sA szöveges adatbevitel legegyszerűbb módjairól ugyan volt szó az előző részben a kép azonban közel sem teljes, hiszen ablakok rendszerint nem csupán beviteli mezőkből állnak, hiszen magukból a mezőkből igencsak nehezen lehetne rájönni arra, hogy voltaképpen a mezőkbe mit is kellene írnunk. Ebben a részben néhány nélkülözhetetlen, teljesen általánosan használt megjelenítő \textit{widget}et veszünk górcső alá, úgy is mint a címkék, képek, súgó-, vagy leíróbuborékok.

\section{Fogalmak}

\subsection{Igazítás és helykitöltés}

A \textit{GTK+} definiál egy olyan ősként szolgáló osztályt (\texttt{GtkMisc}), ami önmagában nem példányosítható, csupán a belőle származó osztályok (\texttt{GtkLabel}, \texttt{GtkImage}, \texttt{GtkArrow}) közös beállításait fogja össze. Mindössze két tulajdonságról -- igazítás (\textit{align}), helykitöltés (\textit{padding}) van szó tulajdonképpen, amik a \textit{widget} tartalmának elhelyezkedését befolyásolják magán a \textit{widget}en belül, azaz függetlenül a \textit{widget} konténerben elfoglalt helyétől és annak paramétereitől. Ez talán némiképp ködösen hangzik, de a származtatott osztályok ismeretében hamar világossá válik.

Fontos már itt megjegyezni, hogy a \texttt{GtkMisc} némiképp idejétmúlt, szolgáltatásai egyszerűsített -- ugyanakkor a gyakorlati alkalmazás szempontjából mégis ugyanannyira kielégítő -- formában a \texttt{GtkWidget} típus által is implementáltak, így ezen típus tulajdonságainak használata újólag írt kódokban ellenjavallt. Két oknál fogva mégis célszerű ezzel a típussal foglalkozni. Az egyik a kézenfekvő ok, hogy a \texttt{GtkMisc} még mindig része a \textit{GTK+} függvénykönyvtárnak és meglehetősen régen az, ennek okán pedig számos korábbi kódban találkozhatunk vele.

\subsection{\textit{Widget}ek}

\subsubsection{Címkék}
\index{GtkLabel@\texttt{GtkLabel}}

Talán a legfontosabb és leggyakrabban használt kiegészítő \textit{widget}ek a címkék, ahol értelemszerű az igazítás (alignment) jelentése,ami azonos azzal, amit a szövegszerkesztő szoftverek esetén megszokhattunk. A helykitöltés (padding) működése a konténereknél már tárgyaltakkal egyezik meg.

\includesinglegraphics
{Címke}
{label}
{label.png}

\subsubsection{Képek}
\index{GtkImage@\texttt{GtkImage}}

A \textit{GTK} -- mint minden más felhasználó felület fejlesztéséhez használt eszközkészlet -- lehetővé teszi képek megjelenítését a felhasználó felületek részeként. Az igazítás és a helykitöltés funkciója ebben az esetben is értelemszerű.

\includesinglegraphics
{Kép}
{image}
{image.png}

Képek természetesen több forrásból is származhatnak. A felhasználói felületen természetesen megjeleníthetünk külső forrásból származó képeket, ikonokat, animációkat, melyeket fájlból tölthetünk be, ugyanakkor a \textit{GTK} maga is szállít számos olyan ikont, ami nehezen nélkülözhető még a legegyszerűbb felületeken sem. Ilyenek például a leggyakrabban előforduló gombok ( Ok, Mégsem, Alkalmaz, \dots ), az üzenetablakok ( hiba, figyelmeztető, információs, \dots ) ikonjai.

\paragraph{Beépített ikonok}

Ezek a beépített (\textit{stock}) ikonok a széles körben használt menüelemek, illetve eszközkészletek ikonjait jelentik, amikre azonosítók segítségével (\textit{id}) hivatkozhatunk képek, gombok, dialógusok létrehozásánál. A beépített ionokat sajátjainkra lecserélhetjük, illetve saját \textit{stock} ikonok regisztrálására is lehetőség van.

\paragraph{Ikonhalmazok}

Egy adott azonosítóhoz tartozó ikon méretbeli (menünek, gombnak, dialógusnak, \dots megfelelő méret), illetve a \textit{widget}ek lehetséges állapotainak (normál, kiválasztott, aktív,  \dots ) megfelelő variációk ikonhalmazokat hoznak létre, melyekben az egyes elemek a beépített ikonokhoz hasonlóan cserélhetőek.

\subsubsection{Buborékok}
\index{GtkTooltip@\texttt{GtkTooltip}}

Némiképp méltatlanul hanyagolt \textit{widget} a súgó-, vagy más néven leíróbuborék (tooltip), pedig egy magára valamit is adó applikáció nem nélkülözheti ezt az eszközt, lévén ez a felhasználó informálásának egyik leginkább bevett módszere.

\includesinglegraphics
{Buborék}
{tooltip}
{tooltip.png}

\subsection{Szövegformázás}

A \textit{GTK}, pontosabban szólva a bevezető részben már említett \textit{Pango}, rendelkezik egy saját leíró (markup) nyelvvel, a szövegek formázását teszi lehetővé a felhasználó felületen. Ezen nyelv segítségével állíthatjuk be a szöveg megváltoztatni kívánt paramétereit, úgy mint betű típusa, mérete, stílusa, színe és így tovább. Ezen tulajdonságok leírását magával a szöveggel együtt adjuk meg.

\lstoneline
{language=xml}
{<span foreground="blue" size="100">Blue text</span> is <i>cool</i>!}

A példában először a \texttt{Blue text} szöveg színét, illetve méretét állítjuk át igényeknek megfelelően, úgy hogy a kívánt szöveg köré az egyes tulajdonságok (\texttt{foreground}, \texttt{size}), illetve azok értékeinek leírását helyezzük el. Ezzel a módszerrel ez egyes tulajdonságokat külön-külön adhatjuk meg, ugyanakkor a gyakran, és jellemzően egymagukban használt beállításokhoz (félkövér, dőlt, aláhúzott) léteznek önálló leírók is (\texttt{b}, \texttt{i}, \texttt{u}), ahogy ez a \texttt{cool} szövegrésznél is látszik.

\subsection{\textit{Widget}ek összefüggései}

Bizonyos típusú \textit{widget}ek sajátja, hogy nem önmagukban léteznek, hanem vagy valamilyen csoportnak tagjai, vagy egy konkrét \textit{widget}tel állnak valamilyen összefüggésben. Ez utóbbi igaz a címkék esetén is, amiknél megadható, hogy melyik másik \textit{widget} az, amire vonatkoznak, amit leírnak. Ez az összefüggés egyes \textit{widget}típusok esetén (pl: gombok) esetén automatikus, míg más esetekben (pl: beviteli mezők, listák, \dots ) magunknak kell megadnunk. Ezen kapcsolat megadásának közvetlen előnye egyrészről a tesztelés során mutatkozik meg, ahol a címke segít a leírt \textit{widget} megtalálásában, másrészről a felhasználó felület billentyűzetről történő használatát könnyíti meg, ahogy arról a későbbiekben szó esik.

\section{Alapműveletek}

\subsection{Létrehozás}

Az ebben a fejezetben tárgyalt \textit{widget}ek esetén -- hasonlóan a korábbiakhoz -- a létrehozás maga nem különösebben bonyolult feladat. Némi rutint csak a létrehozást követő testreszabás igényel, amihez szükséges az alapvető működési sajátosságok tisztázása. A kezdeti beállításokat követően jellemzően ezen \textit{widget}ek nemigen változnak, úgyhogy ezt követően már csak a \textit{widget} konténerbe való behelyezése jelenthet kihívást.

\subsubsection{\texttt{GtkLabel}}

\index{GtkLabel@\texttt{GtkLabel}!függvények!new@\texttt{new}} 
A legegyszerűbb eset, ha szeretnénk valamilyen statikus szöveget, mindenféle formázás nélkül megjeleníteni a felhasználói felületen. Erre a problémára a megoldás is rendkívül egyszerű. Mindhárom nyelvi változatban esetén csupán egyetlen paramétert kell megadnunk a címkét létrehozó függvénynek, ez pedig a címke szövege.

\lsttriplesource
[numbers=none]
{sources/label_create.h}
{sources/label_create.hpp}
{sources/label_create.py}
{Címke létrehozása}
{lst:labelcreate}

\index{GtkLabel@\texttt{GtkLabel}!függvények!new@\texttt{new}} 
Ahogy látszik a különböző nyelvi változatoknál a címkék már létrehozáskor ennél változatosabban paraméterezhető. A létrehozás követően pedig további nyilván paraméterek is állíthatóak.  Későbbiekben meglátjuk mik ezek a paraméterek és milyen haszonnal bírnak a hétköznapi használat során, most vegyük sorra a létrehozás paramétereit.

\begin{description}
  \item[label] A címke szövege. Általánosságban véve a címkék szövegeinek megalkotásánál célszerű valamilyen egységes irányelvet követni. A \textit{GNOME} által követett irányelvek szerint a \texttt{GtkLabel} típus felhasználási területein -- legyen az a beviteli mezők címkéi, jelölőnégyzetek szövegei, vagy más egyéb -- a mondatok első szava írandó nagybetűvel, míg a többi kisbetűs.
  \index{GtkLabel@\texttt{GtkLabel}!függvények!new\_with\_mnemonic@\texttt{new\_with\_mnemonic}} 
  \item[mnemonic] A címkék leggyakoribb felhasználása a beviteli mezők illetve jelölőnégyzetek címkézése. Ezen esetekben a címkék szövegében megjelölhetünk egy karakter úgy, hogy egy aláhúzást karakter írunk elé a szövegben (pl: \texttt{"\_Label text"}), és azt a karaktert később a címke által hivatkozott \textit{widget} elérésére használhatjuk. A gyakorlatban ez a billentyűzetről történő navigációt -- ezzel együtt a felhasználói felület hatékonyabb használatát -- segíti elő azáltal, hogy a \textit{widget}, az előbb említett példánál maradva, az \texttt{Alt+L} billentyűvel aktiválható lesz.
  
  A címke által hivatkozott \textit{widget} alapértelmezés szerint az a \textit{widget} -- pontosabban az a konténer -- lesz amibe a címkét helyeztük, vagy annak első olyan szülője, ami implementálja a \texttt{GtkActivatable} interfészt. Az olyan \textit{widget}ek, amik maguk is tartalmaznak címkét (pl: gombok, jelölőnégyzetek, \dots ) ez a feltétel kézenfekvően adott, mivel a \textit{widget} implementálja a \texttt{GtkActivatable} interfészt. Amennyiben nem ez a helyzet, hanem például egy beviteli mezőt címkézünk, akkor magunknak kell megadnunk a kapcsolódó \textit{widget}et a \texttt{set\_mnemonic\_widget} függvény segítségével, aminek értelemszerűen az aktiválandó \textit{widget} a paramétere. Az aktiváláskor a beállított \textit{widget} az aktiválás hatására fókuszba kerül, ami egy beviteli mezőnél például azzal az előnnyel jár, hogy a nem kell váltogatnunk az egér és a billentyűzet között a felület használatakor.
  \index{GtkMisc@\texttt{GtkMisc}!tulajdonságok!xalign@\texttt{xalign}}
  \index{GtkMisc@\texttt{GtkMisc}!tulajdonságok!yalign@\texttt{yalign}}
  \item[xalign, yalign] A vízszintes, illetve függőleges igazítás 0 és 1 közötti lebegőpontos értékekkel adhatóak meg, ahol a 0 a bal oldalt, illetve a felső pozíciót, az 1 pedig a jobb oldalt, illetve az alsó pozíciót jelenti. Mind a vízszintes, mind a függőleges igazítás alapértelmezett értéke 0,5. Ez az esetek jelentékeny részében nem felel meg az igényeknek, hiszen a címkéket többnyire balra, esetenként jobbra igazítjuk, vagyis az \texttt{xalign} tulajdonságot értéke 0, illetve 1 kell legyen. Bár a \texttt{GtkMisc} osztály, illetve annak tulajdonságai még érvényben vannak, a \textit{GTK} fejlesztői nem ajánlják használatukat újólag írt kódban, lévén a \texttt{GtkWidget} típus korábban már említett \texttt{halign}, valamint \texttt{valign} tulajdonságai helyettesítik őket.
  \index{GtkMisc@\texttt{GtkMisc}!tulajdonságok!xpad@\texttt{xpad}}
  \index{GtkMisc@\texttt{GtkMisc}!tulajdonságok!ypad@\texttt{ypad}}
  \item[xpad, ypad] Az itt megadott értéknek megfelelően a vízszintesen, illetve függőlegesen ad térközt a \textit{widget} köré. Hasonlóan azonban az előbbiekhez ennek a módszernek a használata sem javasolt új kódokban, helyette a \texttt{GtkWidget} \texttt{margin} tulajdonsága állítandó.
\end{description}

\subsubsection{\texttt{GtkImage}}

Ahogy arról szó esett a bevezetőben, képek létrehozására számos mód kínálkozik. Ezek közül most csak a népszerűbbeket vesszük számba. Az első és talán legfontosabb a fájlból való betöltés, amire a \texttt{new\_from\_file} függvényt használhatjuk. Amennyiben a megadott fájl betöltése valamilyen oknál fogva sikertelen (pl: fájl nem létezik, jogosultsági problémák, \dots ), akkor egy olyan képet kapunk vissza, ami a betöltési hibára utal.

Amennyiben a fájl betöltése során felmerülő hibákat magunk szeretnénk kezelni egy alacsonyabb szintű megoldást kell választanunk, ami egyébiránt a \textit{GTK} fájlból való betöltés végző függvény hátterében is áll. Ezt a megoldás nem meglepő módon a \textit{GDK} adja, hiszen a kifejezetten grafikai kódok itt kapnak helyet. A \texttt{GdkPixbuf} típus szintén rendelkezik fájlból való betöltésre alkalmas függvénnyel (\texttt{new\_from\_file}), ami a \texttt{GtkImage} hasonló függvénnyel szemben hiba esetén a nyelvi változatnak megfelelő hibajelzés történik. A \textit{C} nyelvű változat esetén a hibát egy \texttt{GError} típusú változóban kaphatjuk vissza, míg a \texttt{C++}, illetve \texttt{Python} változat esetén kivél váltódik ki.

%TODO: code sniplet of error handling

A harmadik eset amikor egy beépített ikont szeretnénk képként használni, amit megtehetünk a \texttt{new\_from\_stock} függvény használva, ami paraméterként a beépített ikon (\textit{stock icon}) nevét veszi át paraméterként, éppúgy, ahogy teszi azt a gomb létrehozáskor.

\subsubsection{\texttt{GtkTooltip}}

A súgóbuborék funkciója és megadása is hasonlít némiképp címkééhez, hiszen mindkét \textit{widget} egy másik \textit{widget} azonosítására, szerepének tisztázására szolgál. Mindkét \textit{widget} valamilyen szöveges leírást ad hozzá tartozó felületi elemről, a címke rövidebb, míg a súgóbuborék rendszerint hosszabb formában. Ennek okán a súgóbuborék megadásának legegyszerűbb módja azonos a címke létrehozásánál leírtakkal, vagyis csupán a kívánt szöveget kell megadnunk paraméterként. Mivel a súgóbuborék csak konkrét \textit{widget}hez tartozhat, a függvény a \texttt{GtkWidget} típus függvénye (\texttt{set\_tooltip\_text}).

\subsection{Megjelenítés}

\subsubsection{\texttt{GtkLabel}}

\paragraph{Formázás}

A címkék szövegének formázására a bevezetőben említett \textit{Pango Markup Language} elnevezésű leírónyelv használható. A címkék létrehozása úgy történik, hogy a megadott szöveget alapértelmezetten nem tekintjük leíró nyelven megfogalmazottnak, azaz a \texttt{use-markup} tulajdonság értéke \texttt{FALSE}. Ez azonban a névkonvenciónak megfelelően a \texttt{set\_use\_markup} függvénnyel megváltoztatható. Ezzel azonban kellő óvatossággal kell bánnunk. Amennyiben a \texttt{use-markup} tulajdonság értéke \texttt{TRUE}, a címke szövege meg kell feleljen leíró nyelv szabályainak.

Ez két esetben kritikus. Az egyik, ha a szöveg olyan elemeket tartalmaz, amik a leíró nyelvben is értelmezettek. Ebben az esetben az ilyen elemeket úgy kell megváltoztatnunk (escape), hogy leíró nyelvnek megfeleljen, ugyanakkor a jelentése ne változzon. A \textit{Glib} \texttt{markup\_escape\_text} függvénye pontosan a leírtakat implementálja. Ezen függvényt minden olyan esetben használandó, amikor nem statikus szövegről van szó, hanem a szöveg például felhasználótól származik. Ilyen lehet például egy korábban bekért név, ami tartalmazhat például kisebb, vagy nagyobb jelet. A másik kritikusnak mondható eset, ha a leíró nyelvű szöveget egy \texttt{printf} típusú formátumleíróval akarjuk létrehozni. Ehhez a szintén a \textit{Glib} részeként elérhető \texttt{markup\_printf\_escaped} függvény nyújt segítséget.

\paragraph{Tördelés}

Hosszabb szövegű címkék használata esetén -- ami jellemzően akkor fordul elő, ha egy magyarázó szöveget akarunk például egy üzenetablakban megjeleníteni -- célszerű tördelnünk a szöveget elkerülendő, hogy a szöveg helyigénye miatt a címke -- és ezzel együtt az üzenetablak -- vízszintes helyigénye aránytalanul megnőjön, ami esetleg ahhoz vezethet, hogy az ablak kilóg a munkaterületről, így annak egyes funkció -- legrosszabb esetben a bezáró gomb -- elérhetetlenné váljanak.

Erre természetesen több módszer is kínálkozik. Magunk is megadhatunk a szövegben tördelést azáltal, hogy sortörést teszünk a szövegbe. Ez egyszerűbb estekben megteszi, de nem túl elegáns megoldás, mivel nem számol a megjelenítendő címke számára aktuálisan rendelkezésre álló hellyel, vagyis ha a címkét tartalmazó ablak méretét növeljük, a sortörés helye nem változik, a szöveg nem használja ki a rendelkezésre álló helyet, csökkenteni pedig csak addig lehet az ablak méretét, amíg el nem érjük a leghosszabb sor minimális helyigényét. A probléma akkor igazán szembetűnő, ha címke szövege nem statikus, hanem valamilyen felhasználótól származó adat (pl: objektum neve, IP cím, \dots ), vagy valamilyen módszerrel szűr lista (pl: hibás elemek listája) szerepel benne.

Célravezető a \textit{GTK}, konkrétabban a \textit{Pango} által nyújtott kész megoldás használata. Ez lehetővé teszi a címke különböző módokon történő automatikus tördelését. Ehhez először is engedélyeznünk kell ezt a funkciót (\texttt{set\_line\_wrap}), másrészről választanunk kell tördelési módot. Ez utóbbi alapértelmezetten a szóhatárokon való tördelés (\texttt{PANGO\_WORD}), ami helyett a \texttt{set\_line\_wrap\_mode} függvény meghívásával beállíthatunk karakterenkénti (\texttt{PANGO\_WORD}), illetve vegyes (\texttt{PANGO\_WORD\_CHAR}), ami azt jelenti, hogy egy szó kifér a rendelkezésre álló helyen a sor végén, akkor szóhatáron történik meg a sortörés, ha nem, akkor az adott szóból annyi karaktert teszünk az adott sorba amennyi még oda kifér.

\paragraph{Részben megmutatás}

Ha a rendelkezésre álló hely csekély és a címke valamely részletéből (eleje, vége, esetleg mindkettő együtt) következtethetünk a címke teljes szövegére\footnote{bizonyos elnevezési konvenciók esetén a nevek első része mindig azonos, vagyis ez nem hordoz érdemben információt, így helyszűke esetén érdemes csak a szövegek végét megmutatni}, akkor elegendő lehet csak a ténylegesen információt hordozó részt megmutatnunk. A feleslegesnek ítélt részeket pedig úgymond kihagyhatjuk.

Ez a kihagyás a gyakorlatban úgy történik, hogy megadjuk, melyik az a része a címkének (eleje, közepe, vagy vége) amit elhagyhatónak ítélünk és hely szűkében a tényleges szöveg helyett csak a kihagyás jelző karakter (ellipsis: '\dots ') írunk ki. A beállításra szolgáló függvény értelemszerűen a \texttt{set\_ellipsize}, paramétere pedig az elhanyagolás pozíciója, ami lehet a címke eleje (\texttt{PANGO\_ELLIPSIZE\_START}), közepe (\texttt{PANGO\_ELLIPSIZE\_MIDDLE}), vége (\texttt{PANGO\_ELLIPSIZE\_END}), illetve módunk van kikapcsolni ezt a funkciót (\texttt{PANGO\_ELLIPSIZE\_NONE}).

\paragraph{Szélesség}

Az előző két funkció használatakor két tulajdonság révén kontrollálhatjuk a címke szövegének szélességét. Az egyik (\texttt{width-chars}) a szöveg kívánt szélességét adja meg karakterekben, míg a másik (\texttt{max-width-cars}) révén a maximális szélességet határozhatjuk meg szintén karakterekben. Mindkét tulajdonság esetén megadható \texttt{-1}, mint szélesség, ami automatikus szélességkalkulációt jelent, ami érthető okoknál fogva az alapértelmezett érték.

\subsection{Kezelés}

\subsubsection{\texttt{GtkLabel}}

\paragraph{Kijelölés}

Alapértelmezetten a címkék nem kijelölhetőek és ez a működés a legtöbb esetben meg is felel a kívánalmaknak. Vannak azonban olyan esetek, ahol kifejezetten zavaró, ha a szövegeket nem lehet kijelölni és ezzel együtt másolni sem. Ilyen eset például, amikor hibaüzeneteket jelenítünk meg címkék segítségével. Az ehhez hasonló esetekben a felhasználó számára roppant bosszantó, hogy bár a hibaüzenet látható, mégsem lehet az egyszerűen átmásolni például egy hibabejelentő űrlapra. Ennek elkerülésére a címkét kijelölhetővé tehetjük (\texttt{set\_selectable}, ám célszerű ezt csak akkor megtenni, ha a címke fontos és manuálisan nehézkesen reprodukálható információt tartalmaz, mivel a kijelölhetőség egyben fókuszálhatóságot is jelent, ami billentyűzetről való kezelést megnehezíti.

\section{Haladó műveletek}

\subsection{\texttt{GtkLabel}}

\paragraph{Hivatkozás}

A címkék formázásának egy speciális esete, amikor hivatkozást szeretnénk a szövegben elhelyezni. A \textit{HTML} esetén használatos módszert alkalmazhatjuk a \textit{Pango} leíró nyelv esetén is, vagyis a hivatkozás szövege \texttt{<a>} nyitó-, illetve \texttt{</a>} záróelem között helyezkedik el, míg a hivatkozás a \texttt{href} attribútum értékeként adható meg. A hivatkozások épp úgy viselkednek, mint ahogy azt a böngészők esetén megszoktuk, vagyis a hivatkozás szövege aláhúzott, színe pedig megváltozik a hivatkozás első aktiválása után.

A hivatkozás aktiválásának eseményét magunk is kezelhetjük, a \texttt{activate-link} szignál segítségével. A kezelőfüggvényben a kívánt műveletsor hajtható végre, annak függvényében, hogy a címke melyik hivatkozása került aktiválásra, amit az értéket a kezelőfüggvény paramétereként is megkapunk. A függvény visszatérési értéke -- hasonlóan a \texttt{delete-event} szignálhoz -- azt fejezi, ki, hogy az eseményt kezeltük-e. Ezért az ott leírtak ennek az szignálnak a kezelésénél is alkalmazhatóak.

\subsection{\texttt{GtkTooltip}}

\paragraph{Testre szabott súgóablak}

\subparagraph{Formázás}

A korábban leírt formázó nyelv a súgóablakok szövegében is alkalmazható. A \texttt{GtkWidget} osztály \texttt{set\_toolip\_markup} függvénye paraméterként ilyen leírónyelvű szöveget vesz át paraméterként és annak megfelelően formázza a súgóablakban megjelenő szöveget.

\subparagraph{Saját ablak}

Amennyiben ennél is tovább szeretnénk menni -- esetleg képeket helyeznénk el a súgóablakban -- akkor saját súgóablakra van szükségünk. Ezt a saját ablakot a korábban már ismertetett módszerekkel hozhatjuk létre és abban gyakorlatilag bármit elhelyezhetünk. A megjelenítésről,  eltüntetésről, illetve ezek időzítéséről továbbra is a \textit{GTK} gondoskodik, nekünk csak a tartalmat kell biztosítanunk. Az elkészült ablak beállítható a \texttt{GtkWidget} osztály \texttt{set\_tooltip\_window} nevű függvényével, aminek a beállítandó ablak a paramétere, vagy a nyelvnek megfelelő \texttt{NULL} érték, ami az alapértelmezett ablak visszaállítását jelenti. Amennyiben a szokásos sárga súgóablak témát szeretnénk viszontlátni, az ablak nevét \texttt{gtk-tooltip} értékre kell állítanunk, amit könnyedén megtehetünk a \texttt{GtkWidget} osztály \texttt{set\_name} függvényével.

\section{Tesztelés}

\subsection{Objektum}

A \textit{Dogtail} \texttt{Node} típusa által reprezentált objektum az amin keresztül gyakorlatilag minden információ elérésére lehetőségünk van, amire a tesztelés során szükségünk lehet. Számos esetben közvetlenül az objektum függvényének meghívása, vagy attribútum kiolvasása révén, más esetekben valamilyen interfészen keresztül. Előbbiekből néhányat veszünk a továbbikban sorra.

\paragraph{Keresés}

Az ebben a részben tárgyalt \textit{widget}típusok közül a \texttt{GtkTooltip} a legegyszerűbb és egyszersmind a leggyakoribb esetben nem valódi típus, mint ahogy azt a fejlesztésről szóló részben tárgyaltuk, ennek okán a tesztelés során sem jelenik meg külön típusként, kiolvasására \texttt{Node} objektumból közvetlenül van lehetőség. A \texttt{GtkLabel} és a \texttt{GtkImage} típusú objektumok az akadálymentesítés szempontjából betöltött szerepének neve (\textit{roleName}) \texttt{label}, illetve \texttt{icon}. Az ilyen típusú elemek keresésénél tehát ezeket az értékeket kell megadni a \texttt{roleName} paraméternek.

A másik keresési lehetőség, hogy az objektum nevére keresünk. A címke esetén az objektum 

\paragraph{Súgószöveg}

Az egyes \textit{widget}hez tartozó súgóbuborék szövegét rendkívül egyszerűen megállapíthatjuk, ez a \\texttt{Node} objektum \texttt{description} attribútumának értéke.

\subsection{Állapotok}

\subsubsection{\texttt{GtkLabel}}

\paragraph{Többsoros szöveg}

%TODO ref
A címkék alapvetően alkalmasak többsoros megjelenítésre, így egy állapot ezen \textit{widget}ek kapcsán bizonyos lesz mégpedig a \texttt{MULTI\_LINE}, amit a korábbiakban ismertetetteknek megfelelően a \texttt{getState} függvénnyel tudunk lekérdezni. Létezik egy másik -- ezzel ellentétes értelmű -- státusz is, ami a címkék esetén természetesen sosem lesz része az állapothalmaznak, ez pedig a \texttt{SIGNLE\_LINE}.

\paragraph{Fókuszálhatóság}

A címkék kapcsán fontos lehet, hogy a szöveget ki lehessen jelölni, ami egyúttal magával vonja, hogy az adott \textit{widget} fókuszálhatóvá is válik, ami alapértelmezés szerint nem igaz. Ezt a működést tehát ellenőrizni is tudjuk a megfelelő állapoton (\texttt{FOCUSABLE}) meglétén keresztül.

\subsection{Interfészek}

\subsubsection{Csak olvasható szöveg}

A csak olvasható szövegek kezelésére az \textit{ATK} egy külön interfészt definiált (\texttt{AtkText}). Az olyan elemekhez, amik az akadálymentesítés és éppúgy a tesztelés szempontjából is csak olvasni szokás, az \textit{ATK} megvalósító implementáció csak olvasási hozzáférést ad, annak ellenére is, hogy természetesen megvalósítható lenne, az írást biztosító implementáció is. Ilyen elemek például a címkék, a táblázatos megjelenítést biztosító \textit{widget}ek egyes cellái, illetve minden olyan elem, ami írható hozzáférést is ad, mint például egy egysoros beviteli mező.

Az interfészen keresztül nem csupán a szöveget magát, de annak egyes részeit, illetve az egyes részek formázási paramétereit is elérhetjük. A \texttt{queryText} függvénnyel lekérdezhető interfész objektum számos szolgáltatást nyújt, bár ezek jelentékeny részére -- mint amilyen például a szövegrészek kezelése -- a címkék során nem, vagy csak nagyon ritkán lesz szükségünk, így ezekkel majd egy későbbi részben foglalkozunk. A legfontosabb információ természetesen maga a szöveg, esetenként annak hossza, előbbi a \texttt{getText} függvényen, utóbbi a \texttt{characterCount} attribútum révén érhető el. A \texttt{getText} függvény esetén fontos körülmény, hogy az két kötelező paraméterrel is rendelkezik, a kíván szöveg kezdeti- és végpozíciójának indexével, ahol a második paraméter esetén a $-1$ érték a szöveg végét jelenti. Mivel teljes szöveg a leggyakrabban érdeklődésre számot tartó érték, a \textit{Dogtail} ehhez a \textit{text} interfészt, illetve annak működését elrejtő elérést is biztosít a \texttt{Node} osztály \texttt{text} tagja révén, aminek kiolvasása a \texttt{queryText().getText(0, -1)} szerinti hívásával közel egyenértékű. A különbség az, hogy \texttt{text} attribútum kiolvasása kezeli azt a helyzetet, hogy az adott objektum nem implementálja a \textit{text} interfészt, ilyen esetben \texttt{None} értékkel tér vissza, míg ilyen esetekben a \texttt{queryText} függvényhívás -- minden más interfész lekérdezéséhez hasonlóan -- \texttt{NotImplementedError} kivételt dob.

A szöveg, amit a \texttt{text} interfészen keresztül kiolvashatunk csak a nyers szöveget tartalmazza aze egyszerű kezelés érdekében annak formázási paramétereit nem. Ha azt szeretnénk megtudni, hogy az egyes karakterek miként vannak formázva, akkor a \textit{text} interfész \texttt{getAttributeRun} függvényét használhatjuk, aminek egyetlen paraméter a karakter sorszáma, visszatérési értéke pedig egy lista, ami első eleme egy újabb lista, ami a karakter formázási paramétereinek tartalmazza, másik két eleme pedig annak szövegrésznek a kezdő és végpozíciója, ami ugyanezen paraméterekkel lett formázva.

\subsubsection{Kép}

A képek kezelésére létezik egy külön interfész az \textit{ATK} függvénykönyvtárban, amihez tartozó implementációt a \textit{Dogatil} egy \texttt{Node} objektumára a \texttt{queryImage} függvénnyel kérdezhetünk le. Az interfész sajnos nem nyújt különösebben széles körű szolgáltatásokat, mindössze a kép méretét, elhelyezkedését és leírását áll módunkban lekérdezni. Sajnálatosan a \textit{Dogatil} ehhez nem sok segítsége nyújt, ezért is kell az \textit{Image} interfészt elkérnünk és azt magunknak kezelnünk. Ez az interfész viszont már átvisz minket a \textit{Dogtail} alatt meghúzó \textit{AT SPI}\footnote{a rövidítés az angol \textit{Assistive Technology Service Provider Interface} kifejezést takarja} világába, aminek részleteiben nem célunk elmerülni, így itt csak azokat a hívásokat ismertetjük, amikre a képek tesztelésénél szükségünk lehet.

\index{AtkObject@\texttt{AtkObject}}
\index{AtkObject@\texttt{AtkObject}!függvények!set\_image\_description@\texttt{set\_image\_description}}
%TODO ref to at spi in intro
Az előbb említett paraméterek közül a legegyszerűbb a kép mérete, amit a \texttt{getImageDescription} paraméter nélküli függvény hívása révén kérdezhetünk le, ami az \textit{x}, illetve \textit{y} irányú pixelben vett méretek listáját adja vissza. A pozíció lekérdezése is teljesen hasonlóan működik, méret helyett az \textit{x}, \textit{y} koordinátákat visszakapva a listában, viszont a pozíció maga relatív és hogy mire relatív azt a \texttt{getImagePosition} függvénynek paraméterként kell átadnunk. Ez a paraméter lehet a \texttt{pyatspi} modul \texttt{WINDOW\_COORDS}, vagy \texttt{DESKTOP\_COORDS} értéke, aminek megfelelően vagy az ablakhoz, vagy magához a munkaterülethez képest értelmezendőek a koordináták. Lehetőség vagy a két értékpár együttes lekérdezésére is a \texttt{getImageExtents} függvénnyel, ami ugyanazt a paraméter veszi át, mint a \texttt{getImagePosition} és szintén egy listával tér vissza, aminek első két eleme a relatív koordinátákat, második két elem pedig a méreteket tartalmazza. Ezen értékekből messzemenő következtetéseket levonni nem lehet. Különösen akkor vagyunk bajban ha két azonos méretű ikon között szeretnénk különbséget tennünk a tesztelés során, ami nem ritka példa, hiszen szokás különböző színű ikonokkal mondjuk valamilyen állapotot jelezni. Ebben az esetben mentsvárunk a kép leírása lehet, amit a \texttt{Node} \texttt{imageDescription} értéke tartalmaz és amit alkalmazásunk fejlesztésekor az \texttt{AtkObject} \texttt{set\_image\_description} függvénnyel állíthatunk be.

\subsection{Viszonyok}

Az egyes \textit{widget}ek között bizonyos vonatkozásokban összefüggések állhatnak fenn. Ezeket az összefüggéseket -- amiket szabad fordításban nevezhetünk viszonyoknak (\textit{relation}) --,  a \textit{Dogtail} segítségével is le tudjuk kérdezni. Túlnyomó többségben nem is viszonyokról, hanem viszonypárokról beszélünk, amik között $1:1$ és $1:N$ típusú kapcsolatok egyaránt vannak.

\index{Atspi.Accessible@\texttt{Atspi.Accessible}}
Általánosságban az mondható el, hogy minden egyes \texttt{Accessible} objektum többféle viszonyban is lehet, több más objektummal. Első körben tehát azt tudhatjuk meg, hogy mik azok a viszonyok, amikkel összekapcsolják az adott objektumot más objektumokkal. Ennek lekérdezésére a \texttt{getRelationSet} függvény szolgál, ami egy konténert ad vissza, benne a viszonyok leírására szolgáló \texttt{Atspi.Relation} objektumokkal. Ezen objektumokból kideríthető a viszony típusa a \texttt{getRelationType} függvény hívásával, valamint, hogy a hány objektummal áll fenn az adott viszony, amit a \texttt{getNTargets} függvény visszatérési értéke ad meg. Az $N$ darab objektum egyesével a \texttt{getTarget} függvény révén érhető el paraméterként egy sorszámot átadva.

Az egyik viszony, a címkére és az általa felcímkézett \textit{widget}re vonatkozik, ahol természetesen több címke is vonatkozhat ugyanarra \textit{widget}re, viszont egy címke egyszerre csak egy \textit{widget}et vonatkozhat. Ez következik a \texttt{GtkLabel} osztály \texttt{set\_mnemonic\_widget} függvény használatából is, hiszen paraméterként csak egy \textit{widget} adható meg. A viszony típusa ebben az irányban \texttt{pyatspi.RELATION\_LABEL\_FOR}, ahol tehát a \texttt{getNTargets} függvény mindig egyet ad vissza, míg az ellenkező irányban a viszony típusa \texttt{pyatspi.RELATION\_LABELLED\_BY}, ahol több cél is lehetséges, de ez nem jellemző. Erre a gyakran használt viszonypárra a \texttt{Node} osztály is ad megoldást, a \textit{relation} interfész közvetlen használatánál számottevően egyszerűbbet. Egy adott \texttt{Node} címkéinek listája a \texttt{labeller}, az adott \texttt{Node} által címkézett objektum pedig a \texttt{labelee} attribútum keresztül érhető el.


\appendix

\chapter{Licencelési feltételek}

Ez a mű a Creative Commons \textit{Nevezd meg!-Így add tovább!} licencének hatálya alatt áll.

\paragraph{A következőket teheted a művel:}

\begin{itemize}
 \item szabadon másolhatod
 \item terjesztheted
 \item bemutathatod és előadhatod a művet 
 \item származékos műveket (feldolgozásokat) hozhatsz létre
\end{itemize}

\paragraph{Az alábbi feltételekkel:}

\begin{description}
 \item[Nevezd meg!] A szerző vagy a jogosult által meghatározott módon fel kell tüntetned a műhöz kapcsolódó információkat (pl. a szerző nevét vagy álnevét, a Mű címét).
 \item[Így add tovább!] Ha megváltoztatod, átalakítod, feldolgozod ezt a művet, az így létrejött alkotást csak a jelenlegivel megegyező licenc alatt terjesztheted.
\end{description}

\paragraph{Az alábbi figyelembevételével:}

\begin{description}
 \item[Elengedés] A szerzői jogok tulajdonosának engedélyével bármelyik fenti feltételtől \href{http://creativecommons.org/licenses/by-sa/2.5/hu/#}{eltérhetsz}.
 \item[Más jogok] A következő jogokat a licenc semmiben nem befolyásolja:
 \begin{itemize}
  \item A fentiek nem befolyásolják \href{http://wiki.creativecommons.org/Frequently_Asked_Questions#Do_Creative_Commons_licenses_affect_fair_use.2C_fair_dealing_or_other_exceptions_to_copyright.3F}{a szabad felhasználáshoz fűződő}, illetve az egyéb jogokat.
  \item A szerző \href{http://wiki.creativecommons.org/Frequently_Asked_Questions#I_don.E2.80.99t_like_the_way_a_person_has_used_my_work_in_a_derivative_work_or_included_it_in_a_collective_work.3B_what_can_I_do.3F}{személyhez fűződő} jogai
  \item Más személyeknek a művet vagy a mű használatát érintő jogai, mint például a \href{http://wiki.creativecommons.org/Frequently_Asked_Questions#When_are_publicity_rights_relevant.3F}{személyiségi jogok} vagy az adatvédelmi jogok.
 \end{itemize}
 \item[Jelzés] Bármilyen felhasználás vagy terjesztés esetén egyértelműen jelezned kell mások felé ezen mű licencfeltételeit. 
\end{description}

Ez a Legal Code (jogi változat, vagyis a teljes licenc) szövegének közérthető nyelven megfogalmazott kivonata, teljes változata a Creative Commons \href{http://creativecommons.org/licenses/by-sa/2.5/hu/legalcode}{oldalán} érhető el.

\newpage
\addcontentsline{toc}{chapter}{Tárgymutató}
\printindex

\newpage
\addcontentsline{toc}{chapter}{Táblázatok jegyzéke}
\listoftables

\newpage
\addcontentsline{toc}{chapter}{Ábrák jegyzéke}
\listoffigures

\newpage
\addcontentsline{toc}{chapter}{\lstlistlistingname}
\lstlistoflistings

\newpage
\addcontentsline{toc}{chapter}{Irodalomjegyzék}
\bibliography{book}
\bibliographystyle{plain}

\end{document}
