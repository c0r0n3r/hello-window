\section{Bevezetés}

A cikksorozat második részében az első alkalommal már bemutatott példaprogramokat újra felhasználva a \textit{GTK+}/\textit{gtkmm} programozás azon területeit vesszük sorra, melyek nélkül rövid távon el lehet boldogulni, de nélkülözhetetlenek az alapos megértést igénylő feladatok megoldásához.

\subsection{Alapfogalmak}

Még mielőtt ismertetnénk néhány alapfogalmat érdemes kitérni arra, hogy ezen túl a \textit{GTK} rövidítést akkor használjuk, ha a felületprogramozási nyelvről általánosságban szólunk, míg a \textit{GTK+} és \textit{gtkmm} kifejezések a konkrét \textit{C}, illetve \textit{C++} nyelvű implementációkat jelölik.

\subsubsection{Objektum-orientált megközelítés}

\begin{description}
 \item[Öröklődés] Annak ellenére, hogy a mechanizmust nyelvi szinten a \textit{C} nem, csak a \textit{C++} támogatja igen is lehetséges objektum-orientált megközelítéssel élni az első esetben is. Erre kitűnő példa -- egyebek mellett -- a \textit{GTK+}. Megoldott a \textit{widget}ek egymásból történő származtatása, sőt felhasználói widgetek is definiálhatóak a már meglévőekre támaszkodva. Meg kell jegyezni, hogy a \textit{gtkmm} esetén -- lévén ott a nyelv \textit{C++} -- természetesen a származtatás egy nagyságrenddel egyszerűbb, de a letölthető példákat felhasználva némi rutinnal a \textit{GTK+} esetén sem igényel különösebb erőfeszítést.

 \item[Típusbiztosság] Hasonlóan a korábbiakhoz pusztán nyelvi szinten ez az eszköz sem megvalósítható (\textit{C} esetén), ugyanakkor a \textit{GTK+} minden \textit{widget}típushoz -- mondhatni osztályhoz -- definiál egy-egy makrót, melyek segítségével, fordítási időben (compile time) ugyan nem, de futásidőben (run time) ellenőrizhető egy adott widget valódi típus, hasonlóan ahhoz, mint amire a \textit{dynamic\_cast} használata jelent a \textit{C++}-ban.
\end{description}

Fentieket figyelembe véve láthatjuk, hogy a \textit{GTK+} ugyan \textit{C} nyelven íródott, de számos -- az objektum-orientált -- nyelvek esetén megszokott terminológiát használ, sőt ezeket a nyelvi eszközök adta mértékben meg is valósítja. Az \textit{OOP} kifejezéseit ezért tudatosan használom az olyan esetekben is, ahol \textit{GTK+} nyelvű fejlesztésről esik szó.

\subsubsection{Felületprogramozási kulcsszavak}

\begin{description}
 \item[Widget] A kifejezést, mint gyűjtőfogalmat használjuk a \textit{GTK} programozás során a felhasználó felület egyes grafikai elemeinek megnevezésére. Az elnevezés egyébiránt az \textit{X} hagyományokból eredeztethető. A szó azonban nem csak erre szolgál, hanem annak az ősosztálynak a neve is -- még ha ez a fogalom \textit{C} nyelvben nem is létezik -- melyből minden egyes megjeleníthető elem -- gomb, menü, csúszka -- származik.

 \item[GTK Main Loop] Ez a \textit{GTK} tulajdonképpeni főciklusa, mely a \textit{Glib}ben implementál általános \textit{main loop}ot felhasználva csatlakozik a \textit{X} szerverhez. Mindezt a \textit{GDK}-n keresztül teszi, mely -- mint azt az előző részben is említettük -- egy burkoló réteg az ablakozó rendszer köré. Ezt azért fontos kiemelni, mert ezzel a módszerrel biztosítható, hogy a \textit{GTK} működőképes legyen különböző grafikus szerverek (\textit{X}, \textit{framebuffer}) és platformokon (\textit{GNU/Linux}, \textit{Windows}, \textit{Mac OS X}) futtatása mellett egyaránt. A \textit{main loop} tehát az aki az imént említett kapcsolaton át eljuttatja az ablakozó rendszer alacsony szintű eseményeit a \textit{GDK} által standardizált formában az egyes \textit{widget}ekhez.

 \item[Signal] A \textit{GTK}, és egyébiránt minden más felületprogramozási nyelv, egyik kulcsszava. Jelentése (jel, jelzés) jól tükrözi funkcióját. Minden \textit{widget}hez tartoz(hat)nak különböző események -- mint amilyen egy gomb esetén annak lenyomása (vagy éppen felengedése), egy beviteli mezőnél az abba történő írás -- melyekről a widgetek tudomást szereznek -- egyébként a \textit{main loop}on keresztül -- elvégzik a megfelelő műveleteket -- gomb lenyomásnál újrarajzolás, beviteli mezőbe írásnál a karakter megjelenítése -- majd értesítést küldenek a program többi része felé. Ezt az értesítést, avagy jelzést nevezzük \textit{signal}nek.

 \item[Callback] Amennyiben egy adott \textit{widget} által küldött meghatározott eseményről tudomást akarunk szerezni a program futása során, ezt megtehetjük azáltal, hogy a \textit{widget} megfelelő \textit{signal}jéhez egy \textit{callback}et, azaz eseménykezelő függvény társítunk. Itt minden olyan esemény elvégezhető, mely nem a \textit{widget}hez, hanem annak programunkban betöltött szerepéhez kötődik. A korábbi példánál maradva ez egy \textit{Szerkesztés} gomb lenyomásánál lehet egy dialógus ablak megjelenítése, egy beviteli mezőbe való írásnál -- amennyiben ez szükséges -- a tartalom ellenőrzése.
\end{description}

Az utóbbi két fogalom, mivel a \textit{GUI}-k fejlesztése tulajdonképpen eseményvezérelt programozás, kiemelt fontosságú. Ennek okán erről a következő részben részletesebben is kívánunk szólni. Elöljáróban csak annyit, hogy a \textit{GTK} lehetőséget ad a \textit{signal}ek \textit{widgetek}től teljesen független használatára is, azaz akár saját osztályainkhoz is rendelhetünk eseményeket. Ez azonban már túlmutat ennek a résznek a keretein...

\subsubsection{A megvalósítás}

Az előző fejezetekben olvasható metodikák és módszerek közül a \textit{GTK} \texttt{GObject}, avagy \texttt{Glib::Object} osztálya az alábbiakat valósítja meg:

\begin{description}
 \item[Öröklődés] Az \textit{oject} osztály típus minden \textit{widget}, és egy más a \textit{GTK}-ban használt nem vizuális elem őse.

 \item[Típusbiztosság] Ez az osztály implementálja azt a mechanizmust, melynek segítségével lehetővé válik a \textit{GTK+}-ban a futás idejű típusellenőrzés.

 \item[Signal] Ezen osztályon keresztül valósul meg a szignálkezelés, mely lehetővé teszi adott eseményekhez kezelőfüggvények (\textit{callback}) kapcsolását. Itt kell megjegyezni, hogy mivel az object nem csupán a \textit{widget}eknek őse, így olyan \textit{GTK}-s, vagy akár saját, elemeknek is lehetnek szignáljai, melyek közvetlenül nem láthatóak, mint az általunk létrehozott \textit{GUI} részei.

 \item[Referencia-számlálás] Minden \textit{object}ből származó osztály, így a widgetek is, rendelkeznek referencia-számmal, mely tulajdonképpen azt fejezi ki, hogy hányan hivatkoznak az adott elemre. A \textit{GTK}, pontosabban ez esetben a \textit{GLib} lebegő referenciát (\textit{floating reference}) alkalmaz, mely azt jelenti, hogy az objektum létrejöttekor annak referenciája 1 lesz, de ezt a referenciát úgymond nem birtokolja senki, azaz ha a \textit{widget}et egy konténer osztályba (melyekről részletesebben a következő rész szól majd) kívánjuk tenni, akkor -- az első ilyen alkalommal -- a refernciaszám nem nő, annak ellenére sem, hogy ez valójában hivatkozást jelent az adott elemre. A változatlanul hagyott referencia-érték jelenti a lebegő referencia elsüllyesztését (\textit{sink}). Minden azt követő esetben a konténerből történő eltávolítás csökkenti, ahoz való hozzáadás pedig növeli a referencia értékét. Érdemes felhívni a figyelmet arra, hogy az elmondottak alapján, ha hozzáadtuk \textit{widget}ünket egy \textit{container}hez, majd pedig eltávolítjuk belőle azt, akkor annak referenciája 0-ra csökken, ami maga után vonja a destruktor lefutását. Ezt elkerülhetjük, ha az eltávolítás előtt explicit módon növeljük a referenciát, amit aztán csökkentenünk kell, ha egy másik osztály ``birtokába'' adjuk a \textit{widget}et.
\end{description}

\section{Első ablakunk háttere}

\subsection{A kód}
\label{sec:gtkminimal}
\label{sec:gtkmmminimal}

Az ismétlés kedvéért -- no meg persze, hogy kéznél legyen -- lássuk most az előző számban már bemutatott példaprogramokat.

\vspace{16pt}
\fontsize{8pt}{8pt}
\begin{verbatim}
1  #include <gtk/gtk.h>                             #include <gtkmm.h>
2
3  int main(int argc, char *argv[])                 int main(int argc, char *argv[])
4  {                                                {
5    GtkWidget *window;
6
7    gtk_init (&argc, &argv);                         Gtk::Main kit(argc, argv);
8
9    window = gtk_window_new (GTK_WINDOW_TOPLEVEL);   Gtk::Window window;
10   gtk_widget_show  (window);
11
12   gtk_main ();                                     Gtk::Main::run(window);
13
14   return 0;                                        return 0;
15 }                                                }
\end{verbatim} 

A fenti példaprogramok \textit{C/C++} nyelvű forrásfájljai, illetve azok eredetijei, a \textit{FLOSSzine}, valamint a \textit{GTK+}/\textit{gtkmm} oldalain az alábbi linkeken érhetőek el:
\ \\\\
\url{http://www.flosszine.org/sources/gtk_window.c}\\
\url{http://www.flosszine.org/sources/gtkmm_window.cc}\\
\url{http://library.gnome.org/devel/gtk-tutorial/stable/c39.html}\\
\url{http://www.gtkmm.org/docs/gtkmm-2.4/docs/tutorial/html/chapter-basics.html}

\subsection{Részletek sorról sorra}

Vegyük a példaprogramokat szó szerint sorról sorra górcső alá.

\begin{enumerate}
 \item[1] A \textit{header} fájlok beszerkesztésének különbözőségeiről az előző részben esett szó, így erre itt nem térnünk ki.

 \item[3] A \texttt{main} függvény a programunk belépési pontja, azaz itt kezdődik meg a futtatás. A függvény paramétereiről részletesebben Vomberg István ``Hello world!'' című cikksorozatának második részében lehet olvasni.

 \item[5] A \textit{C} nyelvi verzióban kénytelenek vagyunk kicsit korábban deklarálni azt a változót, ami ebben az esetben \textit{widget}ünk címét tartalmazza majd, mivel az \textit{ISO C90} szabvány még nem, majd csak az \textit{ISO C99} támogatja a blokkon belül a kifejezés után elhelyezett változódeklarációkat. Megjegyzendő, hogy \textit{3.0}-nál gorábbi \textit{gcc} esetén is problémába ütköznénk a \textit{C++} példában használt módszerrel, így a \textit{C} nyelvű kódokban a biztonság kedvéért maradunk a hagyományoknál, azaz lokális változók deklarációja csak blokkok kezdetén szerepel.

 \item[7] Eljutottunk végre az első \textit{GTK} specifikus híváshoz, mely funkcionalitásában, azonos mégis van köztük árnyalatnyi különbség. A \textit{GTK+}-os verzióban mind az \texttt{argc}, mind az \texttt{argv} változó címét adjuk át, biztosítandó azt, hogy az init függvény a \textit{GTK}-nak szóló paramétereket el tudja távolítani a tömbböl és azok számával csökkenteni tudja \texttt{argc} értékét. Erre a \textit{C++}-os változat esetén csak azért nincs szükség, mert -- még ha nem is látszik -- mindkét változóra referencia adódik át a \texttt{Gtk::Main} konstruktorának. A \textit{GTK} által értelmezett parancssori paramétereket foglalja össze a http://library.gnome.org/devel/gtk/stable/gtk-running.html oldal.

 \item[9] Első \textit{widget}ünk létrehozása. A \textit{C}-s, illetve \textit{C++}-os nevezéktannak megfelelően létezik minden \textit{widget}típushoz létezik konstruktor, előbbi esetben \texttt{gtk\_}\textit{widgetnév}\texttt{\_new} függvény, míg utóbbi esetben \texttt{Gtk::}\textit{WidgetNév}\texttt{::}\textit{WidgetNév} konstruktor formájában. A különbség nem is annyira az nevekben, mintsem a memóriakezelésben rejlik, hiszen egy újjolag allokált objektumot kapunk \textit{C} esetén, aminek a felszabadításáról magunknak kell gondoskodnunk, míg a \textit{C++} a tőle megszokott módon felszabadítja a lokális változókat. Ezen a helyzeten csak a korábban már említett lebegő referencia bonyolít avagy egyszerűsít némiképp.

 \item[10] Némi meglepetés érhet bennünket, ezt a sort látván, hiszen első olvasatra nem teljesen nyilvánvaló, hogy miért is van szükség a \textit{C} nyelvű változat esetén az megjelenítő függvény explicit hívására, és hol marad ez a \textit{C++} változatból.

 \item[12] A megoldás a két -- egyéb tekintetekben egyforma -- függvény különbségében rejlik. Ez pedig az, hogy a \texttt{Gtk::Main::run} a paraméterként kapott \textit{widget} (jelen esetben \textit{window}) esetén meghívja annak \texttt{show} metódusát. Az azonosságról annyit, hogy a hívások a -- korábban már részletezett -- \textit{GTK main loop}ot indítják, azaz itt kezdődik meg az az eseményvezérelt szakasz, mely a \texttt{gtk\_main\_quit}, illetve \texttt{Gtk::Main::quit} hívásokkal ér véget.

 \item[14] Visszatérési értékünk mindkét esetben 0, amivel rendszerint azt jelezzük a hívó félnek, hogy a futás rendben lezajlott.
\end{enumerate}

\subsection{Vágyaink ablaka}

\subsubsection{Fordítás és linkelés}

A korábbiakhoz hasonlóan az alábbi parancssorok segítségével fordíthatóak e\-lem\-zett programjaink:

\fontsize{8pt}{8pt}
\ \\
\texttt{gcc gtk\_window.c -o gtk\_window \`{}pkg-config {-}-cflags {-}-libs gtk+-2.0}
\ \\
\texttt{g++ gtkmm\_window.cc -o gtkmm\_window \`{}pkg-config gtkmm-2.4 {-}-cflags {-}-libs\`{}}

\subsubsection{Futtatás}

Próbálkozzunk ezúttal is a \texttt{./gtk\_window}, illetve a \texttt{./gtkmm\_window} paranccsokkal abban a könyvtárban, ahol a fordítást elkövettük.

\subsubsection{Eredmény}

Bármily hihetetlen ezúttal sem történik semmi egyéb, mint az előző alkalommal. Remélhetőleg azonban a különbség mégis érzékelhető annyiban, hogy legutóbb a meglepetéssel teli borzongást ablakunk váratlan felbukkanása, míg most a bennünk szikraként felvillanó megértés okozza.
