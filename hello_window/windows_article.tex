\documentclass[a4paper,titlepage,10pt]{article}

\usepackage[utf8]{inputenc}
\usepackage[hungarian]{babel}
\usepackage[T1]{fontenc}
\usepackage{pslatex}
\usepackage{listings}
\usepackage{textcomp}
\usepackage{url}

\title{Helló Window! - GTK+/gtkmm programozás GNU/Linux alatt}
\author{Pfeiffer Szilárd}
\date{\today}

\begin{document}

\pagenumbering{roman}

\begin{titlepage}

\maketitle

\begin{abstract}
A \textit{GTK+} (\textit{GIMP Toolkit}) egy \textit{C} nyelven -- ám objektum-orientált megközelítéssel -- íródott, grafikus felhasználói felületek (\textit{GUI}) létrehozására használatos alkalmazás-programozási interfész. A \textit{gtkmm} nem más, mint ennek a függvénykönyvtárnak a \textit{C++} változata, pontosabban fogalmazva \textit{wrapper}e. Mindkét függvénykönyvtár terjesztése \textit{LGPL} licenc alatt történik, így bátran felhasználható mind szabad/ingyenes, mind kereskedelmi szoftverek létrehozására.

A most kezdődő sorozat célja az abszolút kezdetektől indulva bemutatni a \textit{GTK+} és a \textit{gtkmm} hasonlóságait, különbözőségeit, sajátosságait eljutva egy olyan szintre, ahol remélhetőleg a több éves tapasztalattal rendelkező fejlesztők is találnak hasznos, megfontolásra érdemes ötleteket, információkat.
\end{abstract}

\tableofcontents

\end{titlepage}

\pagenumbering{arabic}

\documentclass[a4paper,10pt]{article}

\usepackage[latin2]{inputenc}
\usepackage[hungarian]{babel}
\usepackage{t1enc}
\usepackage{listings}
\usepackage{textcomp}
\usepackage{url}

%opening
\title{Hello Window! - GTK+/gtkmm programoz�s GNU/Linux alatt}
\author{Pfeiffer Szil�rd}
\date{\today}

\begin{document}

\maketitle

\begin{abstract}
A \textit{GTK+} (\textit{GIMP Toolkit}) egy \textit{C} nyelven -- �m objektum-orient�lt megk�zel�t�ssel -- �r�dott, grafikus felhaszn�l�i fel�letek (\textit{GUI}) l�trehoz�s�ra haszn�latos alkalmaz�s-programoz�si interf�sz. A \textit{gtkmm} nem m�s, mint ennek a f�ggv�nyk�nyvt�rnak a \textit{C++} v�ltozata, pontosabban fogalmazva \textit{wrapper}e. Mindk�t f�ggv�nyk�nyvt�r terjeszt�se \textit{LGPL} licenc alatt t�rt�nik, �gy b�tran felhaszn�lhat� mind szabad/ingyenes, mind kereskedelmi szoftverek l�trehoz�s�ra.

A most kezd�d� sorozat c�lja az abszol�t kezdetekt�l indulva bemutatni a \textit{GTK+} �s a \textit{gtkmm} hasonl�s�gait, k�l�nb�z�s�geit, saj�toss�gait eljutva egy olyan szintre, ahol rem�lhet�leg a t�bb �ves tapasztalattal rendelkez� fejleszt�k is tal�lnak hasznos, megfontol�sra �rdemes �tleteket, inform�ci�kat.
\end{abstract}

\newpage

\tableofcontents

\newpage

\section{Bevezet�s}

A cikksorozat m�sodik r�sz�ben az els� alkalommal m�r bemutatott p�ldaprogramokat �jra felhaszn�lva a \textit{GTK+}/\textit{gtkmm} programoz�s azon ter�leteit vessz�k sorra, melyek n�lk�l r�vid t�von el lehet boldogulni, de n�lk�l�zhetetlenek az alapos meg�rt�st ig�nyl� feladatok megold�s�hoz.

\subsection{Alapfogalmak}

M�g miel�tt ismertetn�nk n�h�ny alapfogalmat �rdemes kit�rni arra, hogy ezen t�l a \textit{GTK} r�vid�t�st akkor haszn�ljuk, ha a fel�letprogramoz�si nyelvr�l �ltal�noss�gban sz�lunk, m�g a \textit{GTK+} �s \textit{gtkmm} kifejez�sek a konkr�t \textit{C}, illetve \textit{C++} nyelv� implement�ci�kat jel�lik.

\subsubsection{Objektum-orient�lt megk�zel�t�s}

\begin{description}
 \item[�r�kl�d�s] Annak ellen�re, hogy a mechanizmust nyelvi szinten a \textit{C} nem, csak a \textit{C++} t�mogatja igen is lehets�ges objektum-orient�lt megk�zel�t�ssel �lni az els� esetben is. Erre kit�n� p�lda --egyebek mellett-- a \textit{GTK+}. Megoldott a \textit{widget}ek egym�sb�l t�rt�n� sz�rmaztat�sa, s�t felhaszn�l�i widgetek is defini�lhat�ak a m�r megl�v�ekre t�maszkodva. Meg kell jegyezni, hogy a \textit{gtkmm} eset�n --l�v�n ott a nyelv \textit{C++}-- term�szetesen a sz�rmaztat�s egy nagys�grenddel egyszer�bb, de a let�lthet� p�ld�kat felhaszn�lva n�mi rutinnal a \textit{GTK+} eset�n sem ig�nyel k�l�n�sebb er�fesz�t�st.

 \item[T�pusbiztoss�g] Hasonl�an a kor�bbiakhoz puszt�n nyelvi szinten ez az eszk�z sem megval�s�that� (\textit{C} eset�n), ugyanakkor a \textit{GTK+} minden \textit{widget}t�pushoz --mondhatni oszt�lyhoz-- defini�l egy-egy makr�t, melyek seg�ts�g�vel, ford�t�si id�ben (compile time) ugyan nem, de fut�sid�ben (run time) ellen�rizhet� egy adott widget val�di t�pus, hasonl�an ahhoz, mint amire a \textit{dynamic\_cast} haszn�lata jelent a \textit{C++}-ban.
\end{description}

Fentieket figyelembe v�ve l�thatjuk, hogy a \textit{GTK+} ugyan \textit{C} nyelven �r�dott, de sz�mos --az objektum-orient�lt-- nyelvek eset�n megszokott terminol�gi�t haszn�l, s�t ezeket a nylevi eszk�z�k adta m�rt�kben meg is val�s�tja. Az \textit{OOP} kifejez�seit ez�rt tudatosan haszn�lom az olyan esetekben is, ahol \textit{GTK+} nyelv� fejleszt�sr�l esik sz�.

\subsubsection{Fel�letprogramoz�si kulcsszavak}

\begin{description}
 \item[Widget] A kifejez�st, mint gy�jt�fogalmat haszn�ljuk a \textit{GTK} programoz�s sor�n a felhaszn�l� fel�let egyes grafikai elemeinek megnevez�s�re. Az elnevez�s egy�bir�nt az \textit{X} hagyom�nyokb�l eredeztethet�. A sz� azonban nem csak erre szolg�l, hanem annak az �soszt�lynak a neve is --m�g ha ez a fogalom \textit{C} nyelvben nem is l�tezik-- melyb�l minden egyes megjelen�thet� elem --gomb, men�, cs�szka-- sz�rmazik.

 \item[GTK Main Loop] Ez a \textit{GTK} tulajdonk�ppeni f�ciklusa, mely a \textit{Glib}ben implement�l �ltal�nos \textit{main loop}ot felhaszn�lva csatlakozik a \textit{X} szerverhez. Mindezt a \textit{GDK}-n kereszt�l teszi, mely --mint azt az el�z� r�szben is eml�tett�k-- egy burkol� r�teg az ablakoz� rendszer k�r�. Ezt az�rt fontos kiemelni, mert ezzel a m�dszerrel biztos�that�, hogy a \textit{GTK} m�k�d�k�pes legyen k�l�nb�z� grafikus szerverek (\textit{X}, \textit{framebuffer}) �s platformokon (\textit{GNU/Linux}, \textit{Windows}, \textit{Mac OS X}) futtat�sa mellett egyar�nt. A \textit{main loop} teh�t az aki az im�nt eml�tett kapcsolaton �t eljuttatja az ablakoz� rendszer alacsony szint� esem�nyeit a \textit{GDK} �ltal standardiz�lt form�ban az egyes \textit{widget}ekhez.

 \item[Signal] A \textit{GTK}, �s egy�bir�nt minden m�s fel�letprogramoz�si nyelv, egyik kulcsszava. Jelent�se (jel, jelz�s) j�l t�kr�zi funkci�j�t. Minden \textit{widget}hez tartoz(hat)nak k�l�nb�z� esem�nyek --mint amilyen egy gomb eset�n annak lenyom�sa (vagy �ppen felenged�se), egy beviteli mez�n�l az abba t�rt�n� �r�s-- melyekr�l a widgetek tudom�st szereznek --egy�bk�nt a \textit{main loop}on kereszt�l-- elv�gzik a megfelel� m�veleteket --gomb lenyom�sn�l �jrarajzol�s, beviteli mez�be �r�sn�l a karakter megjelen�t�se-- majd �rtes�t�st k�ldenek a program t�bbi r�sze fel�. Ezt az �rtes�t�st, avagy jelz�st nevezz�k \textit{signal}nek.

 \item[Callback] Amennyiben egy adott \textit{widget} �ltal k�ld�tt meghat�rozott esem�nyr�l tudom�st akarunk szerezni a program fut�sa sor�n, ezt megtehetj�k az�ltal, hogy a \textit{widget} megfelel� \textit{signal}j�hez egy \textit{callback}et, azaz esem�nykezel� f�ggv�ny t�rs�tunk. Itt minden olyan esem�ny elv�gezhet�, mely nem a \textit{widget}hez, hanem annak programunkban bet�lt�tt szerep�hez k�t�dik. A kor�bbi p�ld�n�l maradva ez egy \textit{Szerkeszt�s} gomb lenyom�s�n�l lehet egy dial�gus ablak megjelen�t�se, egy beviteli mez�be val� �r�sn�l --amennyiben ez sz�ks�ges-- a tartalom ellen�rz�se.
\end{description}

Az ut�bbi k�t fogalom, mivel a \textit{GUI}-k fejleszt�se tulajdonk�ppen esem�nyvez�relt programoz�s, kiemelt fontoss�g�. Ennek ok�n err�l a k�vetkez� r�szben r�szletesebben is k�v�nunk sz�lni. El�lj�r�ban csak annyit, hogy a \textit{GTK} lehet�s�get ad a \textit{signal}ek \textit{widgetek}t�l teljesen f�ggetlen haszn�lat�ra is, azaz ak�r saj�t oszt�lyainkhoz is rendelhet�nk esem�nyeket. Ez azonban m�r t�lmutat ennek a r�sznek a keretein...

\subsubsection{A megval�s�t�s}

Az el�z� fejezetekben olvashat� metodik�k �s m�dszerek k�z�l a \textit{GTK} \texttt{GObject}, avagy \texttt{Glib::Object} oszt�lya az al�bbiakat val�s�tja meg:

\begin{description}
 \item[�r�kl�d�s] Az \textit{oject} oszt�ly t�pus minden \textit{widget}, �s egy m�s a \textit{GTK}-ban haszn�lt nem vizu�lis elem �se.

 \item[T�pusbiztoss�g] Ez az oszt�ly implement�lja azt a mechanizmust, melynek seg�ts�g�vel lehet�v� v�lik a \textit{GTK+}-ban a fut�s idej� t�pusellen�rz�s.

 \item[Signal] Ezen oszt�lyon kereszt�l val�sul meg a szign�lkezel�s, mely lehet�v� teszi adott esem�nyekhez kezel�f�ggv�nyek (\textit{callback}) kapcsol�s�t. Itt kell megjegyezni, hogy mivel az object nem csup�n a \textit{widget}eknek �se, �gy olyan \textit{GTK}-s, vagy ak�r saj�t, elemeknek is lehetnek szign�ljai, melyek k�zvetlen�l nem l�that�ak, mint az �ltalunk l�trehozott \textit{GUI} r�szei.

 \item[Referencia-sz�ml�l�s] Minden \textit{object}b�l sz�rmaz� oszt�ly, �gy a widgetek is, rendelkeznek referencia-sz�mmal, mely tulajdonk�ppen azt fejezi ki, hogy h�nyan hivatkoznak az adott elemre. A \textit{GTK}, pontosabban ez esetben a \textit{GLib} lebeg� referenci�t (\textit{floating reference}) alkalmaz, mely azt jelenti, hogy az objektum l�trej�ttekor annak referenci�ja 1 lesz, de ezt a referenci�t �gymond nem birtokolja senki, azaz ha a \textit{widget}et egy kont�ner oszt�lyba (melyekr�l r�szletesebben a k�vetkez� r�sz sz�l majd) k�v�njuk tenni, akkor --az els� ilyen alkalommal-- a refernciasz�m nem n�, annak ellen�re sem, hogy ez val�j�ban hivatkoz�st jelent az adott elemre. A v�ltozatlanul hagyott referencia-�rt�k jelenti a lebeg� referencia els�llyeszt�s�t (\textit{sink}). Minden azt k�vet� esetben a kont�nerb�l t�rt�n� elt�vol�t�s cs�kkenti, ahoz val� hozz�ad�s pedig n�veli a referencia �rt�k�t. �rdemes felh�vni a figyelmet arra, hogy az elmondottak alapj�n, ha hozz�adtuk \textit{widget}�nket egy \textit{container}hez, majd pedig elt�vol�tjuk bel�le azt, akkor annak referenci�ja 0-ra cs�kken, ami maga ut�n vonja a destruktor lefut�s�t. Ezt elker�lhetj�k, ha az elt�vol�t�s el�tt explicit m�don n�velj�k a referenci�t, amit azt�n cs�kkenten�nk kell, ha egy m�sik oszt�ly ``birtok�ba'' adjuk a \textit{widget}et.
\end{description}

\section{Els� ablakunk h�ttere}

\subsection{A k�d}

Az ism�tl�s kedv��rt --no meg persze, hogy k�zn�l legyen-- l�ssuk most az el�z� sz�mban m�r bemutatott p�ldaprogramokat.

\vspace{16pt}
\fontsize{8pt}{8pt}
\begin{verbatim}
1  #include <gtk/gtk.h>                             #include <gtkmm.h>
2
3  int main(int argc, char *argv[])                 int main(int argc, char *argv[])
4  {                                                {
5    GtkWidget *window;
6
7    gtk_init (&argc, &argv);                         Gtk::Main kit(argc, argv);
8
9    window = gtk_window_new (GTK_WINDOW_TOPLEVEL);   Gtk::Window window;
10   gtk_widget_show  (window);
11
12   gtk_main ();                                     Gtk::Main::run(window);
13
14   return 0;                                        return 0;
15 }                                                }
\end{verbatim} 

A fenti p�ldaprogramok \textit{C/C++} nyelv� forr�sf�jljai, illetve azok eredetijei, a \textit{FLOSSzine}, valamint a \textit{GTK+}/\textit{gtkmm} oldalain az al�bbi linkeken �rhet�ek el:
\ \\\\
\url{http://www.flosszine.org/sources/gtk_window.c}\\
\url{http://www.flosszine.org/sources/gtkmm_window.cc}\\
\url{http://library.gnome.org/devel/gtk-tutorial/stable/c39.html}\\
\url{http://www.gtkmm.org/docs/gtkmm-2.4/docs/tutorial/html/chapter-basics.html}

\subsection{R�szletek sorr�l sorra}

Vegy�k a p�ldaprogramokat sz� szerint sorr�l sorra g�rcs� al�.

\begin{enumerate}
 \item[1] A \textit{header} f�jlok beszerkeszt�s�nek k�l�nb�z�s�geir�l az el�z� r�szben esett sz�, �gy erre itt nem t�rn�nk ki.

 \item[3] A \texttt{main} f�ggv�ny a programunk bel�p�si pontja, azaz itt kezd�dik meg a futtat�s. A f�ggv�ny param�tereir�l r�szletesebben Vomberg Istv�n ``Hello world!'' c�m� cikksorozat�nak m�sodik r�sz�ben lehet olvasni.

 \item[5] A \textit{C} nyelvi verzi�ban k�nytelenek vagyunk kicsit kor�bban deklar�lni azt a v�ltoz�t, ami ebben az esetben \textit{widget}�nk c�m�t tartalmazza majd, mivel az \textit{ISO C90} szabv�ny m�g nem, majd csak az \textit{ISO C99} t�mogatja a blokkon bel�l a kifejez�s ut�n elhelyezett v�ltoz�deklar�ci�kat. Megjegyzend�, hogy \textit{3.0}-n�l gor�bbi \textit{gcc} eset�n is probl�m�ba �tk�zn�nk a \textit{C++} p�ld�ban haszn�lt m�dszerrel, �gy a \textit{C} nyelv� k�dokban a biztons�g kedv��rt maradunk a hagyom�nyokn�l, azaz lok�lis v�ltoz�k deklar�ci�ja csak blokkok kezdet�n szerepel.

 \item[7] Eljutottunk v�gre az els� \textit{GTK} specifikus h�v�shoz, mely funkcionalit�s�ban, azonos m�gis van k�zt�k �rnyalatnyi k�l�nbs�g. A \textit{GTK+}-os verzi�ban mind az \texttt{argc}, mind az \texttt{argv} v�ltoz� c�m�t adjuk �t, biztos�tand� azt, hogy az init f�ggv�ny a \textit{GTK}-nak sz�l� param�tereket el tudja t�vol�tani a t�mbb�l �s azok sz�m�val cs�kkenteni tudja \texttt{argc} �rt�k�t. Erre a \textit{C++}-os v�ltozat eset�n csak az�rt nincs sz�ks�g, mert --m�g ha nem is l�tszik-- mindk�t v�ltoz�ra referencia ad�dik �t a \texttt{Gtk::Main} konstruktor�nak. A \textit{GTK} �ltal �rtelmezett parancssori param�tereket foglalja �ssze a http://library.gnome.org/devel/gtk/stable/gtk-running.html oldal.

 \item[9] Els� \textit{widget}�nk l�trehoz�sa. A \textit{C}-s, illetve \textit{C++}-os nevez�ktannak megfelel�en l�tezik minden \textit{widget}t�pushoz l�tezik konstruktor, el�bbi esetben \texttt{gtk\_}\textit{widgetn�v}\texttt{\_new} f�ggv�ny, m�g ut�bbi esetben \texttt{Gtk::}\textit{WidgetN�v}\texttt{::}\textit{WidgetN�v} konstruktor form�j�ban. A k�l�nbs�g nem is annyira az nevekben, mintsem a mem�riakezel�sben rejlik, hiszen egy �jjolag allok�lt objektumot kapunk \textit{C} eset�n, aminek a felszabad�t�s�r�l magunknak kell gondoskodnunk, m�g a \textit{C++} a t�le megszokott m�don felszabad�tja a lok�lis v�ltoz�kat. Ezen a helyzeten csak a kor�bban m�r eml�tett lebeg� referencia bonyol�t avagy egyszer�s�t n�mik�pp.

 \item[10] N�mi meglepet�s �rhet benn�nket, ezt a sort l�tv�n, hiszen els� olvasatra nem teljesen nyilv�nval�, hogy mi�rt is van sz�ks�g a \textit{C} nyelv� v�ltozat eset�n az megjelen�t� f�ggv�ny explicit h�v�s�ra, �s hol marad ez a \textit{C++} v�ltozatb�l.

 \item[12] A megold�s a k�t --egy�b tekintetekben egyforma-- f�ggv�ny k�l�nbs�g�ben rejlik. Ez pedig az, hogy a \texttt{Gtk::Main::run} a param�terk�nt kapott \textit{widget} (jelen esetben \textit{window}) eset�n megh�vja annak \texttt{show} met�dus�t. Az azonoss�gr�l annyit, hogy a h�v�sok a --kor�bban m�r r�szletezett-- \textit{GTK main loop}ot ind�tj�k, azaz itt kezd�dik meg az az esem�nyvez�relt szakasz, mely a \texttt{gtk\_main\_quit}, illetve \texttt{Gtk::Main::quit} h�v�sokkal �r v�get.

 \item[14] Visszat�r�si �rt�k�nk mindk�t esetben 0, amivel rendszerint azt jelezz�k a h�v� f�lnek, hogy a fut�s rendben lezajlott.
\end{enumerate}

\subsection{V�gyaink ablaka}

\subsubsection{Ford�t�s �s linkel�s}

A kor�bbiakhoz hasonl�an az al�bbi parancssorok seg�ts�g�vel ford�that�ak e\-lem\-zett programjaink:

\fontsize{8pt}{8pt}
\ \\
\texttt{gcc gtk\_window.c -o gtk\_window \`{}pkg-config {-}-cflags {-}-libs gtk+-2.0}
\ \\
\texttt{g++ gtkmm\_window.cc -o gtkmm\_window \`{}pkg-config gtkmm-2.4 {-}-cflags {-}-libs\`{}}

\subsubsection{Futtat�s}

Pr�b�lkozzunk ez�ttal is a \texttt{./gtk\_window}, illetve a \texttt{./gtkmm\_window} paranccsokkal abban a k�nyvt�rban, ahol a ford�t�st elk�vett�k.

\subsubsection{Eredm�ny}

B�rmily hihetetlen ez�ttal sem t�rt�nik semmi egy�b, mint az el�z� alkalommal. Rem�lhet�leg azonban a k�l�nbs�g m�gis �rz�kelhet� annyiban, hogy legut�bb a meglepet�ssel teli borzong�st ablakunk v�ratlan felbukkan�sa, m�g most a benn�nk szikrak�nt felvillan� meg�rt�s okozza.

\nocite{gtktut}
\nocite{gtkmmtut}
\nocite{ggad}
\nocite{gtktutmagy}

\addcontentsline{toc}{section}{Hivatkoz�sok}
\bibliography{windows}
\bibliographystyle{plain}


\end{document}
 


\nocite{gtktut}
\nocite{gtkmmtut}
\nocite{ggad}
\nocite{gtktutmagy}

\addcontentsline{toc}{section}{Hivatkozások}
\bibliography{windows}
\bibliographystyle{plain}

\end{document}
