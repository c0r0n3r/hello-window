\section{\textit{Gimp Tool Kit}}

\subsection{Beszerzése}
\index{GTK}
%FIXME

A \textit{GTK+} beszerzésére alapvetően két módszer kínálkozik. Az egyik megoldás, hogy hagyatkozunk az általunk használt operációs rendszerre és az általa biztosított, vagy legalábbis arra elérhető változatot telepítjük. A másik lehetőség, hogy letöltjük a \textit{GTK+} és a függőségek forráskódját és némi nehézséget vállalva ezeket fordítjuk le. Előbbi eset bőségesen megfelel amennyiben még csak most ismerkedünk a \textit{GTK+} függvénykönyvtárral, illetve nem akarunk túlságosan belebonyolódni az grafikus alkalmazások fejlesztésébe. Elkerülhetetlenül az utóbbi változatot kell választanunk, ha jobban nis el szeretnénk merülni a \textit{GTK+} rejtelmeiben, ha esetleg hibákat javítanánk, vagy változásokat eszközölnénk magán a grafikus eszközkészleten.

\subsubsection{Bináris változat}
\index{GTK}
\index{deb}
\index{rpm}
\index{bináris csomag}

A \textit{GTK} változatok beszerzése nem jelent különösebben feladatot, amennyiben valamelyik népszerű \textit{Linux} disztribúciót használjuk, hiszen azok nagy valószínűséggel már amúgy is telepítve vannak az általunk használt rendszeren, lévén vélhetőleg már használunk ezen eszközök segítségével fejlesztett szoftvereket. A fejlesztéshez, illetve teszteléshez a bináris változatokon túl fejlesztői csomagokra is szükséges lesz, amit \textit{rpm}, illetve \textit{deb} alapú disztribúciók esetén rendre az alábbi parancsok kiadásával tehetünk meg:

\lstcommand{sudo yum install gtk-devel-package}\\
\lstcommand{sudo apt-get install gtk-dev-package}

\paragraph{GTK+}
\index{nyelvi változatok!GTK+@\textit{GTK+}}
\index{C}

A \textit{GTK+} fejlécfájlokat és egyéb állományokat -- melyekről a későbbiekben (\ref{sec:compilingandlinking}) még részletesebben is esik szó -- a \textit{Debian}/\textit{Ubuntu}, illetve \textit{Fedora} rendszereken a \texttt{libgtk-3-dev}, illetve a \texttt{gtk3-devel} csomagok tartalmazzák.

\paragraph{gtkmm}
\index{nyelv változatok!gtkmm@\textit{gtkmm}}
\index{C++}

A fenti csomagok természetesen csak a \textit{C} nyelvű változat -- azaz a \textit{GTK+} -- használatához elegendőek, amennyiben a \textit{C++} nyelvet -- ezzel együtt a \textit{gtkmm} függvénykönyvtárat -- kívánjuk használni, további csomagokra (\texttt{libgtkmm-3.0-dev}, vagy \texttt{gtkmm3-devel}) is szert kell tennünk.

\paragraph{PyGObject}
\index{nyelv változatok!PyGObject}
\index{Python}

Amennyiben a \textit{GTK} alapú fejlesztéssel a \textit{Python} nyelv révén ismerkednénk a már említett fordított nyelvek helyett, akkor a fenti csomagokat nem, a \texttt{python-gtk2}, vagy a \texttt{pygtk2} csomagokat viszont be kell szereznünk.

\paragraph{Dogtail}
\index{Dogtail@\textit{Dogtail}}
\index{Python}

Az automata tesztek készítéséhez szükséges \textit{Python} függvénykönyvtár --a \textit{Dogtail}-- állományait az azonos nevű csomag tartalmazza, melyek installálása a fentiekhez hasonlóan történik.

\subsubsection{Forráskód}

A \textit{GTK} forrásának beszerzésére több módszer is kínálkozik. Egyrészről az általunk használt \textit{Linux} disztribúció biztosít eszközöket forráscsomagok telepítésére. A korábbi \textit{deb}, illetve \textit{rpm} alapú rendszerek példájánál maradva ez rendre az alábbiak szerint történik.

\lstcommand{apt-get source gtk-src-package}\\
\lstcommand{yumloader --source gtk-src-package}\\

Lehetőség van természetesen az egyes verziók letöltésére a \textit{GNOME} projekt weboldaláról is,

\lstcommand{wget http://ftp.gnome.org/pub/gnome/sources/gtk+/major.minor/gtk+-major.minor.micro.tar.gz}\\
\lstcommand{wget http://ftp.gnome.org/pub/gnome/sources/gtk+/major.minor/gtk+-major.minor.micro.tar.bz2}\\

valamint rendelkezésre állnak az egyes projektek verziókelői is, ha szeretnénk mindig a legfrissebb forrás mellett dolgozni.

\lstcommand{git clone git://git.gnome.org/gtk+}\\

Ha azonban magunk szeretnénk a teljes \textit{GTK}-t fordítani --legyen szó a \textit{C}, vagy a \textit{C++} nyelvű változatról-- számolnunk kell azzal, hogy számos egyéb komponens (\textit{GLib}, \textit{Pango}, \textit{Cairo}, \textit{ATK}, \dots) fordítására, illetve az frissítéseket követő újrafordítására válik szükségessé, ami meglehetősen időigényes és fáradságos feladat, amit a \textit{Linux} disztribúción összeállítói már megtettek helyettünk. Így célszerű kezdetben ezt kihasználni és a saját fordításba csak akkor belekezdeni, ha arra feltétlenül szükségünk van.

\subsection{Függőségei}
\index{GDK@\textit{GDK}}
\index{GLib@\textit{GLib}}

\subsubsection{GTK+}


%    The GTK library (-lgtk), the widget library, based on top of GDK.
%
%    The GDK library (-lgdk), the Xlib wrapper.
%
%    The gdk-pixbuf library (-lgdk_pixbuf), the image manipulation library.
%
%    The Pango library (-lpango) for internationalized text.
%
%    The gobject library (-lgobject), containing the type system on which GTK is based.
%
%    The gmodule library (-lgmodule), which is used to load run time extensions.
%
%    The GLib library (-lglib), containing miscellaneous functions; only g_print() is used in this particular example. GTK is built on top of GLib so you will always require this library. See the section on GLib for details.
%
%    The Xlib library (-lX11) which is used by GDK.
%
%    The Xext library (-lXext). This contains code for shared memory pixmaps and other X extensions.
%
%    The math library (-lm). This is used by GTK for various purposes.

\subsubsection{gtkmm}

\paragraph{libsigc++}

\subsubsection{pygtk}

\subsubsection{dogtail}

\subsection{Fordítása}

\subsubsection{Linux}
\index{Autotools@\textit{Autotools}}
\index{Automake@\textit{Automake}}
\index{Autoconf@\textit{Autoconf}}
\index{Libtool@\textit{Libtool}}
%FIXME

Amennyiben a korábban említett nehézségek ellenére mégis nekivágunk a \textit{GTK} fordításának, mindenekelőtt szükségünk lesz néhány eszközre, melyek a \textit{Linux} alapú rendszereken szokványosnak nevezhetők. A \textit{GNOME} projekt --sok más fejlesztési projekthez hasonlóan-- az \textit{Autotools}t\footnote{a \textit{GNU} fordítási rendszere, mely tulajdonképpen az \textit{Automake}, \textit{Autoconf}, \textit{Libtool} együttese.} használja moduljainak fordításához, melynek részleteibe nem célunk elmerülni, ugyanakkor néhány mondatban mégis érdemes megjegyezni az \textit{Autotools} egyes részeiről.

\paragraph{Autoconf}
\paragraph{Automake}
\paragraph{Libtool}

célszerű tisztában lennünk azzal, hogy mire szeretnénk felhasználni az így létrejött keretrendszerrel (bináris-, fejlécfájlok, egyéb fejlesztői eszközök), mivel a különböző igények különböző megoldási módozatokat kívánnak. Amennyiben --bár ez egy kevéssé valószínű eset-- a rendszeren lévő \textit{GTK} állományokat szeretnénk helyettesíteni, akkor kissé másként kell eljárnunk, mint ha egy alternatív változatot szeretnénk lefordítani, amit csak egyes esetekben szeretnénk használni.




Előbbi esetben a \textit{GTK} egyes moduljainak fordításakor egy alkalmas könyvtárat\footnote{a különböző \textit{Linux} alapú rendszereken erre jellemzően a \texttt{/usr/local} könyvtár a megfelelő} kell megadnunk 

\lstinputfile
{bash}
{Fordítási környezet kialakítása}
{lst:compileenv}
{sources/compile_env}

\subsubsection{Windows}

\subsection{Ellenőrzése}
%FIXME

\section{Saját alkalmazások}
%FIXME

\subsection{Fordítás és linkelés}
\label{sec:compilingandlinking}
%FIXME

Az alábbi parancssorok segítségével fordíthatóak elkészült programjaink:

\lstcompiles
{gtk\_sourcefile.c}{gtk\_binary}
{gtkmm\_sourcefile.cc}{gtkmm\_binary}

Segítségünkre a \texttt{pkg-config} parancs van, hogy a \textit{gcc}-nek a megfelelő paramétereket meg tudjuk adni. A \texttt{--cflags} paraméter hatására a fordításhoz, míg a \texttt{--libs} eredményeképp a linkeléshez szükséges opciókat kapjuk vissza. A parancs két \texttt{\`} (backtick) közé zárt. aminek hatsára annak kimenete része lesz a fordító parancssorának, amivel pont az áhytott hatást érjük el.

\subsection{Futtatás}

Ezel után már csak az örömteli pillanat van hátra, mikor is két különböző nyelven és függvénykönyvtárral lekódolt teljesen azonos funkciójú programunkat lefuttatjuk a \texttt{./gtk\_window}, illetve a \texttt{./gtkmm\_window} paranccsal.

\subsection{Eredmény}

Voil\`{a}! Túlvagyunk két ablak --vagyis egy Windows-- lekódolásán. Ami eddi is közismert volt, újabb bizonyságot nyert, a Windows kódja lehet, hogy elegáns és egyszerű. ám az eredmény mégis hasznavehetetlen. Az utóbbin fogunk segíteni a következő részekben.
