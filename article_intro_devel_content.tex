\section{\textit{Gimp Tool Kit}}

\subsection{Beszerzése}
\index{GTK}
%FIXME

A \textit{GTK+} beszerzésére alapvetően két módszer kínálkozik. Az egyik megoldás, hogy hagyatkozunk az általunk használt operációs rendszerre elérhető \textit{GTK+} változatra és azt telepítjük. A másik lehetőség, hogy letöltjük a \textit{GTK+} és a függőségek forráskódját és némi nehézséget vállalva ezeket fordítjuk le. Előbbi eset bőségesen megfelel amennyiben még csak most ismerkedünk a \textit{GTK+}-val, illetve nem akarunk túlságosan belebonyolódni az grafikus alkalmazások fejlesztésébe. Elkerülhetetlenül az utóbbi változatot kell választanunk, ha mélyebben is szeretnénk megismerkedni a \textit{GTK+} rejtelmeivel, ha komolyabb hibákat szeretnénk javítani, vagy szeretnénk változásokat szeretnénk eszközölni magán a grafikus függvénykönyvtáron.

\subsubsection{Bináris változat}
\index{Debian}
\index{Linux}
\index{Fedora}
\index{Ubuntu}
\index{GTK}

A \textit{GTK} nyelv változatok beszerzése nem jelent különösebben nehéz feladatot amennyiben valamelyik népszerű \textit{Linux} disztribúciót használjuk, hiszen azok nagy valószínűséggel már amúgy is telepítve vannak az általunk használt rendszeren, lévén vélhetőleg már amúgy is használunk ezen eszközök segítségével fejlesztett szoftvereket. A fejlesztéshez, illetve teszteléshez egyéb csomagokat telepítése is szükséges lesz, amit a disztribúciónktól függően az alábbi parancsokkal tehetünk meg:

\lstcommand{%
sudo apt-get install gtk-deb-package\linebreak
sudo yum install gtk-rpm-package
}

\paragraph{GTK+}
\index{GTK+}

A \textit{GTK+} fejlécfájlokat és egyéb állományokat --melyekről a későbbiekben (\ref{sec:compilingandlinking}) még részletesebben is esik szó-- a \textit{Debian}/\textit{Ubuntu}, illetve \textit{Fedora} rendszereken a \texttt{libgtk-3-dev}, illetve a \texttt{libgtk3-devel} csomagok tartalmazzák.

\paragraph{gtkmm}
\index{gtkmm}

A fenti csomagok természetesen csak a \textit{C} nyelvű változat --azaz a \textit{GTK+}-- használatához elegendőek, amennyiben a \textit{C++} nyelvet --ezzel együtt a \textit{gtkmm} függvénykönyvtárat-- kívánjuk használni, további csomagokra (\texttt{libgtk-3-dev}, vagy \texttt{libgtk3-devel}) is szert kell tennünk.

\paragraph{PyGTK}
\index{PyGTK}
\index{Python}

Amennyiben a \textit{GTK} alapú fejlesztéssel a \textit{Python} nyelv révén ismerkednénk a már említett fordított nyelvek helyett, akkor a fenti csomagokra nem, a \texttt{python-gtk2}, vagy a \texttt{pygtk2} csomagokat viszont be kell szereznünk.

\paragraph{Dogtail}
\index{Dogtail}
\index{Python}

Az automata tesztek készítéséhez szükséges \textit{Python} függvénykönyvtár --a \texttt{Dogtail}-- állományait az azonos nevű csomag tartalmazza, melyek installálása a fentiekhez hasonlóan történik.

\subsubsection{Forráskód}



\subsection{Függőségei}
%FIXME

\subsection{Fordítása}
%FIXME

\subsection{Ellenőrzése}
%FIXME

\section{Saját alkalmazások}
%FIXME

\subsection{Fordítás és linkelés}
\label{sec:compilingandlinking}
%FIXME

Az alábbi parancssorok segítségével fordíthatóak elkészült programjaink:

\lstcompiles
{gtk\_sourcefile.c}{gtk\_binary}
{gtkmm\_sourcefile.cc}{gtkmm\_binary}

Segítségünkre a \texttt{pkg-config} parancs van, hogy a \textit{gcc}-nek a megfelelő paramétereket meg tudjuk adni. A \texttt{--cflags} paraméter hatására a fordításhoz, míg a \texttt{--libs} eredményeképp a linkeléshez szükséges opciókat kapjuk vissza. A parancs két \texttt{\`} (backtick) közé zárt. aminek hatsára annak kimenete része lesz a fordító parancssorának, amivel pont az áhytott hatást érjük el.

\subsection{Futtatás}

Ezel után már csak az örömteli pillanat van hátra, mikor is két különböző nyelven és függvénykönyvtárral lekódolt teljesen azonos funkciójú programunkat lefuttatjuk a \texttt{./gtk\_window}, illetve a \texttt{./gtkmm\_window} paranccsal.

\subsection{Eredmény}

Voil\`{a}! Túlvagyunk két ablak --vagyis egy Windows-- lekódolásán. Ami eddi is közismert volt, újabb bizonyságot nyert, a Windows kódja lehet, hogy elegáns és egyszerű. ám az eredmény mégis hasznavehetetlen. Az utóbbin fogunk segíteni a következő részekben.
