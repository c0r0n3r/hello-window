\section{\textit{Gimp Tool Kit}}

\subsection{Beszerzése}
\index{GTK}

A \textit{GTK+} beszerzésére alapvetően két módszer kínálkozik. Az egyik megoldás, hogy hagyatkozunk az általunk használt operációs rendszerre és az általa biztosított, vagy legalábbis arra elérhető változatot telepítjük. A másik lehetőség, hogy letöltjük a \textit{GTK+} és a függőségek forráskódját és némi nehézséget vállalva ezeket fordítjuk le. Előbbi eset bőségesen megfelel amennyiben még csak most ismerkedünk a \textit{GTK+} függvénykönyvtárral, illetve nem akarunk túlságosan belebonyolódni az grafikus alkalmazások fejlesztésébe. Elkerülhetetlenül az utóbbit kell azonban választanunk, ha az átlagnál jobban el szeretnénk merülni a \textit{GTK+} rejtelmeiben, ha esetleg hibákat javítanánk, vagy változásokat eszközölnénk magán a grafikus eszközkészleten.

\subsubsection{Bináris változat}
\index{GTK}
\index{deb}
\index{rpm}
\index{bináris csomag}

A \textit{GTK} változatok beszerzése nem jelent különösebb feladatot, amennyiben valamelyik népszerű \textit{Linux} disztribúciót használjuk, hiszen azok nagy valószínűséggel már amúgy is telepítve vannak az általunk használt rendszeren, lévén vélhetőleg már használunk ezen eszközök segítségével fejlesztett szoftvereket. A fejlesztéshez, illetve teszteléshez a bináris változatokon túl fejlesztői csomagokra is szükséges lesz, amit \textit{rpm}, illetve \textit{deb} alapú disztribúciók esetén rendre az alábbi parancsok kiadásával tehetünk meg:

\lstcommand{sudo yum install gtk-devel-package}\\
\lstcommand{sudo apt-get install gtk-dev-package}

\paragraph{GTK+}
\index{nyelvi változatok!GTK+@\textit{GTK+}}
\index{C}

A \textit{GTK+} fejlécfájlokat és egyéb állományokat -- melyekről a későbbiekben (\ref{sec:compilingandlinking}) még részletesebben is esik szó -- a \textit{Debian}/\textit{Ubuntu}, illetve \textit{Fedora} rendszereken a \texttt{libgtk-3-dev}, illetve a \texttt{gtk3-devel} csomagok tartalmazzák.

\paragraph{gtkmm}
\index{nyelv változatok!gtkmm@\textit{gtkmm}}
\index{C++}

A fenti csomagok természetesen csak a \textit{C} nyelvű változat -- azaz a \textit{GTK+} -- használatához elegendőek, amennyiben a \textit{C++} nyelvet -- ezzel együtt a \textit{gtkmm} függvénykönyvtárat -- kívánjuk használni, további csomagokra (\texttt{libgtkmm-3.0-dev}, vagy \texttt{gtkmm3-devel}) is szert kell tennünk.

\paragraph{PyGObject}
\index{nyelv változatok!PyGObject}
\index{Python}

Amennyiben a \textit{GTK} alapú fejlesztéssel a \textit{Python} nyelv révén ismerkednénk a már említett fordított nyelvek helyett, akkor a fenti csomagokat nem, a \texttt{python-gi}, vagy a \texttt{python-gobject} csomagokat viszont be kell szereznünk.

\paragraph{Dogtail}
\index{Dogtail@\textit{Dogtail}}
\index{Python}

Az automata tesztek készítéséhez szükséges \textit{Python} függvénykönyvtár -- a \textit{Dogtail} -- állományait az azonos nevű csomag tartalmazza, melyek installálása a fentiekhez hasonlóan történik.

\subsubsection{Forráskód}

A \textit{GTK} forrásának beszerzésére szintén több módszer kínálkozik. Egyrészről az általunk használt \textit{Linux} disztribúció biztosít eszközöket forráscsomagok telepítésére. A korábbi \textit{deb}, illetve \textit{rpm} alapú rendszerek példájánál maradva ez rendre az alábbiak szerint történik.

\lstcommand{apt-get source gtk-src-package}\\
\lstcommand{yumloader --source gtk-src-package}\\

Lehetőség van természetesen az egyes verziók letöltésére a \textit{GNOME} projekt weboldaláról is,

\lstcommand{wget http://ftp.gnome.org/pub/gnome/sources/gtk+/major.minor/gtk+-major.minor.micro.tar.gz}\\
\lstcommand{wget http://ftp.gnome.org/pub/gnome/sources/gtk+/major.minor/gtk+-major.minor.micro.tar.bz2}\\

valamint használhatóak e célra egyes projektek verziókelői is, ha szeretnénk mindig az aktuális forráskóddal dolgozni.

\lstcommand{git clone git://git.gnome.org/gtk+}\\

Ha azonban magunk szeretnénk a teljes \textit{GTK}-t fordítani -- legyen szó a \textit{C}, vagy a \textit{C++} nyelvű változatról -- számolnunk kell azzal, hogy számos egyéb komponens (\textit{GLib}, \textit{Pango}, \textit{Cairo}, \textit{ATK}, \dots) fordítására, illetve az frissítéseket követő újrafordítására válik szükségessé, ami meglehetősen időigényes és fáradságos feladat, amit a \textit{Linux} disztribúción összeállítói már megtettek helyettünk. Így célszerű kezdetben ezt kihasználni és a saját fordításba csak akkor belekezdeni, ha arra feltétlenül szükségünk.

\subsection{Fordítása}
\index{Autotools@\textit{Autotools}}
\index{Automake@\textit{Automake}}
\index{Autoconf@\textit{Autoconf}}
\index{Libtool@\textit{Libtool}}

Amennyiben a korábban említett nehézségek ellenére mégis nekivágunk a \textit{GTK} saját fordításának, akkor sem vagyunk magunkra hagyva. A \textit{GNOME} projekt része egy \textit{JHBuild} elnevezésű szoftver\footnote{telepítése a korábbiaknak leírtaknak megfelelőn történik}, amit a \textit{GNOME} projekt moduljaiból összeállított halmazok -- mint amilyen a \textit{GTK} és annak függőségei -- letöltésére, frissítésére, fordítására és fordítás eredményeként létrejött futtatási környezetben való munkára használhatunk.

Mindenekelőtt azonban szükségünk lesz néhány olyan eszközre, amik a \textit{Linux} alapú rendszereken fordítási feladatok ellátására kvázi szabványnak számítanak. A \textit{GNOME} -- sok más fejlesztési projekthez hasonlóan -- az \textit{Autotools}t\footnote{a \textit{GNU} fordítási rendszere, mely tulajdonképpen az \textit{Automake}, \textit{Autoconf}, \textit{Libtool} együttese.} használja moduljainak fordításához. Ennek részleteibe nem célunk ezen dokumentum keretében elmerülni, már csak azért sem, mert a \textit{JHBuild} ezen, fordításhoz szükséges, függőségek telepítését megoldja helyettünk.

\lstcommand{jhbuild sanitycheck}\\
\lstcommand{jhbuild bootstrap}

Az parancs futtatásának hatására ellenőrzésre kerül a konfigurációban megadott könyvtárak írhatósága, a szükséges fordítási eszközök telepített mivolta. Ezt követően kerülhet sor a forráskódok letöltésére, a fordítás előtti konfigurálásra, magára  fordításra, valamint az elkészült bináris állományok telepítésére. Ehhez a következő parancsokat használhatjuk.

\lstcommand{jhbuild update}\\
\lstcommand{jhbuild make}

A fordítás végeztével lehetőségünk van az újólag létrejött környezetbe úgymond belépni, vagy közvetlenül parancsokat futtatni. Ez gyakorlatilag ennyit tesz, hogy számos környezeti változó beállításának eredményeként saját a fordítandó alkalmazásaink és a fordítás eredményeként létrejött futtatandó állományok is a az új környezetben létrehozott függvénykönyvtárakat fogják használni a rendszeren található változatuk helyett.

\lstcommand{jhbuild run}\\
\lstcommand{jhbuild shell}

Ennek akkor van leginkább haszna, ha szeretnénk az általunk fejlesztett alkalmazást a rendszeren elérhető \textit{GTK} változaton kívül a legfrissebb verzión is kipróbálni, \textit{GTK} valamely modulján szeretnénk változtatni és ennek hatását látni alkalmazásunkra, vagy éppen csak nyomon követnénk a \textit{GTK} változásaik, aktuális fejlesztéseit még mielőtt azok az általunk használt disztribúcióban is megjelenik.

\section{Saját alkalmazások}

Mivel a továbbiakban részletesen foglalkozunk majd a \textit{GTK} alapú alkalmazások létrehozásával, itt most csak a legfontosabb parancsokat vesszük számba, melyek révén a \textit{C}, illetve \textit{C++} nyelvű forrásfájlokból futtatható bináris állományokat hozhatunk létre.

\subsection{Fordítás és linkelés}
\label{sec:compilingandlinking}

Akár a rendszeren található, akár \textit{JHBuild}, vagy más eszköz révén létrehozott környezetben lévő \textit{GTK} változatot is használunk, az alábbi parancssorok segítségével fordíthatóak le forrásfájljaink.

\lstcompiles
{gtk\_sourcefile.c}{gtk\_binary}
{gtkmm\_sourcefile.cc}{gtkmm\_binary}

Ahhoz, hogy a \textit{GTK+}, illetve \textit{gtkmm} fordítási függőségeit ne magunknak kelljen megadnunk a \texttt{pkg-config} parancsot hívhatjuk segítségül, hogy a \textit{GCC} részére a megfelelő paramétereket meg tudjuk adni. A \texttt{--cflags} paraméter hatására a fordításhoz, míg a \texttt{--libs} eredményeképp a linkeléshez szükséges opciókat kapjuk vissza. A parancs két \texttt{\`} (backtick) közé zárt. aminek hatására a program kimenete része lesz a fordító parancssorának, amivel pont az kívánt hatást érjük el.

\subsection{Futtatás}

Ezek után már csak az örömteli pillanat van hátra, mikor a két különböző nyelven és függvénykönyvtárral lekódolt teljesen azonos funkciójú programunkat lefuttatjuk a \texttt{./gtk\_binary}, illetve a \texttt{./gtkmm\_binary} paranccsal.

Amennyiben a \textit{Python} nyelvű változat mellett tesszük le voksunkat a fordítás, mint lépés kimarad, a futtatás történhet közvetlenül (\texttt{./gtk\_script.py}), amennyiben van az adott fájlon futtatási jog, vagy a \textit{Python} interpreternek paraméterként (\texttt{python gtk\_sctipt.py}). A továbbiakban ezt az utóbbi sémát követjük.

Ezzel túl is vagyunk azon a rövid áttekintésen ami után már épp ideje nekilátnunk első ablakunk implementálásának, futtatásának és tesztelésének.
