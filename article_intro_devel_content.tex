\section{\textit{Gimp Tool Kit}}

\subsection{Beszerzése}
%FIXME

\subsection{Függőségei}
%FIXME

\subsection{Fordítása}
%FIXME

\subsection{Ellenőrzése}
%FIXME

\section{Saját alkalmazások}
%FIXME

\subsection{Fordítás és linkelés}
%FIXME

Az alábbi parancssorok segítségével fordíthatóak elkészült programjaink:

\lstcompiles
{gtk\_sourcefile.c}{gtk\_binary}
{gtkmm\_sourcefile.cc}{gtkmm\_binary}

Segítségünkre a \texttt{pkg-config} parancs van, hogy a \textit{gcc}-nek a megfelelő paramétereket meg tudjuk adni. A \texttt{--cflags} paraméter hatására a fordításhoz, míg a \texttt{--libs} eredményeképp a linkeléshez szükséges opciókat kapjuk vissza. A parancs két \texttt{\`} (backtick) közé zárt. aminek hatsára annak kimenete része lesz a fordító parancssorának, amivel pont az áhytott hatást érjük el.

\subsection{Futtatás}

Ezel után már csak az örömteli pillanat van hátra, mikor is két különböző nyelven és függvénykönyvtárral lekódolt teljesen azonos funkciójú programunkat lefuttatjuk a \texttt{./gtk\_window}, illetve a \texttt{./gtkmm\_window} paranccsal.

\subsection{Eredmény}

Voil\`{a}! Túlvagyunk két ablak --vagyis egy Windows-- lekódolásán. Ami eddi is közismert volt, újabb bizonyságot nyert, a Windows kódja lehet, hogy elegáns és egyszerű. ám az eredmény mégis hasznavehetetlen. Az utóbbin fogunk segíteni a következő részekben.
